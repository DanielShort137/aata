%%%%(c)
%%%%(c)  This file is a portion of the source for the textbook
%%%%(c)
%%%%(c)    Abstract Algebra: Theory and Applications
%%%%(c)    Copyright 1997 by Thomas W. Judson
%%%%(c)
%%%%(c)  See the file COPYING.txt for copying conditions
%%%%(c)
%%%%(c)
\chap{Isomorphisms}{isomorph}

Many groups may appear to be different at first glance, but can be shown to be the same by a simple renaming of the group elements.  For example, ${\mathbb Z}_4$ and  the subgroup of the circle group ${\mathbb T}$ generated by $i$ can be shown to be the same by demonstrating a one-to-one correspondence between the elements of the two groups and between the group operations. In such a case we say that the groups are isomorphic.  


\section{Definition and Examples}\label{isomorph_section_1}

Two groups $(G, \cdot)$ and $(H, \circ)$ are \boldemph{isomorphic}\index{Group!isomorphic} if there exists a one-to-one and onto map $\phi : G \rightarrow H$ such that the group operation is preserved;  that~is, 
\[
\phi( a \cdot b) = \phi( a) \circ \phi( b)
\]
for all $a$ and $b$ in $G$. If $G$ is isomorphic to $H$, we write $G \cong H$\label{noteisomorph}. The map $\phi$ is called an \boldemph{isomorphism}\index{Group!isomorphism of}\index{Isomorphism!of groups}. 

\begin{example}{Z4_isomorph}
To show that ${\mathbb Z}_4 \cong \langle i \rangle$, define a map $\phi: {\mathbb Z}_4 \rightarrow \langle i \rangle$ by $\phi(n) = i^n$.  We must show that $\phi$ is bijective and preserves the group operation.  The map $\phi$ is one-to-one and onto because
\begin{align*}
\phi(0) & = 1 \\
\phi(1) & = i \\
\phi(2) & = -1 \\
\phi(3) & = -i.
\end{align*}
Since
\[
\phi(m + n) = i^{m+n} = i^m i^n = \phi(m) \phi( n),
\]
the group operation is preserved.
\end{example}

\begin{example}{RealIsomorph}
We can define an isomorphism $\phi$ from the additive group of real numbers $( {\mathbb R}, + )$ to the multiplicative group of positive real numbers  $( {\mathbb R^+}, \cdot )$  with the exponential map; that is,
\[
\phi( x + y) = e^{x + y} = e^x e^y = \phi( x ) \phi( y).
\]
Of course, we must still show that $\phi$ is one-to-one and onto, but this can be determined using calculus. 
\end{example}

\begin{example}{rational_isomorph}
The integers are isomorphic to the subgroup of ${\mathbb Q}^\ast$ consisting of elements of the form $2^n$.  Define a map $\phi: {\mathbb Z} \rightarrow {\mathbb Q}^\ast$ by $\phi( n ) = 2^n$. Then
\[
\phi( m + n ) = 2^{m + n} = 2^m 2^n = \phi( m ) \phi( n ).
\]
By definition the map $\phi$ is onto the subset $\{2^n :n \in {\mathbb Z} \}$ of  ${\mathbb Q}^\ast$.  To show that the map is injective, assume that $m \neq n$.  If we can show that $\phi(m) \neq \phi(n)$, then we are done.  Suppose that $m > n$ and assume that $\phi(m) = \phi(n)$.  Then $2^m = 2^n$ or $2^{m-n} = 1$, which is impossible since $m-n > 0$. 
\end{example}

\begin{example}{units}
The groups ${\mathbb Z}_8$ and ${\mathbb Z}_{12}$  cannot be isomorphic since they have different orders; however, it is true that $U(8) \cong U(12)$.  We know that
\begin{align*}
U(8) & = \{1, 3, 5, 7 \} \\
U(12) & = \{1, 5, 7, 11 \}.
\end{align*}
An isomorphism $\phi : U(8) \rightarrow U(12)$ is then given by
\begin{align*}
1 & \mapsto  1 \\
3 & \mapsto  5 \\
5 & \mapsto  7 \\
7 & \mapsto  11.
\end{align*}
The map $\phi$ is not the only possible isomorphism between these two groups.  We could define another isomorphism $\psi$ by $\psi(1) = 1$, $\psi(3) = 11$, $\psi(5) = 5$, $\psi(7) = 7$. In fact, both of these groups are isomorphic to ${\mathbb Z}_2 \times {\mathbb Z}_2$ (see Example~\ref{example:groups:Z2xZ2} in Chapter~\ref{groups}). 
\end{example}

\begin{example}{not_isomorph}
Even though $S_3$ and ${\mathbb Z}_6$ possess the same number of elements, we would suspect that they are not isomorphic, because ${\mathbb Z}_6$ is abelian and $S_3$ is nonabelian.  To demonstrate that this is indeed the case, suppose that $\phi : {\mathbb Z}_6 \rightarrow  S_3$ is an isomorphism.  Let $a , b \in S_3$ be two elements such that $ab \neq ba$.  Since $\phi$ is an isomorphism, there exist elements $m$ and $n$ in ${\mathbb Z}_6$ such~that 
\[
\phi( m )  = a \quad \text{and} \quad
\phi( n )  = b.
\]
However,
\[
ab = \phi(m ) \phi(n) = \phi(m + n) = \phi(n + m) = \phi(n )
\phi(m) = ba,
\]
which contradicts the fact that $a$ and $b$ do not commute.
\end{example}

\begin{theorem}\label{isomorph_theorem_1}
Let $\phi : G \rightarrow H$ be an isomorphism of two groups.  Then the following statements are true. 
\begin{enumerate}
 
\rm \item \it
$\phi^{-1} : H \rightarrow G$ is an isomorphism. 

\rm \item \it
$|G| = |H|$. 

\rm \item \it
If $G$ is abelian, then $H$ is abelian. 

\rm \item \it
If $G$ is cyclic, then $H$ is cyclic. 

\rm \item \it
If $G$ has a subgroup of order $n$, then $H$ has a subgroup of order $n$.
 
\end{enumerate}
\end{theorem}

\begin{proof}
Assertions (1) and (2) follow from the fact that $\phi$ is a bijection.  We will prove (3) here and leave the remainder of the theorem to be proved in the exercises.
 
(3)
Suppose that $h_1$ and $h_2$ are elements of $H$.  Since $\phi$ is onto, there exist elements $g_1, g_2 \in G$ such that $\phi(g_1) = h_1$ and $\phi(g_2) = h_2$.  Therefore, 
\[
h_1 h_2 = \phi(g_1) \phi(g_2) =  \phi(g_1 g_2) = \phi(g_2 g_1) = \phi(g_2) \phi(g_1) = h_2 h_1. 
\]
\end{proof}

\medskip

We are now in a position to characterize all cyclic groups.

\begin{theorem}\label{isomorph_theorem_2}
All cyclic groups of infinite order are isomorphic to ${\mathbb Z}$.
\end{theorem}

\begin{proof}
Let $G$ be a cyclic group with infinite order and suppose that $a$ is a generator of $G$.  Define a map $\phi : {\mathbb Z} \rightarrow  G$ by $\phi : n \mapsto a^n$. Then 
\[
\phi( m+n ) = a^{m+n} = a^m a^n = \phi( m ) \phi( n ).
\]
To show that $\phi$ is injective, suppose that $m$ and $n$ are two elements in ${\mathbb Z}$, where $m \neq n$.  We can assume that $m > n$.  We must show that $a^m \neq a^n$. Let us suppose the contrary; that is, $a^m = a^n$. In this case $a^{m - n} = e$, where $m - n > 0$, which contradicts the fact that $a$ has infinite order.  Our map is onto since any element in $G$ can be written as $a^n$ for some integer $n$ and $\phi(n) = a^n$.   
\end{proof}

\begin{theorem}\label{isomorph_theorem_3}
If $G$ is a cyclic group of order $n$, then $G$ is isomorphic to~${\mathbb Z}_n$.  
\end{theorem}
 
\begin{proof}
Let $G$ be a cyclic group of order $n$ generated by $a$ and define a map $\phi : {\mathbb Z}_n \rightarrow  G$ by $\phi : k \mapsto a^k$, where $0 \leq k < n$. The proof that $\phi$ is an isomorphism is one of the end-of-chapter exercises. 
\end{proof}

\begin{corollary}\label{isomorph_theorem_4}
If $G$ is a  group of order $p$, where $p$ is a prime number, then $G$ is isomorphic to ${\mathbb Z}_p$. 
\end{corollary}

\begin{proof}
The proof is a direct result of Corollary~\ref{cosets_theorem_7}.
\end{proof}
 
\medskip
 
The main goal in group theory is to classify all groups; however, it makes sense to consider two groups to be the same if they are isomorphic.  We state this result in the following theorem, whose proof is left as an exercise. 

\begin{theorem}\label{isomorph_theorem_5}
The isomorphism of groups determines an equivalence relation on the class of all groups. 
\end{theorem}
 
Hence, we can modify our goal of classifying all groups to classifying all groups \boldemph{up to isomorphism}; that is, we will consider two groups to be the same if they are isomorphic.

 
\subsection*{Cayley's Theorem}

Cayley proved that if $G$ is a group, it is isomorphic to a group of permutations on some set; hence, every group is a permutation group.  Cayley's Theorem is what  we call a representation theorem.  The aim of representation theory is to find an isomorphism of some group $G$ that we wish to study into a group that we know a great deal about, such as a group of permutations or matrices.

\begin{example}{cayley_isomorph}
Consider the group ${\mathbb Z}_3$.  The Cayley table for ${\mathbb Z}_3$ is as follows. 
\begin{center}
\begin{tabular}{c|ccc}
$+$   & 0 & 1 & 2 \\
\hline
0     & 0 & 1 & 2 \\
1     & 1 & 2 & 0 \\
2     & 2 & 0 & 1
\end{tabular}
\end{center}
The addition table of ${\mathbb Z}_3$ suggests that it is the same as the permutation group $G = \{ (0), (0 1 2), (0 2 1) \}$.  The isomorphism here is 
\begin{align*}
0 & \mapsto
\begin{pmatrix}
0 & 1 & 2 \\
0 & 1 & 2
\end{pmatrix}
= (0) \\
1 & \mapsto
\begin{pmatrix}
0 & 1 & 2 \\
1 & 2 & 0
\end{pmatrix}
= (0 1 2) \\
2 & \mapsto
\begin{pmatrix}
0 & 1 & 2 \\
2 & 0 & 1
\end{pmatrix}
= (0 2 1).
\end{align*}
\end{example}
 
\begin{theorem}[Cayley]\index{Cayley's Theorem}\label{isomorph_theorem_6}
Every group is isomorphic to a group of permutations.
\end{theorem}

\begin{proof}
Let $G$ be a group.  We must find a group of permutations $\overline{G}$ that is isomorphic to $G$.  For any $g \in G$, define a  function $\lambda_g : G \rightarrow G$ by $\lambda_g(a) = ga$.  We claim that $\lambda_g$ is a permutation of $G$.  To show that $\lambda_g$ is one-to-one, suppose that $\lambda_g(a) = \lambda_g(b)$.  Then  
\[
ga =\lambda_g(a) = \lambda_g(b) = gb.
\]
Hence, $a = b$.  To show that $\lambda_g$ is onto, we must prove that for each $a \in G$, there is a $b$ such that $\lambda_g (b) = a$.  Let $b = g^{-1} a$.  

Now we are ready to define our group $\overline{G}$. Let
\[
\overline{G} = \{ \lambda_g : g \in G \}.
\]
We must show that $\overline{G}$ is a group under composition of functions and find an isomorphism between $G$ and $\overline{G}$.  We have closure under composition of functions since 
\[
(\lambda_g \circ \lambda_h )(a) = \lambda_g(ha) = gha = \lambda_{gh} (a).
\]
Also,
\[
\lambda_e (a) = ea = a
\]
and
\[
(\lambda_{g^{-1}} \circ \lambda_g) (a) = \lambda_{g^{-1}} (ga) = g^{-1} g a = a = \lambda_e (a).
\]

We can define an isomorphism from $G$ to $\overline{G}$ by $\phi : g
\mapsto \lambda_g$. The group operation is preserved since
\[
\phi(gh) = \lambda_{gh} = \lambda_g \lambda_h = \phi(g) \phi(h).
\]
It is also one-to-one, because if $\phi(g)(a) = \phi(h)(a)$, then
\[
ga = \lambda_g a = \lambda_h a=  ha.
\]
Hence, $g = h$.  That $\phi$ is onto follows from the fact that $\phi( g ) = \lambda_g$ for any $\lambda_g \in \overline{G}$. 
\end{proof}

\medskip

The isomorphism $g \mapsto \lambda_g$ is known as the \boldemph{left regular representation}\index{Left regular representation} of~$G$. 


\histhead

\noindent{\small \histf
Arthur Cayley\index{Cayley, Arthur} was born in England in 1821, though he spent much of the first part of his life in Russia, where his father was a merchant.  Cayley was educated at Cambridge, where he took the first Smith's Prize in mathematics.  A lawyer for much of his adult life, he wrote several papers in his early twenties before entering the legal profession at the age of 25.  While practicing law he continued his mathematical research, writing more than 300 papers during this period of his life.  These included some of his best work.  In 1863 he left law to become a professor at Cambridge.  Cayley wrote more than 900 papers in fields such as group theory, geometry, and linear algebra. His legal knowledge was very valuable to Cambridge; he participated in the writing of many of the university's statutes.  Cayley was also one of the people responsible for the admission of women to Cambridge. 
\histbox
} 
 

\section{Direct Products}\label{isomorph_section_2}

Given two groups $G$ and $H$, it is possible to construct a new group from the Cartesian product of $G$ and $H$, $G \times H$.  Conversely, given a large group, it is sometimes possible to decompose the group; that is, a group is sometimes isomorphic to the direct product of two smaller groups.  Rather than studying a large group $G$, it is often easier to study the component groups of $G$. 
 
 
\subsection*{External Direct Products}

If $(G,\cdot)$ and $(H, \circ)$ are groups, then we can make the Cartesian product of $G$ and $H$ into a new group.  As a set, our group is just the ordered pairs $(g, h) \in G \times H$ where $g \in G$ and $h \in H$. We can define a binary operation on $G \times H$ by 
\[
(g_1, h_1)(g_2, h_2) = (g_1 \cdot g_2, h_1 \circ h_2);
\]
that is, we just multiply elements in the first coordinate as we do in $G$ and elements in the second coordinate as we do in $H$.  We have specified the particular operations $\cdot$ and $\circ$ in each group here for the sake of clarity; we usually just write $(g_1, h_1)(g_2, h_2) = (g_1  g_2, h_1 h_2)$.  

\begin{proposition}\label{isomorph_theorem_7}
Let $G$ and $H$ be groups. The set $G \times H$ is a group under the operation $(g_1, h_1)(g_2, h_2) = (g_1  g_2, h_1 h_2)$ where $g_1, g_2 \in G$ and $h_1, h_2 \in H$. 
\end{proposition}

\begin{proof}
Clearly the binary operation defined above is closed. If $e_G$ and $e_H$ are the identities of the groups $G$ and $H$ respectively, then $(e_G, e_H)$ is the identity of $G \times H$.  The inverse of $(g, h) \in G \times H$ is $(g^{-1}, h^{-1})$.  The fact that the operation is associative follows directly from the associativity of $G$ and~$H$.
\end{proof}

\begin{example}{R2_prodiuct}
Let ${\mathbb R}$ be the group of real numbers under addition.  The Cartesian product of ${\mathbb R}$ with itself, ${\mathbb R} \times {\mathbb R} = {\mathbb R}^2$, is also a group, in which the group operation is just addition in each coordinate; that is, $(a, b) + (c, d) = (a + c, b + d)$.  The identity is $(0,0)$ and the inverse of $(a, b)$ is $(-a, -b)$.
\end{example}

\begin{example}{Z2xZ2}
Consider
\[
{\mathbb Z}_2 \times {\mathbb Z}_2 = \{ (0, 0), (0, 1), (1, 0),(1, 1) \}.
\]
Although ${\mathbb Z}_2 \times {\mathbb Z}_2$ and ${\mathbb Z}_4$ both contain four elements, it is easy to see that they are not isomorphic since for every element $(a,b)$ in ${\mathbb Z}_2 \times {\mathbb Z}_2$, $(a,b) + (a,b) = (0,0)$, but ${\mathbb Z}_4$ is cyclic.
\end{example}

The group $G \times H$ is called the \boldemph{external direct product}\index{Direct product of groups!external}\index{External direct product} of  $G$ and $H$. Notice that there is nothing special about the fact that we have used only two groups to build a new group. The direct product
\[
\prod_{i = 1}^n G_i = G_1 \times G_2 \times \cdots \times G_n
\]
of the groups $G_1, G_2, \ldots, G_n$ is defined in exactly the same manner. If $G = G_1 = G_2 = \cdots = G_n$, we often write $G^n$ instead of $G_1 \times G_2 \times \cdots \times G_n$.
 
\begin{example}{Z2^n}
The group ${\mathbb Z}_2^n$, considered as a set, is just the set of all
binary $n$-tuples. The group operation is the ``exclusive or'' of two
binary $n$-tuples. For example, 
\[
(01011101) + (01001011) = (00010110).
\]
This group is important in coding theory, in cryptography, and in many
areas of computer science.  
\end{example}

 
\begin{theorem}\label{isomorph:lcm_theorem}
Let $(g, h) \in G \times H$. If $g$ and $h$ have finite orders $r$ and
$s$ respectively, then the order of $(g, h)$ in $G \times H$ is the
least common multiple of $r$ and $s$. 
\end{theorem}

 
\begin{proof}
Suppose that $m$ is the least common multiple of $r$ and $s$ and let
$n = |(g,h)|$. Then 
\begin{gather*}
(g,h)^m  = (g^m, h^m) = (e_G,e_H) \\
(g^n, h^n)  = (g, h)^n = (e_G,e_H).
\end{gather*}
Hence, $n$ must divide $m$, and $n \leq m$.  However, by the second
equation, both $r$ and $s$ must divide $n$; therefore, $n$ is a common
multiple of $r$ and $s$. Since $m$ is the \emph{least common multiple}
of $r$ and $s$, $m \leq n$.  Consequently, $m$ must be equal to~$n$.
\end{proof}
 

\begin{corollary}
Let $(g_1, \ldots, g_n) \in \prod G_i$. If $g_i$ has finite order
$r_i$ in $G_i$, then the order of $(g_1, \ldots, g_n)$ in $\prod G_i$
is the least common multiple of $r_1, \ldots, r_n$.
\end{corollary}
 
 
\begin{example}{Z12xZ60}
Let $(8, 56) \in {\mathbb Z}_{12} \times  {\mathbb Z}_{60}$. Since
$\gcd(8,12) = 4$, the order of 8 is $12/4 = 3$ in ${\mathbb Z}_{12}$.
Similarly, the order of $56$ in ${\mathbb Z}_{60}$ is $15$. The least
common multiple of 3 and 15 is 15; hence, $(8, 56)$ has order 15 in
${\mathbb Z}_{12} \times  {\mathbb Z}_{60}$.
\end{example}

 
\begin{example}{Z2xZ3}
The group ${\mathbb Z}_2 \times {\mathbb Z}_3$ consists of the pairs
\[
\begin{array}{cccccc}
(0,0),& (0, 1),& (0, 2),& (1,0),& (1, 1),& (1, 2).
\end{array}
\]
In this case, unlike that of ${\mathbb Z}_2 \times {\mathbb Z}_2$ and
${\mathbb Z}_4$, it 
is true that ${\mathbb Z}_2  \times {\mathbb Z}_3 \cong {\mathbb Z}_6$. We need
only show that ${\mathbb Z}_2  \times {\mathbb Z}_3$ is cyclic.  It is
easy to see that $(1,1)$ is a generator for ${\mathbb Z}_2  \times {\mathbb
Z}_3$. 
\end{example}

 
The next theorem tells us exactly when the direct product of two
cyclic groups is cyclic. 
 

\begin{theorem}\label{Z_pq_theorem}
The group ${\mathbb Z}_m \times {\mathbb Z}_n$ is isomorphic to ${\mathbb
Z}_{mn}$ if and only if $\gcd(m,n)=1$. 
\end{theorem}
 

\begin{proof}
Assume first that if ${\mathbb Z}_m \times {\mathbb Z}_n \cong {\mathbb
Z}_{mn}$, then $\gcd(m, n) = 1$. To show this, we will prove the
contrapositive; that is, we will show that if $\gcd(m, n) = d >
1$, then ${\mathbb Z}_m \times {\mathbb Z}_n$ cannot be cyclic. Notice that
$mn/d$ is divisible by both $m$ and $n$; hence, for any element $(a,b)
\in {\mathbb Z}_m \times {\mathbb Z}_n$,  
\[
\underbrace{(a,b) + (a,b)+ \cdots + (a,b)}_{mn/d \; {\rm
times}}
= (0, 0).
\]
Therefore, no $(a, b)$ can generate all of ${\mathbb Z}_m \times {\mathbb
Z}_n$. 

 
The converse follows directly from Theorem~\ref{isomorph:lcm_theorem} since
$\lcm(m,n) = mn$ if and only if $\gcd(m,n)=1$. 
\end{proof}
 

\begin{corollary}\label{RelativelyPrime}
Let $n_1, \ldots, n_k$ be positive integers. Then
\[
\prod_{i=1}^k {\mathbb Z}_{n_i} \cong {\mathbb Z}_{n_1 \cdots n_k}
\]
if and only if $\gcd( n_i, n_j) =1$ for $i \neq j$.
\end{corollary}

 
\begin{corollary}
If
\[
m = p_1^{e_1} \cdots  p_k^{e_k},
\]
where the $p_i$s are distinct primes, then
\[
{\mathbb Z}_m \cong {\mathbb Z}_{p_1^{e_1}} \times \cdots \times {\mathbb
Z}_{p_k^{e_k}}.
\]
\end{corollary}
 
 
\begin{proof}
Since the greatest common divisor of $p_i^{e_i}$ and $p_j^{e_j}$ is
1 for $i \neq j$, the proof follows from Corollary~\ref{RelativelyPrime}.
\end{proof}


\medskip


In Chapter~\ref{struct}, we will prove that all finite abelian groups are %This reference needs to be fixed - TWJ 6/2/2010
isomorphic to direct products of
the form
\[
{\mathbb Z}_{p_1^{e_1}} \times \cdots \times {\mathbb
Z}_{p_k^{e_k}}
\]
where $p_1, \ldots, p_k$ are (not necessarily distinct) primes.

 
 
\subsection*{Internal Direct Products}
 

The external direct product of two groups builds a large group out of
two smaller groups.   We would like to be able to reverse this process
and conveniently break down a group into its direct product
components; that is, we would like to be able to say when a group is
isomorphic to the direct product of two of its subgroups.
 

Let $G$ be a group with subgroups $H$ and $K$ satisfying the following
conditions.
\begin{itemize}
 
\item
$G = HK = \{ hk : h \in H, k \in K  \}$;
 
\item
$H \cap K = \{ e \}$;
 
\item
$hk = kh$ for all $k \in K$ and $h \in H$.
 
\end{itemize}
Then $G$ is the \boldemph{internal direct product}\index{Direct product of
groups!internal}\index{Internal direct product} of $H$ and $K$.

 
\begin{example}{U8}
The group $U(8)$ is the internal direct product of
\[
H  = \{1, 3 \} \quad \text{and} \quad K  = \{1, 5 \}.
\]
\end{example}

 
\begin{example}{D6_product}
The dihedral group $D_6$ is an internal direct product of its two
subgroups 
\[
H  = \{id, r^3  \} \quad \text{and} \quad
K  = \{id, r^2, r^4, s, r^2s, r^4 s   \}.
\]
It can easily be shown that $K \cong S_3$; consequently, $D_6 \cong
{\mathbb Z}_2 \times S_3$. 
\end{example}

 
\begin{example}{S3_not_a_product}
Not every group can be written as the internal direct product of two
of its proper subgroups.  If the group $S_3$ were an internal direct
product of its proper subgroups $H$ and $K$, then one of the  subgroups,
say $H$, would have to have order 3. In this case $H$ is the subgroup $\{
(1), (123), (132) \}$. The subgroup $K$ must have order 2, but no
matter which subgroup we choose for $K$, the condition that $hk = kh$
will never be satisfied for $h \in H$ and $k \in K$.
\mbox{\hspace{1in}}
\end{example}

 
\begin{theorem}\label{isomorph:directproducts}
Let $G$ be the internal direct product of  subgroups $H$ and $K$. Then
$G$ is isomorphic to $H \times K$. 
\end{theorem}
 

\begin{proof}
Since $G$ is an internal direct product, we can write any element $g
\in G$ as $g =hk$ for some $h \in H$ and some $k \in K$. Define a map
$\phi : G \rightarrow H \times K$ by $\phi(g) = (h,k)$.

 
The first problem that we must face is to show that $\phi$ is a
well-defined map; that is, we must show that $h$ and $k$ are uniquely
determined by $g$. Suppose that $g = hk=h'k'$. Then $h^{-1} h'= k
(k')^{-1}$ is in both $H$ and $K$, so it must be the identity.
Therefore, $h = h'$ and $k = k'$, which proves that $\phi$ is, indeed,
well-defined. 

 
To show that $\phi$ preserves the group operation, let $g_1 = h_1 k_1$
and $g_2 = h_2 k_2$ and observe that 
\begin{align*}
\phi( g_1 g_2 ) & = \phi( h_1 k_1 h_2 k_2 )\\
& = \phi(h_1  h_2 k_1 k_2) \\
& = (h_1  h_2, k_1 k_2) \\
& = (h_1, k_1)( h_2, k_2) \\
& = \phi( g_1 ) \phi(  g_2 ).
\end{align*}
We will leave the proof that $\phi$ is one-to-one and onto
as an exercise.
\end{proof}

 
\begin{example}{Z6_product}
The group ${\mathbb Z}_6$ is an internal direct product isomorphic to $\{
0, 2, 4\} \times \{ 0, 3 \}$. 
\end{example}

 
We can extend the definition of an internal direct product of $G$ to a
collection of subgroups $H_1, H_2, \ldots, H_n$ of $G$, by requiring
that 
\begin{itemize}
 
\item
$G = H_1 H_2 \cdots H_n = \{ h_1 h_2 \cdots h_n : h_i \in H_i \}$;
 
\item
$H_i \cap \langle \cup_{j \neq i} H_j \rangle = \{ e \}$;
 
\item
$h_i h_j = h_j h_i$ for all $h_i \in H_i$ and $h_j \in H_j$.
 
\end{itemize}
We will leave the proof of the following theorem as an exercise. 
 
\begin{theorem}
Let $G$ be the internal direct product of subgroups $H_i$, where $i =
1, 2, \ldots, n$. Then $G$ is isomorphic to $\prod_i H_i$. 
\end{theorem}

 


 
\markright{EXERCISES}
\section*{Exercises}
\exrule

 
 
{\small
\begin{enumerate}
 
%**********************Computations
 
\item
Prove that ${\mathbb Z} \cong n{\mathbb Z}$ for $n \neq 0$.
 

\item
Prove that ${\mathbb C}^\ast$ is isomorphic to the subgroup of $GL_2(
{\mathbb R} )$ consisting of matrices of the form 
\[
\begin{pmatrix}
a & b \\
-b & a
\end{pmatrix}
\]
 

\item
Prove or disprove: $U(8) \cong {\mathbb Z}_4$.
 

\item
Prove that $U(8)$ is isomorphic to the group of matrices
\[
\begin{pmatrix}
1 & 0 \\
0 & 1
\end{pmatrix},
\begin{pmatrix}
1 & 0 \\
0 & -1
\end{pmatrix},
\begin{pmatrix}
-1 & 0 \\
0 & 1
\end{pmatrix},
\begin{pmatrix}
-1 & 0 \\
0 & -1
\end{pmatrix}.
\]
 

\item
Show that $U(5)$ is isomorphic to $U(10)$, but $U(12)$ is not.
 

\item
Show that the $n$th roots of unity are isomorphic to ${\mathbb Z}_n$. 
 

\item 
Show that any cyclic group of order $n$ is isomorphic to ${\mathbb Z}_n$. 
 

\item
Prove that ${\mathbb Q}$ is not isomorphic to ${\mathbb Z}$.
 

\item
Let $G = {\mathbb R} \setminus \{ -1 \}$ and define a binary operation on
$G$ by 
\[
a \ast b = a + b + ab.
\]
Prove that $G$ is a group under this operation. Show that $(G, *)$ is
isomorphic to the multiplicative group of nonzero real numbers.
 

\item
Show that the matrices
\begin{gather*}
\begin{pmatrix}
1 & 0 & 0 \\
0 & 1 & 0 \\
0 & 0 & 1
\end{pmatrix}
\quad
\begin{pmatrix}
1 & 0 & 0 \\
0 & 0 & 1 \\
0 & 1 & 0
\end{pmatrix}
\quad
\begin{pmatrix}
0 & 1 & 0 \\
1 & 0 & 0 \\
0 & 0 & 1
\end{pmatrix} \\
\begin{pmatrix}
0 & 0 & 1 \\
1 & 0 & 0 \\
0 & 1 & 0
\end{pmatrix}
\quad
\begin{pmatrix}
0 & 0 & 1 \\
0 & 1 & 0 \\
1 & 0 & 0
\end{pmatrix}
\quad
\begin{pmatrix}
0 & 1 & 0 \\
0 & 0 & 1 \\
1 & 0 & 0
\end{pmatrix}
\end{gather*}
form a group. Find an isomorphism of $G$ with a more familiar group of
order~6. 

 
\item
Find five non-isomorphic groups of order 8.
 

\item
Prove $S_4$ is not isomorphic to $D_{12}$.
 
% TWJ, 2010/04/21
% Made correction to exercise at the suggestion of C. Thon

\item
Let $\omega = \cis(2 \pi /n)$ be a primitive $n$th root of
unity.  Prove that the matrices 
\[
A=
\begin{pmatrix}
\omega & 0 \\
0 & \omega^{-1}
\end{pmatrix}
\quad \text{and} \quad
B =
\begin{pmatrix}
0 & 1 \\
1 & 0
\end{pmatrix}
\]
generate a multiplicative group isomorphic to $D_n$.
 

\item
Show that the set of all matrices of the form
\[
\begin{pmatrix}
\pm 1 & k\\
0 & 1
\end{pmatrix},
\]
is a group isomorphic to $D_n$, where all entries in the matrix are in ${\mathbb Z}_n$.

%TWJ 10/21/2012
%Statement of exercise corrected.  Suggested by R. Beezer and B. Whetter.
 

\item
List all of the elements of ${\mathbb Z}_4 \times {\mathbb Z}_2$.
 

\item
Find the order of each of the following elements.

\begin{enumerate}
 
 \item
$(3, 4)$ in ${\mathbb Z}_4 \times {\mathbb Z}_6$

 \item
$(6, 15, 4)$ in ${\mathbb Z}_{30} \times {\mathbb Z}_{45} \times {\mathbb
Z}_{24}$

 \item
$(5, 10, 15)$ in ${\mathbb Z}_{25} \times {\mathbb Z}_{25} \times {\mathbb
Z}_{25}$

 \item
$(8, 8, 8)$ in ${\mathbb Z}_{10} \times {\mathbb Z}_{24} \times {\mathbb
Z}_{80}$
 
\end{enumerate}
 

\item
Prove that $D_4$ cannot be the internal direct product of two of its
proper subgroups. 
 

\item
Prove that the subgroup of ${\mathbb Q}^\ast$ consisting of elements of
the form $2^m 3^n$ for $m,n \in {\mathbb Z}$ is an internal direct
product isomorphic to ${\mathbb Z} \times {\mathbb Z}$.
 

\item
Prove that $S_3 \times {\mathbb Z}_2$ is isomorphic to $D_6$. Can you
make a conjecture about $D_{2n}$? Prove your conjecture. [\emph{Hint:}
Draw the picture.] 
 

\item
Prove or disprove: Every abelian group of order divisible by 3
contains a subgroup of order 3.  


\item
Prove or disprove: Every nonabelian group of order divisible by 6
contains a subgroup of order 6. 
 

\item
Let $G$ be a group of order 20. If $G$ has subgroups $H$ and $K$ of
orders 4 and 5 respectively such that $hk = kh$ for all $h \in H$ and
$k \in K$, prove that $G$ is the internal direct product of $H$ and $K$. 
 

\item
Prove or disprove the following assertion. Let $G$, $H$, and $K$ be
groups. If $G \times K \cong H \times K$, then $G \cong H$. 
 

\item
Prove or disprove: There is a noncyclic abelian group of order 51. 
 

\item
Prove or disprove: There is a noncyclic abelian group of order 52. 
 
%*****************************Theory
 

\item
Let $\phi : G_1 \rightarrow G_2$ be a group isomorphism. Show that
$\phi( x) = e$ if and only if $x=e$. 
 

\item
Let $G \cong H$. Show that if $G$ is cyclic, then so is $H$.
 

\item
Prove that any group $G$ of order $p$, $p$  prime, must be isomorphic
to ${\mathbb Z}_p$. 
 

\item
Show that $S_n$ is isomorphic to a subgroup of $A_{n+2}$.  

\item
Prove that $D_n$ is isomorphic to a subgroup of $S_n$.
 

\item
Let $\phi : G_1 \rightarrow G_2$ and  $\psi : G_2 \rightarrow G_3$  be
isomorphisms. Show that  $\phi^{-1}$ and $\psi \circ \phi$ are both
isomorphisms. Using these results, show that the isomorphism of groups
determines an equivalence relation on the class of all groups.
 

\item
Prove $U(5) \cong {\mathbb Z}_4$. Can you generalize this result to show
that $U(p) \cong {\mathbb Z}_{p-1}$? 
 

\item
Write out the permutations associated with each element of $S_3$ in
the proof of Cayley's Theorem. 
 
%*****************Automorphisms
 

\item
An \boldemph{automorphism}\index{Automorphism!of a
group}\index{Group!automorphism of} of a group $G$ is an isomorphism
with itself. Prove that complex conjugation is an automorphism of the
additive group of complex numbers; that is, show that the map $\phi(
a + bi ) = a - bi$ is an isomorphism from ${\mathbb C}$ to ${\mathbb C}$. 
 

\item
Prove that $a + ib \mapsto a - ib$ is an automorphism of ${\mathbb C}^*$. 
 

\item
Prove that $A \mapsto B^{-1}AB$ is an automorphism of $SL_2({\mathbb R})$
for all $B$ in $GL_2({\mathbb R})$. 
 

\item
We will denote the set of all automorphisms of $G$ by
$\aut(G)$\label{noteauto}.  Prove that  $\aut(G)$ is a subgroup of
$S_G$, the group of permutations of $G$. 
 

\item
Find $\aut( {\mathbb Z}_6)$.
 

\item
Find $\aut( {\mathbb Z})$.
 

\item
Find two nonisomorphic groups $G$ and $H$ such that $\aut(G) \cong \aut(
H)$. 
 

\item
Let $G$ be a group and $g \in G$. Define a map $i_g : G \rightarrow
G$\label{noteinner} 
by $i_g(x) = g x g^{-1}$.  Prove that $i_g$ defines an automorphism of
$G$.  Such an automorphism is called an \boldemph{inner
automorphism}\index{Automorphism!inner}. The set of all inner
automorphisms is denoted by $\inn(G)$\label{noteinneraut}. 
 

\item
Prove that $\inn(G)$ is a subgroup of $\aut(G)$.
 

\item
What are the inner automorphisms of the quaternion group $Q_8$? Is
$\inn(G) = \aut(G)$ in this case? 
 

\item
Let $G$ be a group and $g \in G$.  Define maps $\lambda_g :G
\rightarrow G$ and $\rho_g :G \rightarrow G$\label{noterightreg}
 by $\lambda_g(x) = gx$
and $\rho_g(x) = xg^{-1}$. Show that $i_g = \rho_g \circ \lambda_g$ is
an automorphism of $G$. The isomorphism $g \mapsto \rho_g$ is called
the \boldemph{right regular representation}\index{Right regular
representation} of $G$. 
% Fixed the definition of right regular representation.  Suggested by Z. Teitler.
% TWJ - 12/19/2011
 

\item
Let $G$ be the internal direct product of subgroups $H$ and $K$.  Show
that the map $\phi : G \rightarrow H \times K$ defined by  $\phi(g) =
(h,k)$ for $g =hk$,  where $h \in H$ and  $k \in K$, is one-to-one and
onto. 
 

\item
Let $G$ and $H$ be isomorphic groups. If $G$ has a subgroup of order
$n$, prove that $H$ must also have a subgroup of  order $n$.
 

\item
If $G \cong \overline{G}$ and $H \cong \overline{H}$, show that $G
\times H \cong \overline{G} \times \overline{H}$.
 

\item
Prove that $G \times H$ is isomorphic to $H \times G$.
 

\item
Let $n_1, \ldots, n_k$ be positive integers. Show that
\[
\prod_{i=1}^k {\mathbb Z}_{n_i} \cong {\mathbb Z}_{n_1 \cdots n_k}
\]
if and only if $\gcd( n_i, n_j) =1$ for $i \neq j$.
 

\item
Prove that $A \times B$ is abelian if and only if $A$ and $B$ are
abelian. 
 

\item
If $G$ is the internal direct product of $H_1, H_2, \ldots, H_n$,
prove that $G$ is isomorphic to $\prod_i H_i$. 
 

\item
Let $H_1$ and $H_2$ be subgroups of $G_1$ and $G_2$, respectively. Prove that $H_1 \times H_2$ is a subgroup of $G_1 \times G_2$. 
 

\item % Change from \cong to equality suggested by Z. Teitler - TWJ 12/19/2011
Let $m, n \in {\mathbb Z}$. Prove that $\langle m,n \rangle = \langle d \rangle$ if and only if $d = \gcd(m,n)$.
 

\item  %Correction suggested by K. Brooks. - TWJ 11/21/2011
% Change from \cong to equality suggested by Z. Teitler - TWJ 12/19/2011
Let $m, n \in {\mathbb Z}$. Prove that $\langle m \rangle \cap \langle n \rangle = \langle l \rangle$ if and only if $l = \lcm(m,n)$. 

\item %Exercise suggested by R. Beezer. - TWJ 8/30/2011
\textbf{Groups of order $2p$.}
In this series of exercises we will classify all groups of order $2p$, where $p$ is an odd prime.
\begin{enumerate}

\item
Assume $G$ is a group of order $2p$, where $p$ is an odd prime.  If $a \in G$, show that $A$ must have order 1, 2, $p$, or $2p$.


\item
Suppose that $G$ an element of order $2p$.  Prove that $G$ isomorphic to ${\mathbb Z}_{2p}$.  Hence, $G$ is cyclic.



\item
Suppose that $G$ does not contain an element of order $2p$.  Show that $G$ must contain an element of order $p$.  \emph{Hint}:  Assume that $G$ does not contain an element of order $p$.



\item
Suppose that $G$ does not contain an element of order $2p$.  Show that $G$ must contain an element of order 2. 



\item
Let $P$ be a subgroup of $G$ with order $p$ and $y \in G$ have order 2.  Show that $yP = Py$.



\item
Suppose that $G$ does not contain an element of order $2p$ and $P = \langle z \rangle$ is a subgroup of order $p$ generated by $z$.  If $y$ is an element of order 2, then $yz = z^ky$ for some $2 \leq k < p$.



\item
Suppose that $G$ does not contain an element of order $2p$.  Prove that $G$ is not abelian.



\item
Suppose that $G$ does not contain an element of order $2p$ and $P = \langle z \rangle$ is a subgroup of order $p$ generated by $z$ and $y$ is an element of order 2.
Show that we can list the elements of $G$ as $\{z^iy^j\mid 0\leq i < p, 0\leq j < 2\}$.



\item
Suppose that $G$ does not contain an element of order $2p$ and $P = \langle z \rangle$ is a subgroup of order $p$ generated by $z$ and $y$ is an element of order 2.  Prove that the product
$(z^iy^j)(z^ry^s)$ can be expressed as a uniquely as $z^m y^n$ for some non negative integers $m, n$.  Thus, conclude that there is only one possibility for a non-abelian group of order $2p$, it must therefore be the one we have seen already, the dihedral group.



 
\end{enumerate}

 
\end{enumerate}
}

\sagesection




