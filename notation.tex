%%%%(c)
%%%%(c)  This file is a portion of the source for the textbook
%%%%(c)
%%%%(c)    Abstract Algebra: Theory and Applications
%%%%(c)    Copyright 1997 by Thomas W. Judson
%%%%(c)
%%%%(c)  See the file COPYING.txt for copying conditions
%%%%(c)
%%%%(c)
\chapter*{Notation}

% Print versions need page headers, ToC entry
% tex4ht versions get their own ToC automatically
\ifthenelse{\boolean{basic}\or\boolean{vcu}}{%
\phantomsection
\addcontentsline{toc}{chapter}{Notation}
\pagestyle{myheadings}
\markboth{NOTATION}{NOTATION}
}{}
 
The following table defines the  notation used in this book. Page numbers
refer to the first appearance of each symbol.
 
\begin{tabbing}
\hspace{1.5in} \= \hspace{2.5in} \= \kill
{\bf Symbol}  \> {\bf Description} \>  \` {\bf Page} \\ 
     \mbox{\hspace*{1in}} \\
$a \in A$ \> $a$ is in the set $A$ \> \` \pageref{setmembership} \\
${\Bbb N}$ \> the natural numbers \> \` \pageref{naturalnum} \\
${\Bbb Z}$ \> the integers \> \` \pageref{integers} \\
${\Bbb Q}$ \> the rational numbers \> \` \pageref{rationals} \\
${\Bbb R}$ \> the real numbers \> \` \pageref{reals} \\
${\Bbb C}$ \> the complex numbers \> \` \pageref{complexnum} \\
$A \subset B$ \> $A$ is a subset of $B$ \> \` \pageref{setcontain} \\
$\emptyset$ \> the empty set \> \` \pageref{theemptyset} \\
$A \cup B$ \> union of sets $A$ and $B$ \> \` \pageref{union} \\
$A \cap B$ \> intersection of sets $A$ and $B$ \> 
     \` \pageref{intersection} \\
$A'$ \> complement of the  set $A$	 \> \` \pageref{setcomplement} \\
$A \setminus B$ \> difference between sets $A$ and $B$ \>
     \` \pageref{setdifference} \\
$A \times B$ \> Cartesian product of sets $A$ and $B$ \>
     \` \pageref{cartesian} \\
$A^n$ \> $A \times \cdots \times A$ ($n$ times) \> 
     \` \pageref{ncartesian} \\
$id$ \> identity mapping \> \` \pageref{noteidentity} \\
$f^{-1}$ \> inverse of the function $f$	\> \` \pageref{inversefunc} \\
$a \equiv b \pmod{n}$ \> $a$ is congruent to $b$ modulo $n$ \> 
     \` \pageref{amodb} \\
$n!$ \> $n$ factorial \> \` \pageref{factorial} \\
$\left(\begin{array}{c}n \\ k \end{array} \right)$ \> binomial
     coefficient $n!/(k! (n-k)!)$ \> \` \pageref{binomial} \\
$m \mid n$ \> $m$ divides $n$ \> \` \pageref{divides} \\
$\gcd(m, n)$ \> greatest common divisor of $m$ and $n$ \>
     \` \pageref{greatestcd}
\end{tabbing} \clearpage
\begin{tabbing}
\hspace{1.5in} \= \hspace{2.5in} \= \kill     
{\bf Symbol}  \> {\bf Description} \>  \` {\bf Page} \\ 
     \mbox{\hspace*{1in}} \\
${\cal P}(X)$ \> power set of $X$ \> \` \pageref{powerset} \\
${\Bbb Z}_n$ \> the integers modulo $n$ \> \` \pageref{integersmodn} \\
$\lcm(m,n)$ \> least common multiple of $m$ and $n$ \>
     \` \pageref{leastcm} \\
$U(n)$ \> group of units in ${\Bbb Z}_n$ \> \` \pageref{groupofunits} \\
${\Bbb M}_n ( {\Bbb R})$ \> the $n \times n$ matrices with entries in
     ${\Bbb R}$ \> \`  \pageref{notematrices} \\
$\det A$ \> determinant of $A$ \> \` \pageref{determinant} \\
$GL_n({\Bbb R})$ \> general linear group \> \` \pageref{generallinear} \\
$Q_8$ \> the group of quaternions \> \` \pageref{notequateriongroup} \\
${\Bbb C}^\ast$ \> the multiplicative group of complex numbers \>
     \` \pageref{noteCstar} \\
$|G|$ \> order of a group $G$ \> \` \pageref{noteorder} \\
${\Bbb R}^*$ \> the multiplicative group of real numbers \>
     \` \pageref{noteRstar} \\
${\Bbb Q}^*$ \> the multiplicative group of rational numbers \>
     \` \pageref{noteQstar} \\
$SL_n({\Bbb R})$ \> special linear group \> \` \pageref{speciallinear} \\
$Z(G)$ \> center of a group $G$ \> \` \pageref{centerofagroup} \\
$\langle a \rangle$ \> cyclic subgroup generated by $a$ \>
     \` \pageref{generatedby} \\  
$|a|$ \> order of an element $a$ \> \`
     \pageref{noteelementorder} \\
$\cis \theta$ \> $\cos \theta + i \sin \theta$ \> 
     \` \pageref{cosisin} \\
${\Bbb T}$ \> the circle group \> \` \pageref{notecirclegroup} \\
$S_n$ \> symmetric group on $n$ letters \> 
     \` \pageref{symmetricgroup} \\
$(a_1, a_2, \ldots, a_k )$ \> cycle of length $k$ \> 
     \` \pageref{notecycle} \\
$A_n$ \> alternating group on $n$ letters \> 
     \` \pageref{alternatinggroup} \\
$D_n$ \> dihedral group \> \` \pageref{dihedralgroup} \\
$[G:H]$ \> index of a subgroup $H$ in a group $G$ \> 
     \` \pageref{indexofasubgroup}  \\
${\cal L}_H$ \> set of left cosets of $H$ in a group $G$ \> 
     \` \pageref{notesetleft} \\
${\cal R}_H$ \> set of right cosets of $H$ in a group $G$ \>
     \` \pageref{notesetright}  \\
$d({\bold x}, {\bold y})$ \> Hamming distance between ${\bold x}$ and
     ${\bold y}$ \> \` \pageref{noteHammingdist} \\
$d_{\min}$ \> minimum distance of a code \> \` \pageref{notemindist}\\ 
$w({\bold x})$ \> weight of ${\bold x}$ \> \` \pageref{noteweight} \\
${\Bbb M}_{m \times n}({\Bbb Z}_2)$ \> set of $m$ by $n$ matrices with 
     entries in ${\Bbb Z}_2$ \> \` \pageref{notembyn} \\
${\rm Null}(H)$ \> null space of a matrix $H$ \> 
     \` \pageref{notenull} \\
$\delta_{ij}$ \> Kronecker delta \> \` \pageref{notekron} \\
$G \cong H$ \> $G$ is isomorphic to $H$ \> \` \pageref{noteisomorph} \\
$Aut(G)$ \> automorphism group of $G$ \> \` \pageref{noteauto} \\
$i_g$ \> $i_g(x) = gxg^{-1}$ \> \` \pageref{noteinner} \\
$Inn(G)$ \> inner automorphism group of $G$ \> \` \pageref{noteinneraut} \\
$\rho_g$ \> right regular representation \> \` \pageref{noterightreg}
\end{tabbing} \clearpage
\begin{tabbing}
\hspace{1.5in} \= \hspace{2.5in} \= \kill
{\bf Symbol}  \> {\bf Description} \>  \` {\bf Page} \\ 
     \mbox{\hspace*{1in}} \\
$G/N$ \> factor group of $G$ mod $N$ \> \` \pageref{notefactor} \\
$\ker \phi$ \> kernel of $\phi$ \> \` \pageref{kernelofphi} \\
$G'$ \> commutator subgroup of $G$ \> \` \pageref{commutatorsubgroup} \\
$(a_{ij})$ \> matrix \> \` \pageref{matrixnote} \\
$O(n)$ \> orthogonal group \> \` \pageref{noteorthogonal} \\
$\| {\bold x} \|$ \> length of a vector ${\bold x}$ \> 
     \` \pageref{notelengthvect} \\
$SO(n)$ \> special orthogonal group \> \` \pageref{notespecialorthog} \\
$E(n)$ \> Euclidean group \> \` \pageref{noteeuclidgroup} \\
${\cal O}_x$ \> orbit of $x$ \> \` \pageref{noteorbit} \\
$X_g$ \> fixed point set of $g$ \> \` \pageref{notefixed} \\
$G_x$ \> isotropy subgroup of $x$ \> \` \pageref{noteisotropy} \\
$X_G$ \> set of fixed points in a $G$-set $X$ \> 
     \` \pageref{noteXG} \\
$N(H)$ \> normalizer of a subgroup $H$ \> 
     \` \pageref{notenormalizer} \\ 
${\Bbb H}$ \> the ring of quaternions \> \` \pageref{noteringH} \\
\mbox{char$\, R$} \> characteristic of a ring $R$ \> \` \pageref{ringchar} \\
${\Bbb Z}[i ]$ \> the Gaussian integers \> \` \pageref{gaussianintegers} \\
${\Bbb Z}_{(p)}$ \> ring of integers localized at $p$ \> 
     \` \pageref{notelocalint} \\
$R[x]$ \> ring of polynomials over $R$ \> 
     \` \pageref{polynomialring} \\
$\deg p(x)$ \> degree of $p(x)$ \> \` \pageref{polydegree} \\
$R[x_1, x_2, \ldots, x_n]$ \> ring of polynomials in $n$ variables \>
     \` \pageref{notepolynvar} \\
$\phi_{\alpha}$ \> evaluation homomorphism at $\alpha$ \> \`
     \pageref{noteevalhomo} \\
${\Bbb Q}(x)$ \> field of rational functions over ${\Bbb Q}$ \> 
     \` \pageref{noteratpoly} \\
$\nu(a)$ \> Euclidean valuation of $a$ \> 
     \` \pageref{notevaluation} \\ 
$F(x)$ \> field of rational functions in $x$ \> 
     \` \pageref{noteratfun} \\
$F(x_1, \ldots, x_n)$ \> field of rational functions in 
     $x_1, \ldots, x_n$ \> \` \pageref{noteratnvar} \\ 
$a \preceq b$ \>  $a$ is less than $b$ \> \` \pageref{lessthan} \\
$a \wedge b$ \> meet of $a$ and $b$ \> \` \pageref{meet} \\
$a \vee b$ \> join of $a$ and $b$ \> \` \pageref{join} \\
$I$ \> largest element in a lattice \>
     \` \pageref{notelargeposet} \\
$O$ \> smallest element in a lattice \> 
     \` \pageref{notesmallposet} \\
$a'$ \> complement of $a$ in a lattice \> 
     \`\pageref{notedlatticecomp}  \\
$\dim V$ \> dimension of a vector space $V$ \> \` \pageref{vectdim} \\
$U \oplus V$ \> direct sum of vector spaces $U$ and $V$ \>
     \` \pageref{notedirectsum} \\
$\mbox{Hom}(V, W)$ \> set of all linear transformations from $U$ to
     $V$ \> \` \pageref{noteHom} \\
$V^\ast$ \> dual of a vector space $V$ \> \` \pageref{notedual} \\
$F( \alpha_1, \ldots, \alpha_n)$ \> smallest field containing $F$ and
     $\alpha_1, \ldots, \alpha_n$ \> \` \pageref{notefieldcont}	
\end{tabbing} \clearpage
\begin{tabbing}
\hspace{1.5in} \= \hspace{2.5in} \= \kill
{\bf Symbol}  \> {\bf Description} \>  \` {\bf Page} \\ 
     \mbox{\hspace*{1in}} \\
$[E:F]$ \> dimension of a field extension of $E$ over $F$ \> 
     \` \pageref{notedegext} \\
GF$(p^n)$ \> Galois field of order $p^n$ \> 
     \` \pageref{galoisfield} \\
$F^*$ \> multiplicative group of a field $F$ \> 
     \` \pageref{ntmultgrp} \\
$G(E/F)$ \> Galois group of $E$ over $F$ \> \`\pageref{notegalois} \\
$F_{\{\sigma_i \}}$ \> field fixed by automorphisms $\sigma_i$ \>
     \` \pageref{noteFixedfield} \\
$F_G$ \> field fixed by automorphism group $G$ \> 
     \` \pageref{noteFixedG} \\
$\Delta^2$ \> discriminant of a polynomial \>  
     \` \pageref{discriminant}
\end{tabbing}
 
 
 
 
