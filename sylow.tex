%%%%(c)
%%%%(c)  This file is a portion of the source for the textbook
%%%%(c)
%%%%(c)    Abstract Algebra: Theory and Applications
%%%%(c)    Copyright 1997 by Thomas W. Judson
%%%%(c)
%%%%(c)  See the file COPYING.txt for copying conditions
%%%%(c)
%%%%(c)
\chap{The Sylow Theorems}{sylow}
 
We already know that the converse of Lagrange's Theorem is false.  If $G$ is a group of order $m$ and $n$ divides $m$, then $G$ does not necessarily possess a subgroup of order~$n$.  For example, $A_4$ has order 12 but does not possess a subgroup of order 6.  However, the Sylow Theorems do provide a partial converse for Lagrange's Theorem: in certain cases they guarantee us subgroups of specific orders.  These theorems yield a powerful set of tools for the classification of all finite nonabelian groups.  

 
\section{The Sylow Theorems}

We will use the idea of  group actions to prove the Sylow Theorems.  Recall for a moment what it means for $G$ to act on itself by conjugation and how conjugacy classes are distributed in the group according to the class equation, discussed in Chapter~\ref{actions}.   A group  $G$ acts on itself by conjugation via the map $(g,x) \mapsto gxg^{-1}$. Let $x_1, \ldots, x_k$ be representatives from each  of the distinct conjugacy classes of $G$ that consist of more than one element. Then the class equation can be written as 
\[
|G| = |Z(G)| + [G: C(x_1) ] + \cdots + [ G: C(x_k)],
\]
where $Z(G) = \{g \in G : gx = xg \text{ for all } x \in G\}$ is the center of $G$ and  $C(x_i) = \{ g \in G : g x_i = x_i g \}$ is the centralizer subgroup of $x_i$.
 
We now begin our investigation of the Sylow Theorems by examining subgroups of order $p$, where $p$ is prime.  A group $G$ is a {\bfi $p$-group\/}\index{Group!$p$-group} if every element in $G$ has as
its order a power of $p$, where $p$ is a prime number. A subgroup of a group $G$ is a {\bfi $p$-subgroup\/}\index{Subgroup!$p$-subgroup} if it is a  $p$-group.  
 
\begin{theorem} {\bf (Cauchy)}\index{Cauchy's Theorem}
Let $G$ be a finite group and $p$ a prime such that $p$ divides the
order of $G$. Then $G$ contains a subgroup of order $p$.
\end{theorem}
 
\begin{proof}
We will use induction on the order of $G$.  If $|G|=p$, then clearly
$G$ must have an element of order $p$. Now assume that every group of
order $k$, where $p \leq k < n$ and $p$ divides $k$, has an element of
order $p$. Assume that $|G|= n$ and $p \mid n$ and consider the class
equation of $G$: 
\[
|G| = |Z(G)| + [G: C(x_1) ] + \cdots + [ G: C(x_k)].
\]
We have two cases.
 
 
{\em Case 1.}
{\em The order of one of the centralizer subgroups, $C(x_i)$, is
divisible by $p$ for some $i$, $i = 1, \ldots, k$}. 
In this case,  by our induction hypothesis, we are done. Since $C(x_i)$
is a proper subgroup of $G$ and $p$ divides $|C(x_i)|$, $C(x_i)$ must
contain an element of order $p$. Hence, $G$ must contain an element of
order $p$.  
 
 
{\em Case 2.}
{\em The order of no centralizer subgroup is divisible by
$p$}. 
Then $p$ divides $[G:C(x_i)]$, the order of each conjugacy class in
the class equation; hence, $p$ must divide the center of $G$, $Z(G)$.
Since $Z(G)$ is abelian, it must have a subgroup of order $p$ by the
Fundamental Theorem of Finite Abelian Groups. Therefore, the center of
$G$ contains an element of order $p$. 
\end{proof}
 
 
\begin{corollary}
Let $G$ be a finite group. Then $G$ is a $p$-group if and only if $|G|
= p^n$. 
\end{corollary}
 
 
\begin{example}{A5_subgroups}
Let us consider the group $A_5$.  We know that $|A_5| = 60 = 2^2 \cdot
3 \cdot 5$.  By Cauchy's Theorem, we are guaranteed that $A_5$ has
subgroups of orders $2$, $3$ and $5$. The Sylow Theorems give us even
more information about the possible subgroups of $A_5$.
\end{example}
 
 

 
 
We are now ready to state and prove the first of the Sylow Theorems.
The proof is very similar to the proof of Cauchy's Theorem.
 
 
\begin{theorem}[First Sylow Theorem]
Let $G$ be a finite group and $p$ a prime such that $p^r$ divides
$|G|$. Then $G$ contains a subgroup of order $p^r$. 
\end{theorem}
 
 
\begin{proof}
We induct on the order of $G$ once again.  If $|G|=p$, then we are
done. Now suppose that the order of $G$ is $n$ with $n>p$ and that the
theorem is true for all groups of order less than $n$. We shall apply
the class equation once again: 
\[
|G| = |Z(G)| + [G: C(x_1) ] + \cdots + [ G: C(x_k)].
\]
 
 
First suppose that $p$ does not divide $[G:C(x_i)]$ for some $i$. Then
\mbox{$p^r \mid |C(x_i)|$}, since $p^r$ divides $|G| = |C(x_i)| \cdot
[G:C(x_i)]$. Now we can apply the induction hypothesis to $C(x_i)$.  
 
 
Hence, we may assume that $p$ divides $[G:C(x_i)]$ for all $i$.
Since $p$ divides $|G|$, the class equation says that $p$ must
divide $|Z(G)|$; hence, by Cauchy's Theorem, $Z(G)$ has an element of
order $p$, say $g$. Let $N$ be the group generated by $g$. Clearly,
$N$ is a normal subgroup of $Z(G)$ since $Z(G)$ is abelian; therefore,
$N$ is normal in $G$ since every element in $Z(G)$ commutes with every
element in $G$. Now consider the factor group $G/N$ of order $|G|/p$.
By the induction hypothesis, $G/N$ contains a subgroup $H$ of order
$p^{r- 1}$. The inverse image of $H$ under the canonical homomorphism
$\phi : G \rightarrow G/N$ is a subgroup of order $p^r$ in $G$. 
\end{proof}
 
 
\medskip
 
 
A {\bfi Sylow $p$-subgroup}\index{Subgroup!Sylow
$p$-subgroup}\index{Sylow $p$-subgroup}  $P$ of a group $G$ is a
maximal $p$-subgroup of $G$. To prove the other two Sylow Theorems, we
need to consider conjugate subgroups as opposed to conjugate elements
in a group. For a group $G$, let ${\mathcal S}$ be the collection of all
subgroups of $G$. For any subgroup $H$, ${\mathcal S}$ is a $H$-set, where 
$H$ acts on ${\mathcal S}$ by conjugation. That is, we have an action 
\[
H \times  {\mathcal S}  \rightarrow  {\mathcal S}
\]
defined by
\[
h \cdot K \mapsto hKh^{-1}
\]
for $K$ in ${\mathcal S}$.
 
 
The set
\[
N(H) = \{  g \in G : g H g^{-1} = H\}\label{notenormalizer}
\]
is a subgroup of $G$. Notice that $H$ is a normal subgroup of $N(H)$.
In fact, $N(H)$ is the largest subgroup of $G$ in which $H$ is normal.
We call $N(H)$ the {\bfi normalizer\/}\index{Subgroup!normalizer
of}\index{Normalizer} of $H$ in $G$.
 
 
\begin{lemma}\label{p_order_lemma}
Let $P$ be a Sylow $p$-subgroup of a finite group $G$ and let $x$ have
as its order a power of $p$. If $x^{-1} P x = P$. Then $x \in P$.
\end{lemma}
 
 
\begin{proof}
Certainly $x \in N(P)$, and the cyclic subgroup, $\langle xP \rangle
\subset N(P)/P$, has as its order a power of $p$. By the Correspondence
Theorem there exists a subgroup $H$ of $N(P)$ such that $H/P = \langle
xP \rangle$. Since $|H| = |P| \cdot |\langle xP \rangle|$, the order
of $H$ must be a power of $p$. However, $P$ is a Sylow $p$-subgroup
contained in $H$. Since the order of $P$ is the largest power of $p$
dividing $|G|$, $H=P$.  Therefore,  $H/P$ is the trivial subgroup and
$xP = P$, or $x \in P$. 
\end{proof}
 
 
\begin{lemma}\label{distinct_conj_lemma}
Let $H$ and $K$ be subgroups of $G$. The number of distinct
$H$-conjugates of $K$ is $[H:N(K) \cap H]$. 
\end{lemma}
 
 
\begin{proof}
We define a bijection between the conjugacy classes of $K$ and the
right cosets of $N(K) \cap H$ by $h^{-1}Kh \mapsto (N(K) \cap H)h$.
To show that this map is a bijection, let $h_1, h_2 \in H$ and suppose
that $(N(K) \cap H)h_1 = (N(K) \cap H)h_2$. Then $h_2 h_1^{-1} \in
N(K)$. Therefore, $K = h_2 h_1^{-1} K h_1 h_2^{-1}$ or $h_1^{-1} K h_1
= h_2^{-1} K h_2$, and the map is an injection.  It is easy to see
that this map is surjective; hence, we have a one-to-one and onto map
between the $H$-conjugates of $K$ and the right cosets of $N(K) \cap
H$ in $H$.
\end{proof}
 
 
\begin{theorem}[Second Sylow Theorem]\label{sylow:Sylow_two_theorem}
Let $G$ be a finite group and $p$ a prime dividing $|G|$. Then all
Sylow $p$-subgroups of $G$ are conjugate. That is, if $P_1$ and $P_2$
are two Sylow $p$-subgroups, there exists a $g \in G$ such that $g P_1
g^{-1} = P_2$. 
\end{theorem}
 
 
\begin{proof}
Let $P$ be a Sylow $p$-subgroup of $G$ and suppose that $|G|=p^r m$
and $|P|=p^r$. Let 
\[
{\mathcal P} = \{ P = P_1, P_2, \ldots, P_k \}
\]
consist of the distinct conjugates of $P$ in $G$.  By Lemma~\ref{distinct_conj_lemma}, $k =
[G: N(P)]$. Notice that 
\[
|G|= p^r m = |N(P)| \cdot [G: N(P)]= |N(P)| \cdot k.
\]
Since $p^r$ divides $|N(P)|$, $p$ cannot divide $k$. Given any other
Sylow $p$-subgroup $Q$, we must show that $Q \in {\mathcal P}$. Consider
the $Q$-conjugacy classes of each $P_i$. Clearly, these conjugacy
classes partition ${\mathcal P}$. The size of the partition containing
$P_i$ is $[Q :N(P_i) \cap Q]$. Lagrange's Theorem tells us that this
number is a divisor of $|Q|= p^r$. Hence, the number of conjugates in
every equivalence class of the partition is a power of $p$. However,
since $p$ does not divide $k$, one of these equivalence classes must
contain only a single Sylow $p$-subgroup, say $P_j$. Therefore, for
some $P_j$, $x^{-1} P_j x = P_j$ for all $x \in Q$. By Lemma~\ref{p_order_lemma},
$P_j = Q$. 
\end{proof}
 
 
\begin{theorem}[Third Sylow Theorem]
Let $G$ be a finite group and let $p$ be a prime dividing the order of
$G$.  Then the number of Sylow $p$-subgroups is congruent to $1
\pmod{p}$ and divides $|G|$. 
\end{theorem}
 
 
\begin{proof}
Let $P$ be a Sylow $p$-subgroup acting on the set of Sylow
$p$-subgroups, 
\[
{\mathcal P} = \{ P = P_1, P_2, \ldots, P_k \},
\]
by conjugation. From the proof of the Second Sylow Theorem, the only
$P$-conjugate of $P$ is itself and the order of the other
$P$-conjugacy classes is a power of $p$. Each $P$-conjugacy class
contributes a positive power of $p$ toward $|{\mathcal P}|$ except the
equivalence class  $\{ P \}$. Since $|{\mathcal P}|$ is the sum of
positive powers of $p$ and 1, $|{\mathcal P}| \equiv 1 \pmod{p}$.
 
 
Now suppose that $G$ acts on ${\mathcal P}$ by conjugation. Since all
Sylow $p$-subgroups are conjugate, there can be only one orbit under
this action. For $P \in {\mathcal P}$, 
\[
|{\mathcal P}| = |\mbox{orbit of P}| = [G : N(P)].
\]
But $[G : N(P)]$ is a divisor of $|G|$; consequently, the number of
Sylow \mbox{$p$-subgroups} of a finite group must divide the order of the
group. 
\end{proof}
 
 
 
\histhead 
 
 
\noindent {\small \histf
Peter Ludvig Mejdell Sylow\index{Sylow, Ludvig} was born in 1832 in
Christiania, Norway (now Oslo). After attending Christiania
University, Sylow taught high school. In 1862 he obtained a temporary
appointment at Christiania University. Even though his appointment was
relatively brief, he influenced students such as Sophus
Lie\index{Lie, Sophus} (1842--1899).   Sylow had a chance at a
permanent chair in 1869, but failed to obtain the appointment. In 1872,
he published a 10-page paper presenting the theorems that now bear his
name.  Later Lie and Sylow collaborated on a new edition of Abel's
works.  In 1898, a chair at Christiania University was finally created
for Sylow through the efforts of his student and colleague Lie. Sylow
died in 1918.
\histbox
} 
 
 
 
\section{Examples and Applications}
 
 
\begin{example}{5sylow_subgroups}
Using the Sylow Theorems, we can determine that $A_5$ has subgroups of
orders $2$, $3$, $4$, and $5$. The Sylow $p$-subgroups of $A_5$ have
orders $3$, $4$, and $5$. The Third Sylow Theorem tells us exactly how
many Sylow $p$-subgroups $A_5$ has. Since the number of Sylow 
5-subgroups must divide 60 and also be congruent to $1 \pmod{5}$,
there are either one or six Sylow 5-subgroups in $A_5$. All Sylow
5-subgroups are conjugate.  If there were only a single Sylow 
5-subgroup, it would be conjugate to itself; that is, it would be a
normal subgroup of $A_5$. Since $A_5$ has no normal subgroups, this is 
impossible; hence, we have determined that there are exactly six
distinct Sylow 5-subgroups of $A_5$. 
\end{example}
 
 
The Sylow Theorems allow us to prove many useful results about finite
groups. By using them, we can often conclude a great deal about groups 
of a particular order if certain hypotheses are satisfied.
 
 
\begin{theorem}\label{sylow:pq_cyclic_theorem}
If $p$ and $q$ are distinct primes with $p<q$, then every group $G$ of
order $pq$ has a single subgroup of order $q$ and this subgroup is
normal in $G$.  Hence, $G$ cannot be simple.  Furthermore, if $q
\not\equiv 1 \pmod{p}$, then $G$ is cyclic.
\end{theorem}
 
 
\begin{proof}
We know that $G$ contains a subgroup $H$ of order $q$. The number of
conjugates of $H$  divides $pq$ and is equal to $1 +kq$ for $k = 0, 1,
\ldots$. However, $1+q$ is already too large to divide the order of
the group; hence, $H$ can only be conjugate to itself.  That is, $H$
must be normal in $G$. 
 
 
The group $G$ also has a Sylow $p$-subgroup, say $K$. The number of
conjugates of $K$ must divide $q$ and be equal to $1 + kp$ for $k = 0,
1, \ldots$. Since $q$ is prime, either $1 +kp=q$ or $1+kp =1$. If $1
+kp=1$, then $K$ is normal in $G$. In this case, we can easily show
that $G$ satisfies the criteria, given in Chapter~8, for the
internal direct product of $H$ and $K$.  Since $H$ is isomorphic to
${\mathbb Z}_q$ and $K$ is isomorphic to ${\mathbb Z}_p$, $G \cong {\mathbb
Z}_p \times {\mathbb Z}_q  \cong {\mathbb Z}_{pq}$ by Theorem~\ref{Z_pq_theorem}. 
\mbox{\hspace*{1in}}
\end{proof}
 
 
\begin{example}{Z15}
Every group of order 15 is cyclic.  This is true because $15 = 5 \cdot
3$ and $5 \not\equiv 1 \pmod{3}$. 
\end{example}
 
 
\begin{example}{G99_subgroups}
Let us classify all of the groups of order $99 = 3^2 \cdot 11$ up to
isomorphism. First we will show that every group $G$ of order 99 is
abelian.  By the Third Sylow Theorem, there are $1 + 3k$ Sylow
3-subgroups, each of order $9$, for some $k = 0, 1, 2, \ldots$.  Also,
$1 + 3k$ must divide 11; hence, there can only be a single normal
Sylow 3-subgroup $H$ in $G$.  Similarly, there are $1 +11k$ Sylow
11-subgroups and $1 +11k$ must divide $9$.  Consequently, there is
only one Sylow 11-subgroup $K$ in $G$.  By Corollary~\ref{actions:p2abelian}, any group
of order $p^2$ is abelian for $p$ prime; hence, $H$ is isomorphic either 
to ${\mathbb Z}_3 \times {\mathbb Z}_3$ or to ${\mathbb Z}_9$.  Since $K$ has 
order 11, it must be isomorphic to ${\mathbb Z}_{11}$.  Therefore, the only
possible groups of order 99 are ${\mathbb Z}_3 \times {\mathbb Z}_3 \times
{\mathbb Z}_{11}$ or ${\mathbb Z}_9 \times {\mathbb Z}_{11}$ up to isomorphism.
\end{example}
 

 
 
To determine all of the groups of order $5 \cdot 7 \cdot 47 = 1645$,
we need the following theorem. 
 
 
\begin{theorem}\label{sylow:commutator_subgroup_theorem}
Let $G' = \langle a b a^{-1} b^{-1} : a, b \in G \rangle$ be the
subgroup  consisting of all finite products of elements of the form
$aba^{-1}b^{-1}$ in a group $G$. Then $G'$ is a normal subgroup of $G$
and $G/G'$ is abelian. 
\end{theorem}
 
 
The subgroup $G'$ of $G$ is called the {\bfi commutator
subgroup\/}\index{Subgroup!commutator} of $G$. We leave the proof of
this theorem as an exercise. 
 
 
\begin{example}{G1645_subgroups}
We will now show that every group of order $5 \cdot 7 \cdot 47 =
1645$ is abelian, and cyclic by Corollary~\ref{RelativelyPrime}. By the Third Sylow
Theorem, $G$ has only one subgroup $H_1$ of order $47$.  So $G/H_1$
has order 35 and must be abelian by Theorem~\ref{sylow:pq_cyclic_theorem}. Hence, the
commutator subgroup of $G$ is contained in $H$ which tells us that
$|G'|$ is either 1 or 47. If $|G'|=1$, we are done. Suppose that
$|G'|=47$. The Third Sylow Theorem tells us that $G$ has only one
subgroup of order 5 and one subgroup of order 7.  So there exist
normal subgroups $H_2$  and $H_3$ in $G$, where $|H_2| = 5$ and $|H_3|
= 7$. In either case the quotient group is abelian; hence, $G'$ must
be a subgroup of $H_i$, $i= 1, 2$. Therefore, the order of $G'$ is 1,
5, or 7. However, we already have determined that $|G'| =1$ or 47. So
the commutator subgroup of $G$ is trivial, and consequently $G$ is
abelian.
\end{example}
 
 
 
\subsection*{Finite Simple Groups}
 
 
Given a finite group, one can ask whether or not that group has any
normal subgroups. Recall that a simple group is one with no proper
nontrivial normal subgroups. As in the case of $A_5$, proving a group
to be simple can be a very difficult task; however, the Sylow Theorems
are useful tools for proving that a group is not simple. Usually
some sort of counting argument is involved.
 
 
\begin{example}{G20_not_simple}
Let us show that no group $G$ of order 20 can be simple.  By the Third
Sylow Theorem, $G$ contains one or more Sylow $5$-subgroups. The number
of such subgroups is congruent to $1 \pmod{5}$ and must also divide
20.  The only possible such number is 1.  Since there is only a
single Sylow 5-subgroup and all Sylow 5-subgroups are conjugate, this
subgroup must be normal. 
\end{example}
 
 
 
\begin{example}{G_pn}
Let $G$ be a finite group of order $p^n$, $n>1$ and $p$ prime.  By 
Theorem~\ref{pn_theorem}, $G$ has a nontrivial center. Since the center of any
group $G$ is a normal subgroup, $G$ cannot be a simple group.
Therefore, groups of orders 4, 8, 9, 16, 25, 27, 32, 49, 64, and 81
are not simple.  In fact, the groups of order 4, 9, 25, and 49 are 
abelian by Corollary~\ref{actions:p2abelian}.
\end{example}
 
 
 
\begin{example}{G56}
No group of order $56= 2^3 \cdot 7$ is simple.  We have seen that if
we can show that there is only one Sylow $p$-subgroup for some prime
$p$ dividing 56, then this must be a normal subgroup and we are done.
By the Third Sylow Theorem, there are either one or eight Sylow
7-subgroups.  If there is only a single Sylow 7-subgroup, then it must
be normal.  
 
 
On the other hand, suppose that there are eight Sylow 7-subgroups.
Then each of these subgroups must be cyclic; hence, the intersection
of any two of these subgroups contains only the identity of the group.
This leaves $8 \cdot 6 = 48$ distinct elements in the group, each of
order 7. Now let us count Sylow 2-subgroups. There are either one or
seven Sylow 2-subgroups.  Any element of a Sylow 2-subgroup other than
the identity must have as its order a  power of 2; and therefore
cannot be
one of the 48 elements of order 7 in the Sylow 7-subgroups. Since a
Sylow 2-subgroup has order 8, there is only enough room for a single
Sylow 2-subgroup in a group of order 56.  If there is only one Sylow
2-subgroup, it must be normal. 
\end{example}
 
 

 
 
For other groups $G$ it is more difficult to prove that $G$ is not
simple. Suppose $G$ has order 48. In this case the technique that we
employed in the last example will not work.  We need the following
lemma to prove that no group of order 48 is simple.  
 
 
\begin{lemma}\label{sylow:hk_subgroup_lemma}
Let $H$ and $K$ be finite subgroups of a group $G$. Then
\[
|HK| = \frac{|H| \cdot |K|}{|H \cap K|}.
\]
\end{lemma}
 
 
\begin{proof}
Recall that
\[
HK = \{ hk : h \in H, k \in K \}.
\]
Certainly, $|HK| \leq |H| \cdot |K|$ since some element in $HK$
could be written as the product of different elements in $H$ and $K$.
It is quite possible that $h_1 k_1 = h_2 k_2$ for $h_1, h_2 \in H$ and
$k_1, k_2 \in K$.  If this is the case, let
\[
a = (h_1)^{-1} h_2 = k_1 (k_2)^{-1}.
\]
Notice that $a \in H \cap K$, since $(h_1)^{-1} h_2$ is in $H$ and
$k_2 (k_1)^{-1}$ is in $K$; consequently, 
\begin{align*}
h_2 & =  h_1 a^{-1} \\
k_2 & =  a k_1.
\end{align*}
 
 
Conversely, let $h = h_1 b^{-1}$ and $k = b k_1$ for $b \in  H
\cap K$. Then $h k = h_1 k_1$, where $h \in H$ and $k \in K$. Hence,
any element $hk \in HK$ can be written in the form $h_i k_i$ for $h_i
\in H$ and $k_i \in K$, as many times as there are elements in $H
\cap K$; that is, $|H \cap K|$ times. Therefore, $|HK| = (|H| \cdot
|K|)/|H \cap K|$. 
\mbox{\hspace*{1in}}
\end{proof}
 
 
 
\begin{example}{G48}
To demonstrate that a group $G$ of order 48 is not simple, we will
show that $G$ contains either a normal subgroup of order 8 or a normal
subgroup of order 16.  By the Third Sylow Theorem, $G$ has either one
or three Sylow 2-subgroups of order 16.  If there is only one
subgroup, then it must be a normal subgroup. 

%%TWJ 11/30/2011 - Reference to the Lemma fixed.  Suggested by L. Franklin.

 
 
Suppose that the other case is true, and two of the three Sylow
2-subgroups are $H$ and $K$. We claim that  $|H \cap K| = 8$.  If $|H
\cap K| \leq 4$, then by Lemma~\ref{sylow:hk_subgroup_lemma}, 
\[
|HK| = \frac{16 \cdot 16}{4} =64,
\]
which is impossible.  So $H \cap K$ is normal in both $H$ and $K$
since it has index 2. The normalizer of $H \cap K$ contains both $H$
and $K$, and $|H \cap K|$ must both be a multiple of 16 greater than
1 and divide 48. The only possibility is that $|N(H \cap K)|= 48$.
Hence, $N(H \cap K) = G$.
\end{example}
 
 
\medskip
 
 
The following famous conjecture of Burnside was proved in a long and
difficult paper by Feit and Thompson [2].   
 
 
\begin{theorem} 
{\bf (Odd Order Theorem) }\index{Odd Order Theorem}
Every finite simple group of nonprime order must be of even order. 
\end{theorem}
 
 
The proof of this theorem laid the groundwork for a program in the 
1960s and 1970s that classified all finite simple groups.
The success of this program is one of the outstanding achievements of
modern mathematics. 
 
 
 
\markright{EXERCISES}
\section*{Exercises}
\exrule
 
 
{\small
\begin{enumerate}

\item
What are the orders of all Sylow $p$-subgroups where $G$ has order 18, 24, 54, 72, and 80? 
 
\item
Find all the Sylow 3-subgroups of $S_4$ and show that they are all conjugate. 
 
\item
Show that every group of order 45 has a normal subgroup of order 9.
 
\item
Let $H$ be a Sylow $p$-subgroup of $G$.  Prove that $H$ is the only Sylow $p$-subgroup of $G$ contained in $N(H)$. 
 
\item
Prove that no group of order 96 is simple.

\item
Prove that no group of order 160 is simple.

\item
If $H$ is a normal subgroup of a finite group $G$ and $|H| = p^k$ for some prime $p$, show that $H$ is contained in every Sylow $p$-subgroup of $G$. 

\item
Let $G$ be a group of order $p^2 q^2$, where $p$ and $q$ are distinct primes such that $q \notdivide p^2 - 1$ and $p \notdivide q^2 - 1$. Prove that $G$ must be abelian.  List three pairs of primes satisfying these  conditions. 

\item
Show directly that a group of order 33 has only one Sylow 3-subgroup. 

\item
Let $H$ be a subgroup of a group $G$.  Prove or disprove that the normalizer of $H$ is normal in $G$. 
 
 
\item
Let $G$ be a finite group divisible by a prime $p$.  Prove that if there is only one Sylow $p$-subgroup in $G$, it must be a normal subgroup of $G$. 
 
\item
Let $G$ be a group of order $p^r$, $p$ prime. Prove that $G$ contains
a normal subgroup of order $p^{r-1}$. 
 
\item
Suppose that $G$ is a finite group of order $p^n k$, where $k < p$.
Show that $G$ must contain a normal subgroup.  
 
\item
Let $H$ be a subgroup of a finite group $G$. Prove that $g N(H) g^{-1}
= N(gHg^{-1})$ for any $g \in G$. 

\item
Prove that a group of order 108 must have a normal subgroup.

\item
Classify all the groups of order 175 up to isomorphism.

\item
Show that every group of order $255$ is cyclic.

\item
Let $G$ have order $p_1^{e_1} \cdots p_n^{e_n}$ and suppose that $G$
has $n$ Sylow \mbox{$p$-subgroups} $P_1, \ldots, P_n$ where 
$|P_i| = p_i^{e_i}$.  Prove that $G$ is isomorphic to 
$P_1~\times~\cdots~\times~P_n$. 
 
\item
Let $P$ be a normal Sylow $p$-subgroup of $G$.  Prove that every inner
automorphism of $G$ fixes $P$.  
 
 
\item
What is the smallest possible order of a group $G$ such that $G$ is
nonabelian and $|G|$ is odd?  Can you find such a group?
 
 
\item
{\bf The Frattini Lemma.}  
If $H$ is a normal subgroup of a finite group $G$ and $P$ is a Sylow
$p$-subgroup of $H$, for each $g \in G$ show that there is an $h$ in
$H$ such that $gPg^{-1} = hPh^{-1}$.  Also, show that if $N$ is the
normalizer of $P$, then $G= HN$. 
 
 
\item
Show that if the order of $G$ is $p^nq$, where $p$ and $q$ are primes
and $p>q$, then $G$ contains a normal subgroup. 
 
 
\item
Prove that the number of distinct conjugates of a subgroup $H$ of a 
finite group $G$ is $[G : N(H) ]$. 
 
 
\item
Prove that a Sylow 2-subgroup of $S_5$ is isomorphic to $D_4$.
 
 
\item
{\bf Another Proof of the Sylow Theorems.}
\begin{enumerate}
 
\item
Suppose $p$ is prime and $p$ does not divide $m$.  Show that
\[
p \notdivide
\binom{p^k m}{p^k}.
\]
 
\item
Let ${\mathcal S}$ denote the set of all $p^k$ element subsets of $G$.
Show that $p$ does not divide $|{\mathcal S}|$. 
 
\item
Define an action of $G$ on ${\mathcal S}$ by left multiplication, $aT = \{
at : t \in T \}$ for $a \in G$ and $T \in {\mathcal S}$.  Prove that this
is a group action. 
 
\item
Prove $p \notdivide | {\mathcal O}_T|$ for some $T \in {\mathcal S}$.
 
\item
Let $\{ T_1, \ldots, T_u \}$ be an orbit such that $p \notdivide u$ and
$H = \{ g \in G : gT_1 = T_1 \}$.  Prove that $H$ is a subgroup of $G$ 
and show that $|G| = u |H|$. 
 
\item
Show that $p^k$ divides $|H|$ and $p^k \leq |H|$.
 
\item
Show that $|H| = |{\mathcal O}_T| \leq p^k$; conclude that therefore 
$p^k = |H|$. 
 
\end{enumerate}
 
 
\item
Let  $G$ be a group.  Prove that $G' = \langle a b a^{-1} b^{-1} : 
a, b \in G \rangle$ is a normal subgroup of $G$ and $G/G'$ is abelian.
Find an example to show that $\{ a b a^{-1} b^{-1} : 
a, b \in G \}$ is not necessarily a group.
 
 
\end{enumerate}
}
 
 
 
\subsection*{A Project}
 
 
\begin{table}[htb]
{\small
\begin{center}
\begin{tabular}{|cc|cc|cc|cc|}
\hline
Order & Number & Order & Number & Order & Number & Order &
Number \\
\hline
1  & ?  & 16 & 14 & 31 & 1  & 46 & 2  \\
2  & ?  & 17 & 1  & 32 & 51 & 47 & 1  \\
3  & ?  & 18 & ?  & 33 & 1  & 48 & 52 \\
4  & ?  & 19 & ?  & 34 & ?  & 49 & ?  \\
5  & ?  & 20 & 5  & 35 & 1  & 50 & 5  \\
6  & ?  & 21 & ?  & 36 & 14 & 51 & ?  \\
7  & ?  & 22 & 2  & 37 & 1  & 52 & ?  \\
8  & ?  & 23 & 1  & 38 & ?  & 53 & ?  \\
9  & ?  & 24 & ?  & 39 & 2  & 54 & 15 \\
10 & ?  & 25 & 2  & 40 & 14 & 55 & 2  \\
11 & ?  & 26 & 2  & 41 & 1  & 56 & ?  \\
12 & 5  & 27 & 5  & 42 & ?  & 57 & 2  \\
13 & ?  & 28 & ?  & 43 & 1  & 58 & ?  \\
14 & ?  & 29 & 1  & 44 & 4  & 59 & 1  \\
15 & 1  & 30 & 4  & 45 & *  & 60 & 13 \\
\hline
\end{tabular}
\caption{Numbers of distinct groups $G$, $|G| \leq 60$ }
\end{center}}\label{sylow:project}
\end{table}
 
 
{\small
 
\noindent 
The main objective of finite group theory is to classify all possible
finite groups up to isomorphism.  This problem is very difficult even
if we try to classify the groups of order less than or equal to 60.
However, we can break the problem down into several intermediate
problems. 
\begin{enumerate}
 
\item
Find all simple groups $G$ ( $|G| \leq 60$). {\em Do not use the Odd
Order Theorem unless you are prepared to prove it.}
 
\item
Find the number of distinct groups $G$, where the order of $G$ is $n$
for $n = 1, \ldots, 60$. 
 
\item
Find the  actual groups (up to isomorphism) for each $n$.
 
\end{enumerate}
This is a challenging project that requires a working knowledge of the
group theory you have learned up to this point. Even if you do
not complete it, it will teach you a great deal about finite groups. You
can use Table~\ref{sylow:project} as a guide. 
}
 
 
 
\subsection*{References and Suggested Readings}  %%TWJ 8/13/2010 - References checked and updated
 
 
{\small
\begin{itemize}
 
\item[{\bf [1]}]
Edwards, H. ``A Short History of the Fields Medal,'' {\it
Mathematical Intelligencer} {\bf 1} (1978), 127--29.
 
\item[{\bf [2]}]
Feit, W. and Thompson, J. G. ``Solvability of Groups of Odd
Order,'' {\it Pacific Journal of Mathematics} {\bf
13} (1963), 775--1029.
 
\item[{\bf [3]}]
Gallian, J. A. ``The Search for Finite Simple Groups,''
{\it Mathematics Magazine} {\bf 49}(1976), 163--79.
 
\item[{\bf [4]}]
Gorenstein, D. ``Classifying the Finite Simple Groups,''
{\it Bulletin of the American Mathematical Society} {\bf
14} (1986), 1--98.
 
\item[{\bf [5]}]
Gorenstein, D. {\it Finite Groups}. AMS Chelsea Publishing, Providence RI, 1968.

\item[{\bf [6]}]
Gorenstein, D., Lyons, R., and Solomon, R. {\it The Classification of Finite Simple
Groups}.
American Mathematical Society, Providence RI, 1994.
 
\end{itemize}
}
 
\sagesection
 
