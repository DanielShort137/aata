%%%%(c)
%%%%(c)  This file is a portion of the source for the textbook
%%%%(c)
%%%%(c)    Abstract Algebra: Theory and Applications
%%%%(c)    by Thomas W. Judson
%%%%(c)
%%%%(c)    Sage Material
%%%%(c)    Copyright 2011 by Robert A. Beezer
%%%%(c)
%%%%(c)  See the file COPYING.txt for copying conditions
%%%%(c)
%%%%(c)
\begin{sageverbatim}\end{sageverbatim}
%
\sageexercise{1}%
An automorphism is an isomorphism between a group and itself.  The identity function ($x\mapsto x$) is always an isomorphism, which we consider trivial.  Use Sage to construct a nontrivial automorphism of the cyclic group of order 12.  Check that the mapping is both onto and one-to-one by computing the image and kernel and performing the proper tests on these subgroups.  Now construct all of the possible automorphisms of the cyclic group of order 12.
\begin{sageverbatim}\end{sageverbatim}
%
\sageexercise{2}%
The four homomorphisms created by the direct product construction are each an example of a more general construction of homomorphisms involving groups $G$, $H$ and $G\times H$.  By using the same groups as in the example above, see if you can discover and describe these constructions with exact definitions of the four homomorphisms in general.\par
%
Your tools for investigating a Sage group homomorphism are limited, you might take each generator of the domain and see what its image is.  Here is an example of the type of computation you might do repeatedly.  We'll investigate the second homomorphism.  The domain is the dihedral group, and we will compute the image of the first generator.
%
\begin{sageexample}
sage: G = CyclicPermutationGroup(3)
sage: H = DihedralGroup(4)
sage: results = G.direct_product(H)
sage: phi = results[2]
sage: H.gens()
[(1,2,3,4), (1,4)(2,3)]
sage: a = H.gen(0); a
(1,2,3,4)
sage: phi(a)
(4,5,6,7)
\end{sageexample}
%
\begin{sageverbatim}\end{sageverbatim}
%
\sageexercise{3}%
Consider two permutation groups.  The first is the subgroup of $S_7$ generated by $(1, 2, 3)$ and $(4, 5, 6, 7)$.  The second is a subgroup of $S_{12}$ generated by $(1, 2, 3)(4, 5, 6)(7, 8, 9)(10, 11, 12)$ and $(1, 10, 7, 4)(2, 11, 8, 5)(3, 12, 9, 6)$.  Build these two groups and use the proper Sage command to see that they are isomorphic.  Then construct a homomorphism between these two groups that is an isomorphism and include enough details to verify that the mapping is really an isomorphism.
\begin{sageverbatim}\end{sageverbatim}
%
\sageexercise{4}%
The second paragraph of this chapter informally describes a homomorphism from $S_n$ to ${\mathbb Z}_2$, where the even permutations all map to the one of the elements and the odd permutations all map to the other element.  Replace $S_n$ by $S_6$ and replace ${\mathbb Z}_2$ by the permutation version of the cyclic subgroup of order 2, and construct a nontrivial homomorphism between these two groups.  Evaluate your homomorphism with enough even and odd permutations to be convinced that it is correct.  Then construct the kernel and verify that it is the group you expect.\par
%
Hints:  First, decide which element of the group of order 2 will be associated with even permutations and which will be associated with odd permutations.  Then examine the generators of $S_6$ to help decide just how to build the homomorphism.
\begin{sageverbatim}\end{sageverbatim}
%
\sageexercise{5}%
The dihedral group $D_{20}$ has several normal subgroups, as seen below.  Each of these is the kernel of a homomorphism with $D_{20}$ as the domain.  For each normal subgroup of $D_{20}$ construct a homomorphism from $D_{20}$ to $D_{20}$ that has the normal subgroup as the kernel.  There is a pattern to many of these, but the three of order 20 will be a challenge.
%
\begin{sageexample}
sage: G = DihedralGroup(20)
sage: [H.order() for H in G.normal_subgroups()]
[1, 2, 4, 5, 10, 20, 20, 20, 40]
\end{sageexample}
%
%% Other isomorphism theorems, once have intersection of groups
%
\begin{sageverbatim}\end{sageverbatim}
