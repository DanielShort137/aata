%%%%(c)
%%%%(c)  This file is a portion of the source for the textbook
%%%%(c)
%%%%(c)    Abstract Algebra: Theory and Applications
%%%%(c)    by Thomas W. Judson
%%%%(c)
%%%%(c)    Sage Material
%%%%(c)    Copyright 2011 by Robert A. Beezer
%%%%(c)
%%%%(c)  See the file COPYING.txt for copying conditions
%%%%(c)
%%%%(c)
\begin{sageverbatim}\end{sageverbatim}
%
\sageexercise{1}%
This exercise verifies Theorem~\extref{sylow:commutator_subgroup_theorem}{15.9}{quotient by commutator is abelian}.  The commutator subgroup is computed with the permutation group method \verb?.commutator()?.  For the dihedral group of order 40, $D_{40}$ (\verb?DihedralGroup(20)? in Sage), compute the commutator subgroup and form the quotient with the dihedral group.  Then verify that this quotient is abelian.  Can you identify the quotient group exactly (in other words, up to isomorphism)?
\begin{sageverbatim}\end{sageverbatim}
%
\sageexercise{2}%
For each possible prime, find all of the distinct Sylow $p$-subgroups of the alternating group $A_5$.  Confirm that your results are consistent with the Third Sylow Theorem for each prime.  We know that $A_5$ is a simple group.  Explain how this would explain or predict some aspects of your answers.\par
%
Count the number of distinct elements contained in the union of all the Sylow subgroups you just found.  What is interesting about this count?
\begin{sageverbatim}\end{sageverbatim}
%
\sageexercise{3}%
For each possible prime, find all of the distinct Sylow $p$-subgroups of the dihedral group $D_{72}$ (symmetries of a $36$-gon) for each possible prime.  Confirm that your results are consistent with the Third Sylow Theorem for each prime.  It can be proved that {\em any group} with order $72$ is not a simple group, using techniques such as those used in the later examples in this chapter.  Explain how this result would explain or predict some aspects of your answers.\begin{sageverbatim}\end{sageverbatim}
%
\sageexercise{4}%
This exercise verifies Lemma~\extref{distinct_conj_lemma}{15.5}{number of conjugacy classes}.  Let $G$ be the dihedral group of order $36$, $D_{36}$.  Let $H$ be the one Sylow $3$-subgroup.  Let $K$ be the subgroup of order $6$ generated by the two permutations \verb?a? and \verb?b? given below.  First, form a list of the distinct conjugates of $K$ by the elements of $H$, and determine the number of subgroups in this list.  Compare this with the index given in the statement of the lemma, employing a single (long) statement making use of the \verb?.order()?, \verb?.normalizer()? and \verb?.intersection()? methods.\par
%
\begin{sageexample}
sage: G = DihedralGroup(18)
sage: a = G("(1,7,13)(2,8,14)(3,9,15)(4,10,16)(5,11,17)(6,12,18)")
sage: b = G("(1,5)(2,4)(6,18)(7,17)(8,16)(9,15)(10,14)(11,13)")
\end{sageexample}
%
\begin{sageverbatim}\end{sageverbatim}
%
\sageexercise{5}%
Example~\extref{example:sylow:G48}{9}{example on order 48 subgroup} shows that every group of order $48$ has a normal subgroup.  The dicyclic groups are an infinite family of non-abelian groups with order $4n$, which includes the quaternions when $n=2$.  So the permutation group \verb?DiCyclicGroup(12)? has order 48.  Use Sage to follow the logic of the proof in Example~\extref{example:sylow:G48}{9}{example on order 48 subgroup} and construct a normal subgroup in this group.  (In other words, do not just ask for a list of the normal subgroups, but trace through the implications in the example to arrive at the normal subgroup, and check your answer.)
\begin{sageverbatim}\end{sageverbatim}
%
