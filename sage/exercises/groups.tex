%%%%(c)
%%%%(c)  This file is a portion of the source for the textbook
%%%%(c)
%%%%(c)    Abstract Algebra: Theory and Applications
%%%%(c)    by Thomas W. Judson
%%%%(c)
%%%%(c)    Sage Material
%%%%(c)    Copyright 2011 by Robert A. Beezer
%%%%(c)
%%%%(c)  See the file COPYING.txt for copying conditions
%%%%(c)
%%%%(c)
These exercises are about becoming comfortable working with groups in Sage.
\begin{sageverbatim}\end{sageverbatim}
%
\sageexercise{1}%
Create the groups \verb?CyclicPermutationGroup(8)? and \verb?DihedralGroup(4)? and give the two groups names of your choosing.  We will understand these constructions better shortly, but for now just understand that they are both groups.
\begin{sageverbatim}\end{sageverbatim}
%
\sageexercise{2}%
Check that the groups have the same size with the \verb?.order()? method.  Determine which is abelian, and which is not, by using the \verb?.is_abelian()? method.
\begin{sageverbatim}\end{sageverbatim}
%
\sageexercise{3}%
Use the \verb?.cayley_table()? method to create the Cayley table for each group.
\begin{sageverbatim}\end{sageverbatim}
%
\sageexercise{5}%
Write a nicely formatted discussion (Shift-click on a blue bar to bring up the mini-word-processor, use dollar signs to embed bits of \TeX) identifying differences between the two groups that are discernible in properties of their Cayley tables.  In other words, what is {\em different} about these two groups that you can ``see'' in the Cayley tables?
\begin{sageverbatim}\end{sageverbatim}
%
\sageexercise{5}%
For each group, list the elements of one non-trivial subgroup (not just the identity, and not the whole group).  Write out your subgroup as a set of elements from the group, and do {\em not} use letters as the names.  If you need to translate between letters in the table and the actual elements, then employ the \verb?.list()? method for the group to matchup the letters and elements.  Expressing elements using the ``bottom row'' representation might make the most sense at this point.
\begin{sageverbatim}\end{sageverbatim}
%
\sageexercise{6}%
(Advanced)  See if you can actually construct your subgroup within Sage so that the \verb?.is_subgroup()? command returns \verb?True?.  (Hint: extract the right elements from the list of all the group elements and use all of them as generators of a subgroup.)
\begin{sageverbatim}\end{sageverbatim}
%
