%%%%(c)
%%%%(c)  This file is a portion of the source for the textbook
%%%%(c)
%%%%(c)    Abstract Algebra: Theory and Applications
%%%%(c)    by Thomas W. Judson
%%%%(c)
%%%%(c)    Sage Material
%%%%(c)    Copyright 2011 by Robert A. Beezer
%%%%(c)
%%%%(c)  See the file COPYING.txt for copying conditions
%%%%(c)
%%%%(c)
\begin{sageverbatim}\end{sageverbatim}
%
\sageexercise{1}%
This exercise is about putting Cayley's Theorem into practice.  First, read and study the theorem.  Realize that this result by itself is primarily of theoretical interest, but with some more theory we could get into some subtler aspects of this (a subject known as ``representation theory'').\par
%
You should create these representations mostly with pencil-and-paper work, using Sage as a fancy calculator and assistant.  You do not need to include all these computations in your worksheet.  Build the requested group representations and then include enough verifications in Sage to prove that that your representation correctly represents the group.\par
%
Begin by building a permutation representation of the quaternions, $Q$.  There are eight elements in $Q$ ($\pm 1, \pm I, \pm J, \pm K$), so you will be constructing a subgroup of $S_8$.  For each $a\in Q$ form the function $\lambda_a$, as defined in the proof of Cayley's theorem.  To do this, the two-line version of writing permutations could be useful as an intermediate step.  You will probably want to ``code'' each element of $Q$ with an integer in $\{1,2,\dots,8\}$.\par
%
One such representation is included in Sage as \verb?QuaternionGroup()? --- your answer should look very similar, but perhaps not identical.  Do not submit your answer to this, but I strongly suggest working this particular group representation until you are sure you have it right --- the problems below might be very difficult otherwise.  You can use Sage's \verb?.is_isomorphic()? method to check if your representations are correct.  However, do not use this as a substitute for the part of each question that asks you to investigate properties of your representation towards this end.\par
%
(a) Build a permutation representation of ${\mathbb Z}_2\times{\mathbb Z}_4$ (remember this group is additive, while the theorem uses multiplicative notation).  Then construct the group as a subgroup of a full symmetric group generated by exactly two generators.  Hint: which two elements of ${\mathbb Z}_2\times{\mathbb Z}_4$ might you use to generate all of ${\mathbb Z}_2\times{\mathbb Z}_4$?  Use commands in Sage to investigate various properties of your group, other than just \verb?.list()?, to provide evidence that your subgroup is correct --- include these in your submitted worksheet.\par
%
(b) Build a permutation representation of $U(24)$, the group of units mod 24.  Then construct the group as a subgroup of a full symmetric group created with three generators.  To determine these three generators, you will likely need to understand $U(24)$ as an internal direct product.  Use commands in Sage to investigate various properties of your group, other than just \verb?.list()?, to provide evidence that your subgroup is correct --- include these in your submitted worksheet.
\begin{sageverbatim}\end{sageverbatim}
%
\sageexercise{2}%
Consider the symmetries of a 10-gon, $D_{20}$ in your text, \verb?DihedralGroup(10)? in Sage.  Identify the permutation that is a 180 degree rotation and use it to generate a subgroup $R$ of order 2.  Then identify the permutation that is a 72 degree rotation, and any permutation that is a reflection of the 10-gon about a line.  Use these two permutations to generate a subgroup $S$ of order 10.  Use Sage to verify that the full dihedral group is the internal direct product of the subgroups $R$ and $S$.\par
%
We have a theorem which says that if a group is an internal direct product, then it is isomorphic to some external direct product.  Understand that this does not mean that you can use the converse in this problem.  In other words, establishing an isomorphism of $G$ with an external direct product does not prove that $G$ is an internal direct product.
\begin{sageverbatim}\end{sageverbatim}
%
