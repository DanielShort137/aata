%%%%(c)
%%%%(c)  This file is a portion of the source for the textbook
%%%%(c)
%%%%(c)    Abstract Algebra: Theory and Applications
%%%%(c)    Copyright 1997 by Thomas W. Judson
%%%%(c)
%%%%(c)  See the file COPYING.txt for copying conditions
%%%%(c)
%%%%(c)
\chap{The Integers}{integers}


 
 
The integers are the building blocks of mathematics. In this chapter we will investigate  the fundamental properties of the integers, including mathematical induction, the division algorithm, and the Fundamental Theorem of Arithmetic.


\section{Mathematical Induction}\label{integers:math_induction}

Suppose we wish to show that
\[
1 + 2 + \cdots + n = \frac{n(n + 1)}{2}
\]
for any natural number $n$. This formula is easily  verified for small numbers such as $n = 1$, 2, 3, or 4, but it is impossible to verify for all natural numbers on a case-by-case basis.  To prove the formula true in general, a more generic method is required.

Suppose we have verified the equation for the first $n$ cases.  We will attempt to show that we can generate the formula for the $(n + 1)$th case from this knowledge.  The formula is true for $n = 1$ since 
\[
1 = \frac{1(1 + 1)}{2}.
\]
If we have verified the first $n$ cases, then
\begin{align*}
1 + 2 + \cdots + n + (n + 1) & = \frac{n(n + 1)}{2} + n + 1 \\
& = \frac{n^2 + 3n + 2}{2} \\
& = \frac{(n + 1)[(n + 1) + 1]}{2}.
\end{align*}
This is exactly the formula for the $(n + 1)$th case.
 
This method of proof is known as \boldemph{mathematical induction}.  Instead of attempting to verify a statement about some subset $S$ of the positive integers ${\mathbb N}$ on a case-by-case basis, an impossible task if $S$ is an infinite set, we give a specific proof for the smallest integer being considered, followed by a generic argument showing that if the statement holds for a given case, then it must also hold for the next  case in the sequence.  We summarize mathematical induction in the following axiom. 

\medskip

\noindent
{\bf First Principle of Mathematical Induction.}\index{Induction!first principle of} 
Let $S(n)$ be a statement about integers for  $n \in {\mathbb N}$ and suppose $S(n_0)$ is true for some integer $n_0$.  If for all integers $k$ with $k \geq n_0$ $S(k)$ implies that $S(k+1)$ is true, then $S(n)$ is true for all integers $n$ greater than or equal to $n_0$.  

%wording change.  Suggested by P. Diethelm.  TWJ 22/4/2013

\begin{example}{induction_greater_than}
For all integers $n \geq 3$, $2^n > n + 4$. Since
\[
8 = 2^3 > 3 + 4 = 7,
\]
the statement is true for $n_0 = 3$.  Assume that $2^k > k + 4$ for $k \geq 3$.  Then $2^{k + 1} = 2 \cdot 2^{k} > 2(k + 4)$.  But 
\[
2(k + 4) = 2k + 8 > k + 5 = (k + 1) + 4
\]
since $k$ is positive.  Hence, by induction, the statement holds for all integers $n \geq 3$. 
\end{example}

\begin{example}{induction_divisible}
Every integer $10^{n + 1} + 3 \cdot 10^n + 5$ is divisible by 9 for $n \in {\mathbb N}$.  For $n = 1$, 
\[
10^{1 + 1} + 3 \cdot 10 + 5 = 135 = 9 \cdot 15
\]
is divisible by 9.  Suppose that $10^{k + 1} + 3 \cdot 10^k + 5$ is divisible by 9 for $k \geq 1$.  Then 
\begin{align*}
10^{(k + 1) + 1} + 3 \cdot 10^{k + 1} + 5
& =
10^{k + 2} + 3 \cdot 10^{k + 1} + 50 - 45 \\
& =
10 (10^{k + 1} + 3 \cdot 10^{k} + 5) - 45
\end{align*}
is divisible by 9.
\end{example}

\begin{example}{binomial_theorem}
We will prove the binomial theorem using mathematical induction; that is, 
\[
(a + b)^n = \sum_{k = 0}^{n} \binom{n}{k} a^k b^{n - k},
\]
where $a$ and $b$ are real numbers, $n \in \mathbb{N}$, and
\[
\binom{n}{k}
= \frac{n!}{k! (n - k)!}\label{factorial}
\]
is the binomial coefficient.\label{binomial}  We first show that
\[
\binom{n + 1}{k}
=
\binom{n}{k} + \binom{n}{k - 1}.
\]
This result follows from
\begin{align*}
\binom{n}{k} + \binom{n}{k - 1}
& =
\frac{n!}{k!(n - k)!}
+\frac{n!}{(k-1)!(n - k + 1)!} \\
& =
\frac{(n + 1)!}{k!(n + 1 - k)!} \\
& =
\binom{n + 1}{k}.
\end{align*}
If $n = 1$, the binomial theorem is easy to verify. Now assume that the result is true for $n$ greater than or equal to 1.  Then
\begin{align*}
(a + b)^{n + 1}
& = 
(a + b)(a + b)^n \\
& =
(a + b) 
\left(
\sum_{k = 0}^{n} \binom{n}{k} a^k b^{n - k}
\right) \\
& = 
\sum_{k = 0}^{n} \binom{n}{k} a^{k + 1} b^{n - k}   +
\sum_{k = 0}^{n} \binom{n}{k} a^k b^{n + 1 - k} \\
& = 
a^{n + 1} + \sum_{k = 1}^{n} \binom{n}{k - 1} a^{k} b^{n + 1 - k} 
 +
\sum_{k = 1}^{n} \binom{n}{k}  a^k b^{n + 1 - k} + b^{n + 1}\\
&  = 
a^{n + 1} + \sum_{k = 1}^{n} \left[ \binom{n}{k - 1}
+
\binom{n}{k} \right]
a^k b^{n + 1 - k} + b^{n + 1} \\
&  = 
\sum_{k = 0}^{n + 1}   \binom{n + 1}{k} a^k b^{n + 1- k}.
\end{align*}
\end{example}
 
We have an equivalent statement of the Principle of Mathematical Induction that is often very useful.
 
\medskip

\noindent 
{\bf Second Principle of Mathematical Induction.}\index{Induction!second principle of}  
Let $S(n)$ be a statement about integers for  $n \in {\mathbb N}$ and suppose $S(n_0)$ is true for some integer $n_0$.  If $S(n_0), S(n_0 + 1), \ldots, S(k)$ imply that $S(k + 1)$ for $k \geq n_0$, then the statement $S(n)$ is true for all integers $n \geq n_0$.   

%wording change.  Suggested by P. Diethelm.  TWJ 22/4/2013
 
\medskip

A  nonempty subset $S$ of ${\mathbb Z}$ is \boldemph{well-ordered\/}\index{Well-ordered set} if $S$ contains a least element.  Notice that the set ${\mathbb Z}$ is not well-ordered since it does not contain a smallest element.  However, the natural numbers are well-ordered. 
 
\medskip
 
\noindent {\bf Principle of Well-Ordering.} 
Every nonempty subset of the natural numbers is well-ordered. 

\medskip
 
The Principle of Well-Ordering is equivalent to the Principle  of Mathematical Induction. 
 
\begin{lemma}\label{integers:smallest_number_lemma}
The Principle of Mathematical Induction implies that $1$ is the least positive natural number. 
\end{lemma}

\begin{proof}
Let $S = \{ n \in {\mathbb N} : n \geq 1 \}$. Then $1 \in S$.  Now assume that $n \in S$; that is, $n \geq 1$.  Since $n+1 \geq 1$, $n+ 1 \in S$; hence, by induction, every natural number is greater than or equal to 1. 
\end{proof}

% Theorem reworded for clarity.  TWJ 12/17/2011
% Suggested by K. Halasz and R. Beezer.
\begin{theorem}\label{integers_PMI_implies_PWO}
The Principle of Mathematical Induction implies the Principle of Well-Ordering.  That is, every nonempty subset of $\mathbb N$ contains a least element.
\end{theorem}
 
\begin{proof}
We must show that if $S$ is a nonempty subset of the natural numbers, then $S$ contains a least element.  If $S$ contains 1, then the theorem is true by Lemma~\ref{integers:smallest_number_lemma}.  Assume that if $S$ contains an integer $k$ such that $1 \leq k \leq n$, then $S$ contains a least element.  We will show that if a set $S$ contains an integer less than or equal to $n + 1$, then $S$ has a least element.  If $S$ does not contain an integer less than $n+1$, then $n+1$ is the smallest integer in $S$.  Otherwise, since $S$ is nonempty, $S$ must contain an integer less than or equal to $n$. In this case, by induction, $S$ contains a least element. 
\hspace*{1in} 
\end{proof}

%wording change.  Suggested by P. Diethelm.  TWJ 22/4/2013
 
\medskip
 
Induction can also be very useful in formulating definitions.  For instance, there are two ways to define $n!$, the factorial of a positive integer $n$. 
\begin{itemize}
 
\item 
The \emph{explicit} definition: $n! = 1 \cdot 2 \cdot 3 \cdots (n - 1)
\cdot n$. 
 
\item 
The \emph{inductive} or \emph{recursive} definition: $1! = 1$ and
$n! = n(n - 1)!$ for $n > 1$. 
 
\end{itemize}
Every good mathematician or computer scientist knows that looking at problems recursively, as opposed to explicitly, often results in better understanding of complex issues.  

 
\section{The Division Algorithm}

An application of the Principle of Well-Ordering that we will use often is the division algorithm. 

\begin{theorem}[Division Algorithm]\index{Division algorithm!for integers}\label{integers:division_algorithm}
Let $a$ and $b$ be integers, with $b > 0$.  Then there exist unique integers $q$ and $r$ such that 
\[
a = bq + r
\]
where $0 \leq r < b$.
\end{theorem}

\begin{proof}
This is a perfect example of the existence-and-uniqueness type of proof.  We must first prove that the numbers $q$ and $r$ actually  exist. Then we must show that if  $q'$ and $r'$ are two other such numbers, then $q = q'$ and $r = r'$. 
 
\emph{Existence of q and r}.
Let
\[
S = \{ a - bk : k \in {\mathbb Z} \text{ and } a - bk \geq 0 \}.
\]
If $0 \in S$, then $b$ divides $a$, and  we can let $q = a/b$ and $r = 0$.  If $0 \notin S$, we can use the Well-Ordering Principle.  We must first show that $S$ is nonempty.  If $a > 0$, then $a - b \cdot 0 \in S$. If $a < 0$, then $a - b(2a) = a(1 - 2b) \in S$.  In either case $S \neq \emptyset$.  By the Well-Ordering Principle, $S$ must have a smallest member, say $r = a - bq$. Therefore, $a = bq + r$, $r \geq 0$. We now show that $r < b$. Suppose that $r > b$. Then   
\[
a - b(q + 1)= a - bq - b = r - b > 0.
\]
In this case we would have $a - b(q + 1)$ in the set $S$. But then $a - b(q + 1) < a - bq$, which would contradict the fact that $r = a - bq$ is the smallest member  of $S$. So $r \leq b$.  Since $0 \notin S$, $r \neq b$ and so $r < b$. 
 
\emph{Uniqueness of q and r}.
Suppose there exist integers $r$, $r'$, $q$, and $q'$ such that
\[
a = bq + r, 0 \leq r < b 
\qquad
\text{and}
\qquad
a = bq' + r', 0 \leq r' < b.
\]
Then $ bq + r =  bq' + r'$.  Assume that $r' \geq r$.  From the last equation we have $b(q - q') = r' - r$; therefore, $b$ must divide $r' - r$ and $0 \leq r'- r \leq r' < b$.  This is possible only if $r' - r = 0$.  Hence, $r = r'$ and  $q = q'$. 
\end{proof}

\medskip

Let $a$ and $b$ be integers.  If $b = ak$ for some integer $k$, we write $a \mid b$\label{divides}.  An integer $d$ is called a \boldemph{common divisor} of $a$ and $b$ if $d \mid a$ and $d \mid b$.  The \boldemph{greatest common divisor}\index{Greatest common divisor!of two integers} of integers $a$ and $b$ is a positive integer $d$ such that $d$ is a common divisor  of $a$ and $b$ and if $d'$ is any other common divisor of $a$ and $b$, then $d' \mid d$.  We write $d = \gcd(a, b)$\label{greatestcd}; for example, $\gcd( 24, 36) = 12$ and $\gcd(120, 102) = 6$.  We say that two integers $a$ and $b$ are \boldemph{relatively prime} if $\gcd( a, b ) = 1$. 

\begin{theorem}\label{integers:gcd_theorem}
Let $a$ and $b$ be nonzero integers. Then there exist integers $r$ and $s$ such that
\[
\gcd( a, b) = ar + bs.
\]
Furthermore, the greatest common divisor of $a$ and $b$ is unique.
\end{theorem}
 
\begin{proof} %Notation error in proof fixed (pointed out by Rocco Rossi) - TWJ 9/13/2010
Let
\[
S = \{ am + bn : m, n \in {\mathbb Z} \text{ and } am + bn	> 0 \}.
\]
Clearly, the set $S$ is nonempty; hence, by the Well-Ordering Principle $S$ must have a smallest member, say $d = ar + bs$.  We claim that $d = \gcd( a, b)$.  Write $a = dq + r'$ where $0 \leq r' < d$ . If $r '> 0$, then %r changed to r' - TWJ 1/31/2011
\begin{align*}
r '
& = a - dq \\
& = a - (ar + bs)q \\
& = a - arq - bsq \\
& = a( 1 - rq ) + b( -sq ),
\end{align*}
which is in $S$.  But this would contradict the fact that $d$ is the smallest member of $S$.  Hence, $r' = 0$ and $d$ divides $a$.  A similar argument shows that $d$ divides $b$.  Therefore, $d$ is a common divisor of $a$ and $b$.

Suppose that $d'$ is another common divisor of $a$ and $b$, and we want to show that $d' \mid d$. If we let $a = d'h$ and $b = d'k$, then
\[
d = ar + bs = d'hr + d'ks = d'(hr + ks).
\]
So $d'$ must divide $d$. Hence, $d$ must be the unique greatest common divisor of $a$ and $b$. 
\end{proof}

\begin{corollary}\label{integers:coprime_corollary}
Let $a$ and $b$ be two integers that are relatively prime. Then there exist  integers $r$ and $s$ such that $ar + bs = 1$. 
\end{corollary}
 
 
\subsection*{The Euclidean Algorithm}

Among other things, Theorem~\ref{integers:gcd_theorem} allows us to compute the greatest common divisor of two integers. 

\medskip

\begin{example}{gcd}
Let us compute the greatest common divisor of $945$ and $2415$.  First observe that 
\begin{align*}
2415 & = 945 \cdot 2 + 525 \\
945 & = 525 \cdot 1 + 420 \\
525 & = 420 \cdot 1 + 105 \\
420 & = 105 \cdot 4 + 0.
\end{align*}
Reversing our steps, 105 divides 420, 105 divides 525, 105 divides 945, and 105 divides 2415.  Hence, 105 divides both 945 and 2415.  If $d$ were another common divisor of 945 and 2415, then $d$ would also have to divide 105.  Therefore, $\gcd( 945, 2415 ) = 105$.

If we work backward through the above sequence of equations, we can also obtain numbers $r$ and $s$ such that $945 r + 2415 s = 105$.  Observe that 
\begin{align*}
105 & = 525 + (-1) \cdot 420 \\
& = 525 + (-1) \cdot [945 + (-1) \cdot 525] \\
& = 2 \cdot 525 + (-1) \cdot 945 \\
& = 2 \cdot [2415 + (-2) \cdot 945] + (-1) \cdot 945 \\
& = 2 \cdot 2415 + (-5) \cdot 945.
\end{align*}
So $r = -5$ and $s= 2$.  Notice that $r$ and $s$ are not unique, since $r = 41$ and $s = -16$ would also work.
\end{example}

To compute $\gcd(a,b) = d$, we are using repeated divisions to obtain a decreasing sequence of positive integers $r_1 > r_2 > \cdots > r_n = d$; that is,
\begin{align*}
b & = a q_1 + r_1 \\
a & = r_1 q_2 + r_2 \\
r_1 & = r_2 q_3 + r_3 \\
& \vdots  \\
r_{n - 2} & = r_{n - 1} q_{n} + r_{n} \\
r_{n - 1} & = r_n q_{n + 1}.
\end{align*}
To find $r$ and $s$ such that $ar + bs = d$, we begin with this last equation and substitute results obtained from the previous equations:
\begin{align*}
d & = r_n \\
& = r_{n - 2} - r_{n - 1} q_n \\
& = r_{n - 2} - q_n( r_{n - 3} - q_{n - 1} r_{n - 2} ) \\
& = -q_n r_{n - 3} + ( 1+ q_n q_{n-1} ) r_{n - 2}  \\
& \vdots  \\
& = ra + sb.
\end{align*}
The algorithm that we have just used to find the greatest common divisor $d$ of two integers $a$ and $b$ and to write $d$ as the linear combination of $a$ and $b$ is known as the \boldemph{Euclidean algorithm}\index{Euclidean algorithm}\index{Algorithm!Euclidean}.  
 
 
\subsection*{Prime Numbers}

Let $p$ be an integer such that $p > 1$.  We say that $p$ is a \boldemph{prime number}\index{Prime integer}, or simply $p$ is \boldemph{prime}, if the only positive numbers that divide $p$ are 1 and $p$ itself.  An integer $n > 1$ that is not prime is said to be \boldemph{composite}\index{Composite integer}.  

\begin{lemma}[Euclid]\label{integers:prime_divisor_theorem}
Let $a$ and $b$ be integers and $p$ be a prime number.  If $p \mid ab$, then either $p \mid a$ or $p \mid b$. 
\end{lemma}

\begin{proof}
Suppose that $p$ does not divide $a$.  We must show that $p \mid b$. Since $\gcd( a, p ) = 1$, there exist integers $r$ and $s$ such that $ar + ps = 1$.  So 
\[
b = b(ar + ps) = (ab)r + p(bs).
\]
Since $p$ divides both $ab$ and itself, $p$ must divide $b = (ab)r + p(bs)$. 
\end{proof}

\begin{theorem}[Euclid]\label{integers_inifinite_primes}
There exist an infinite number of primes.
\end{theorem}

\begin{proof}
We will prove this theorem by contradiction.  Suppose that there are only a finite number of primes, say $p_1, p_2, \ldots, p_n$.  Let $P = p_1  p_2  \cdots  p_n + 1$.    Then $P$ must be divisible by some $p_i$ for $1 \leq i \leq n$. In this  case, $p_i$ must divide $P - p_1 p_2 \cdots p_n = 1$, which is a contradiction.  Hence, either $P$ is prime or there exists an additional prime number $p \neq p_i$ that divides $P$.
\end{proof}
%Error in proof fixed.  Suggested by R. Rossi.  TWJ 12/19/2011

\begin{theorem} \textbf{(Fundamental Theorem of Arithmetic)}\index{Fundamental Theorem!of Arithmetic} \label{integers_theorem_FTA}
Let $n$ be an integer such that $n > 1$.  Then
\[
n = p_1 p_2 \cdots p_k,
\]
where $p_1, \ldots, p_k$ are  primes (not necessarily distinct).  Furthermore, this factorization is unique; that is, if 
\[
n = q_1 q_2 \cdots q_l,
\]
then $k = l$ and the $q_i$'s are just the $p_i$'s rearranged.
\end{theorem}

\begin{proof}
\emph{Uniqueness}.
To show uniqueness we will use induction on $n$. The theorem is certainly true for $n = 2$ since in this case $n$ is prime.  Now assume that the result holds for all integers $m$ such that $1 \leq m < n$, and 
\[
n = p_1 p_2 \cdots p_k = q_1 q_2 \cdots q_l,
\]
where $p_1 \leq p_2 \leq \cdots \leq p_k$ and $q_1 \leq q_2 \leq \cdots \leq q_l$. By Lemma~\ref{integers:prime_divisor_theorem}, $p_1  \mid  q_i$ for some $i = 1, \ldots, l$ and $q_1  \mid  p_j$ for some $j = 1, \ldots, k$.  Since all of the $p_i$'s and $q_i$'s are prime, $p_1 = q_i$ and  $q_1 = p_j$. Hence, $p_1 = q_1$ since $p_1 \leq p_j = q_1 \leq q_i = p_1$.  By the induction hypothesis, 
\[
n' = p_2 \cdots p_k = q_2 \cdots q_l
\]
has a unique factorization.  Hence, $k=l$ and $q_i = p_i$ for $i = 1, \ldots, k$. 

\emph{Existence}.
To show existence, suppose that there is some integer that cannot be written as the product of primes.  Let $S$ be the set of all such numbers.  By the Principle of Well-Ordering, $S$ has a smallest number, say $a$.  If the only positive factors of $a$ are $a$ and 1, then $a$ is prime, which is a contradiction.  Hence, $a = a_1 a_2$ where $1 < a_1 < a$ and $1 < a_2 < a$.  Neither $a_1\in S$ nor $a_2 \in S$, since $a$ is the smallest element in $S$.  So 
\begin{align*}
a_1 & = p_1 \cdots p_r \\
a_2 & = q_1 \cdots q_s.
\end{align*}
Therefore,
\[
a = a_1 a_2 = p_1 \cdots p_r q_1 \cdots q_s.
\]
So $a \notin S$, which is a contradiction.
\end{proof}
 
\histhead

\noindent
\parbox{\textwidth}
{\small \histf
Prime numbers were first studied by the ancient Greeks.  Two important results from antiquity are Euclid's proof that an infinite number of primes exist and the Sieve of Eratosthenes, a method of computing all of the prime numbers less than a fixed positive integer $n$.  One problem in number theory is to find a function $f$ such that $f(n)$ is prime for each integer $n$. Pierre Fermat (1601?--1665) conjectured that $2^{2^n} + 1$ was prime for all $n$, but later it was shown by Leonhard Euler (1707--1783) that 
\[
2^{2^5} + 1 = \text{4,294,967,297}
\]
is a composite number. One of the many unproven conjectures about prime numbers is Goldbach's Conjecture.  In a letter to Euler in 1742, Christian Goldbach stated the conjecture that every even integer with the exception of 2 seemed to be the sum of two primes:  $4 = 2 + 2$, $6 = 3 + 3$, $8 =3 + 5$, $\ldots$.  Although the conjecture has been verified for the numbers up through 100 million, it has yet to be proven in general.  Since prime numbers play an important role in public key cryptography, there is currently a great deal of interest in determining whether or not a large number is prime.
\histbox
}


\markright{EXERCISES}
\section*{Exercises}
\exrule

{\small
 
\begin{enumerate}
 
\item
Prove that
\[
1^2 + 2^2 + \cdots + n^2 = \frac{n(n + 1)(2n + 1)}{6}
\]
for $n \in {\mathbb N}$.

\item
Prove that
\[
1^3 + 2^3 + \cdots + n^3 = \frac{n^2(n + 1)^2}{4}
\]
for $n \in {\mathbb N}$.

\item
Prove that $n! > 2^n$ for $n \geq 4$.

\item
Prove that
\[
x + 4x + 7x + \cdots + (3n-2)x = \frac{n(3n - 1)x}{2}
\]
for $n \in {\mathbb N}$.

\item
Prove that $10^{n + 1} + 10^n + 1$ is divisible by 3 for $n \in {\mathbb N}$.

\item
Prove that $4 \cdot 10^{2n} + 9 \cdot 10^{2n - 1} + 5$ is divisible by 99 for $n \in {\mathbb N}$.

\item
Show that
\[
\sqrt[n]{a_1 a_2 \cdots a_n} \leq \frac{1}{n} \sum_{k = 1}^{n} a_k.
\]

\item
Prove the Leibniz rule for $f^{(n)} (x)$, where $f^{(n)}$ is the $n$th derivative of $f$; that is, show that 
\[
(fg)^{(n)} (x) = \sum_{k=0}^{n} \binom{n}{k} f^{(k)}(x) g^{(n-k)} (x).
\]

\item
Use induction to prove that $1 + 2 + 2^2 + \cdots + 2^n = 2^{n + 1} - 1$ for $n \in {\mathbb N}$. 

\item 
Prove that
\[
\frac{1}{2}+ \frac{1}{6} + \cdots + \frac{1}{n(n + 1)} = \frac{n}{n + 1} 
\]
for $n \in {\mathbb N}$.

\item 
If $x$ is a nonnegative real number, then show that $(1 + x)^n - 1 \geq nx$ for $n = 0, 1, 2, \ldots$. 
 
\item
\textbf{Power Sets.} 
Let $X$ be a set.  Define the \boldemph{power set}\index{Power set} of $X$, denoted ${\mathcal P}(X)$\label{powerset}, to be the set of all subsets  of $X$.  For example,  
\[
{\mathcal P}( \{a, b\} ) = \{ \emptyset, \{a\}, \{b\}, \{a, b\} \}.
\]
For every positive integer $n$, show that a set with exactly $n$ elements has a power set with exactly $2^n$ elements.

\item
Prove that the two principles of mathematical induction stated in Section~\ref{integers:math_induction} are equivalent. 

\item
Show that the Principle of Well-Ordering for the natural numbers implies that 1 is the smallest natural number.  Use this result to show that the Principle of Well-Ordering implies the Principle of Mathematical Induction; that is, show that if $S \subset {\mathbb N}$ such that $1 \in S$ and $n + 1 \in S$ whenever $n \in S$, then $S = {\mathbb N}$.  

\item
For each of the following pairs of numbers $a$ and $b$, calculate $\gcd(a,b)$ and find integers $r$ and $s$ such that  $\gcd(a,b) = ra + sb$. 
\begin{multicols}{2}
\begin{enumerate}

\item 
14 and 39

\item
234 and 165

\item
1739 and 9923

\item
471 and 562

\item
23,771 and 19,945

\item
$-4357$ and 3754

\end{enumerate}
\end{multicols}
 
\item 
Let $a$ and $b$ be nonzero integers. If there exist integers $r$ and $s$ such that $ar +bs =1$, show that $a$ and $b$ are relatively prime. 
 
 
\item
\textbf{Fibonacci Numbers.}
The Fibonacci numbers are
\[
1, 1, 2, 3, 5, 8, 13, 21, \ldots.
\]
We can define them inductively by $f_1 = 1$, $f_2 = 1$, and $f_{n + 2} = f_{n + 1} + f_n$ for $n \in {\mathbb N}$. 
\begin{enumerate}
 
 \item
Prove that $f_n < 2^n$.
 
 \item
Prove that $f_{n + 1} f_{n - 1} = f^2_n + (-1)^n$, $n \geq 2$.
 
 \item
Prove that $f_n = [(1 + \sqrt{5}\, )^n - (1 - \sqrt{5}\, )^n]/ 2^n \sqrt{5}$.
 
 \item
Show that $\lim_{n \rightarrow \infty} f_n / f_{n + 1} = (\sqrt{5} - 1)/2$. 
 
 \item
Prove that $f_n$ and $f_{n + 1}$ are relatively prime.
 
\end{enumerate}

\item
Let $a$ and $b$ be integers such that $\gcd(a,b) = 1$.  Let $r$ and $s$ be integers such that $ar + bs =1$.  Prove that 
\[
\gcd(a,s) = \gcd(r,b) = \gcd(r,s) =  1.
\]

\item
Let $x, y \in {\mathbb N}$ be relatively prime.  If $xy$ is a perfect square, prove that $x$ and $y$ must both be perfect squares.

\item
Using the division algorithm, show that every perfect square is of the form $4k$ or $4k + 1$ for some nonnegative integer $k$.

\item
Suppose that $a, b, r, s$ are pairwise relatively prime and that
\begin{align*}
a^2 + b^2 & = r^2 \\
a^2 - b^2 & = s^2.
\end{align*}
Prove that $a$, $r$, and $s$ are odd and $b$ is even.

%% TWJ 9/15/2011
%% Changed "coprime" to "pairwise relatively prime"
%% Suggested by R. Beezer
 
\item
Let $n \in {\mathbb N}$.  Use the division algorithm to prove that every integer is congruent mod $n$ to precisely one of the integers $0, 1, \ldots, n-1$.  Conclude that if $r$ is an integer, then there is exactly one $s$ in ${\mathbb Z}$ such that $0 \leq s < n$ and $[r] = [s]$.   Hence, the integers are indeed partitioned by congruence mod $n$. 

\item
Define the \boldemph{least common multiple} of two nonzero integers $a$ and $b$, denoted by $\lcm(a,b)$\label{leastcm}, to be the nonnegative integer $m$ such that both $a$ and $b$ divide $m$, and if $a$ and $b$  divide any other integer $n$, then $m$ also divides $n$.  Prove that any two integers $a$ and $b$ have a unique least common multiple. 

\item
If $d= \gcd(a, b)$ and $m = \lcm(a, b)$, prove that $dm = |ab|$.

\item
Show that $\lcm(a,b) = ab$ if and only if $\gcd(a,b) = 1$.

\item
Prove that $\gcd(a,c) = \gcd(b,c) =1$ if and only if $\gcd(ab,c) = 1$ for integers $a$, $b$, and $c$.

\item
Let $a, b, c \in {\mathbb Z}$.  Prove that if $\gcd(a,b) = 1$ and $a  \mid bc$, then $a  \mid  c$. 
 
\item
Let $p \geq 2$.  Prove that if $2^p-1$ is prime, then $p$ must also be prime.

\item
Prove that there are an infinite number of primes of the form $6n + 1$. 

\item
Prove that there are an infinite number of primes of the form $4n - 1$.

\item
Using the fact that 2 is prime, show that there do not exist integers $p$ and $q$ such that $p^2 = 2 q^2$.  Demonstrate that therefore $\sqrt{2}$ cannot be a rational number.  

\end{enumerate}
}
 
 
\subsection*{Programming Exercises}
 
{\small
\begin{enumerate}
 
\item
\textbf{The Sieve of Eratosthenes.}\index{Sieve of Eratosthenes}  
One method of computing all of the prime numbers less than a certain fixed positive integer $N$ is to list all of the numbers $n$ such that $1 < n < N$.  Begin by eliminating all of the multiples of 2.  Next eliminate all of the multiples of 3. Now eliminate all of the  multiples of 5.  Notice that 4 has already been crossed out.  Continue in this manner, noticing that we do not have to go all the way to $N$; it suffices to stop at $\sqrt{N}$.  Using this method, compute all of the prime numbers less than $N = 250$.  We can also use this method to find all of the integers that are relatively prime to an integer $N$.  Simply eliminate the prime factors of $N$ and all of their multiples.  Using this method, find all of the numbers that are relatively prime to $N= 120$.  Using the Sieve of Eratosthenes, write a program that will compute all of the primes less than an integer $N$. 

\item
Let ${\mathbb N}^0 = {\mathbb N} \cup \{ 0 \}$. Ackermann's function\index{Ackermann's function} is the function $A :{\mathbb N}^0 \times {\mathbb N}^0 \rightarrow {\mathbb N}^0$ defined by the equations 
\begin{align*}
A(0, y) & = y + 1, \\
A(x + 1, 0) & = A(x, 1), \\
A(x + 1, y + 1) & = A(x, A(x + 1, y)).
\end{align*}
Use this definition to compute $A(3, 1)$.  Write a program to evaluate Ackermann's function.  Modify the  program to count the number of statements executed in the program when Ackermann's function is evaluated.  How many statements are executed in the evaluation of $A(4, 1)$?  What about $A(5, 1)$?

\item
Write a computer program that will implement the Euclidean algorithm.  The program should accept two positive integers $a$ and $b$ as input and should output $\gcd( a,b)$ as well as integers $r$ and $s$ such that 
\[
\gcd( a,b) = ra + sb.
\]
 
\end{enumerate}
}
 
 
\subsection*{References and Suggested Readings} %rReferences updated - TWJ 5/4/2010

{\small
References [2], [3], and [4] are good sources for elementary number
theory. 
\begin{itemize}
 
\item[\textbf{[1]}]
Brookshear, J. G. \textit{Theory of Computation: Formal Languages,
Automata, and Complexity}.  Benjamin/Cummings, Redwood City, CA, 1989.
Shows the relationships of the theoretical aspects of computer science
to set theory and the integers.
 
\item[\textbf{[2]}] %Updated - TWJ 5/4/2010
Hardy, G. H. and Wright, E. M. \textit{An Introduction to the Theory of
Numbers}.  6th ed. Oxford University Press, New York, 2008. 
 
 
\item[\textbf{[3]}] %Checked reference - TWJ 5/4/2010
Niven, I. and Zuckerman, H. S. \textit{An Introduction to the Theory of
Numbers}.  5th ed. Wiley, New York, 1991. 
 
\item[\textbf{[4]}] %Updated - TWJ 5/4/2010
Vanden Eynden, C. \textit{Elementary Number Theory}. 2nd ed.  Waveland Press, Long Grove IL, 2001. 
\end{itemize}
 
}
 
\sagesection