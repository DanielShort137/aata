%%%%(c)
%%%%(c)  This file is a portion of the source for the textbook
%%%%(c)
%%%%(c)    Abstract Algebra: Theory and Applications
%%%%(c)    Copyright 1997 by Thomas W. Judson
%%%%(c)
%%%%(c)  See the file COPYING.txt for copying conditions
%%%%(c)
%%%%(c)
\chap{Normal Subgroups and Factor Groups}{normal}

%% TWJ, 2010/03/31
%% The chapter HOMOMORPHISMS AND FACTOR GROUPS is now
%% two chapters: (10) NORMAL SUBGROUPS AND FACTOR GROUPS
%% (11) HOMOMORPHISMS

If $H$ is a subgroup of a group $G$, then right cosets are not always the same as left cosets; that is, it is not always the case that $gH = Hg$ for all $g \in G$.  The subgroups for which this property holds play a critical role in group theory: they allow for the construction of a new class of groups, called factor or quotient groups.  Factor groups may be studied by using homomorphisms, a generalization of isomorphisms. 
 

\section{Factor Groups and Normal Subgroups}
 
\subsection*{Normal Subgroups}

A subgroup $H$ of a group $G$ is {\bfi
normal\/}\index{Subgroup!normal}\index{Normal subgroup} in G if $gH =
Hg$ for all $g \in G$. That is, a normal subgroup of a group $G$ is
one in which the right and left cosets are precisely the same. 
 
\begin{example}{normal_abelian}
Let $G$ be an abelian group. Every subgroup $H$ of $G$ is a normal
subgroup.  Since $gh = hg$ for all $g \in G$ and $h \in H$, it will
always be the case that $gH = Hg$. 
\end{example}
 
\begin{example}{normal_S3}
Let $H$ be the subgroup of $S_3$ consisting of elements $(1)$ and
$(12)$. Since 
\[
(123) H = \{ (123), (13) \}
\quad
\text{and}
\quad
H (123) = \{ (123), (23) \},
\]
$H$ cannot be a normal subgroup of $S_3$.  However, the subgroup $N$,
consisting of the permutations $(1)$, $(123)$, and $(132)$, is normal
since the cosets of $N$ are 
\[
\begin{array}{c}
N  =   \{ (1), (123), (132) \} \\
(12) N =  N (12)  =  \{ (12), (13), (23) \}.
\end{array}
\]
\end{example}
 
 
The following theorem is fundamental to our understanding of normal
subgroups.
 
 
\begin{theorem}
Let $G$ be a group and $N$ be a subgroup of $G$. Then the following
statements are equivalent.
\begin{enumerate}
 
\rm \item \it
The subgroup $N$ is normal in $G$. 
 
\rm \item \it
For all $g \in G$, $gNg^{-1} \subset N$. 
 
\rm \item \it
For all $g \in G$, $gNg^{-1} = N$.
 
\end{enumerate}
\end{theorem}
 
 
\begin{proof}
(1) $\Rightarrow$ (2).
Since $N$ is normal in $G$, $gN = Ng$ for all $g \in G$. Hence, for a
given $g \in G$ and $n \in N$, there exists an $n'$ in $N$ such that
$g n = n' g$. Therefore, $gng^{-1} = n' \in N$ or $gNg^{-1} \subset
N$.
 
 
(2)  $\Rightarrow$ (3).  
Let $g \in G$. Since $gNg^{-1} \subset N$, we need only show $N
\subset gNg^{-1}$. For $n \in N$,  $g^{-1}ng=g^{-1}n(g^{-1})^{-1} \in
N$.  Hence, $g^{-1}ng = n'$ for some $n' \in N$. Therefore, $n = g n'
g^{-1}$ is in $g N g^{-1}$.
 
 
(3) $\Rightarrow$ (1).
Suppose that $gNg^{-1} = N$ for all $g \in G$. Then for any $n \in N$
there exists an $n' \in N$ such that $gng^{-1} = n'$.  Consequently,
$gn = n' g$ or $gN \subset Ng$. Similarly, $Ng \subset gN$.
\end{proof}
 
 
\subsection*{Factor Groups}
 
 
If $N$ is a normal subgroup of a group $G$, then the cosets of $N$ in
$G$ form a group $G/N$\label{notefactor} under the operation $(aN)
(bN) = abN$. This group is called the {\bfi
factor\/}\index{Group!factor} or {\bfi quotient
group\/}\index{Group!quotient} of $G$ and $N$.  Our first task is to
prove that $G/N$ is indeed a group.  
 
 
\begin{theorem}
Let $N$ be a normal subgroup of a group $G$. The cosets of $N$ in $G$
form a group $G/N$ of order $[G:N]$. 
\end{theorem}
 
 
\begin{proof}
The group operation on $G/N$ is $(a N ) (b N)= a b N$.  This operation
must be shown to be well-defined; that is, group multiplication must
be independent of the choice of  coset representative. Let $aN = bN$
and $cN = dN$. We must show that
\[
(aN) (cN) = acN = bd N = (b N)(d N).
\]
Then $a = b n_1$ and $c = d n_2$ for some $n_1$ and $n_2$ in
$N$. Hence, 
\begin{align*}
acN & = b n_1 d n_2 N \\
& = b n_1 d N \\
& = b n_1 N d \\
& = b N d \\
& = b d N.
\end{align*}
The remainder of the theorem is easy: $eN = N$ is the identity and
$g^{-1} N$ is the inverse of $gN$. The order of $G/N$ is, of course,
the number of cosets of $N$ in $G$. 
\end{proof}
 
 
\medskip
 
 
It is very important to remember that the elements in a factor group are
{\em sets of elements\/} in the original group. 
 
 

 
 
\begin{example}{factor_S3}
Consider the normal subgroup of $S_3$, $N = \{ (1), (123), (132)  \}$.
The cosets of $N$ in $S_3$ are $N$ and $(12) N$. The factor group $S_3
/ N$ has the following multiplication table.
\begin{center}
\begin{tabular}{c|cc}
         & $N$      & $(12) N$ \\
\hline
$N$      & $N$      & $(12) N$ \\
$(12) N$ & $(12) N$ & $N$
\end{tabular}
\end{center}
This group is isomorphic to ${\Bbb Z}_2$. At first, multiplying cosets
seems both complicated and strange; however, notice that  $S_3 / N$ is
a smaller group. The factor group displays a certain amount of
information about $S_3$.  Actually, $N = A_3$, the group of even
permutations, and $(12) N = \{ (12), (13), (23) \}$ is the set of odd
permutations. The information captured in $G/N$ is parity; that is,
multiplying two even or two odd permutations results in an even
permutation, whereas multiplying an odd permutation by an even
permutation yields an odd permutation.
\end{example}
 
 
\begin{example}{factor_Z3}
Consider the normal subgroup $3 {\Bbb Z}$ of ${\Bbb Z}$. The cosets of
$3 {\Bbb Z}$ in ${\Bbb Z}$ are 
\begin{align*}
0 + 3 {\Bbb Z} & = \{ \ldots, -3, 0, 3, 6, \ldots \} \\
1 + 3 {\Bbb Z} & = \{ \ldots, -2, 1, 4, 7, \ldots \} \\
2 + 3 {\Bbb Z} & = \{ \ldots, -1, 2, 5, 8, \ldots \}.
\end{align*}
The group   ${\Bbb Z}/ 3 {\Bbb Z}$ is given by the multiplication
table below. 
\begin{center}
\begin{tabular}{c|ccc}
$+$             & $0 + 3{\Bbb Z}$ & $1 + 3{\Bbb Z}$ & $2 + 3{\Bbb Z}$ \\\hline
$0 + 3{\Bbb Z}$ & $0 + 3{\Bbb Z}$ & $1 + 3{\Bbb Z}$ & $2 + 3{\Bbb Z}$ \\
$1 + 3{\Bbb Z}$ & $1 + 3{\Bbb Z}$ & $2 + 3{\Bbb Z}$ & $0 + 3{\Bbb Z}$ \\
$2 + 3{\Bbb Z}$ & $2 + 3{\Bbb Z}$ & $0 + 3{\Bbb Z}$ & $1 + 3{\Bbb Z}$
\end{tabular}
\end{center}
In general, the subgroup $n {\Bbb Z}$ of ${\Bbb Z}$ is normal. The
cosets of ${\Bbb Z } / n {\Bbb Z}$ are 
\[
\begin{array}{c}
n {\Bbb Z} \\
1 + n {\Bbb Z} \\
2 + n {\Bbb Z} \\
\vdots \\
(n-1) + n {\Bbb Z}.
\end{array}
\]
The sum of the cosets $k + {\Bbb Z}$ and $l + {\Bbb Z}$ is $k+l + 
{\Bbb Z}$. Notice that  we have written our cosets additively, 
because the group operation is integer addition. 
\mbox{\hspace*{1in}}
\end{example}
 
 
\begin{example}{factor_Dn}
Consider the dihedral group $D_n$, generated by the two elements $r$
and $s$, satisfying the relations 
\begin{align*}
r^n & = id \\
s^2 & = id \\
srs & = r^{-1}.
\end{align*}
The element $r$ actually generates the cyclic subgroup of rotations,
$R_n$, of $D_n$.  Since $srs^{-1} = srs = r^{-1} \in R_n$, the group
of rotations is a normal subgroup of $D_n$; therefore, $D_n / R_n$ is
a group.  Since there are exactly two elements in this group, it must
be isomorphic to ${\Bbb Z}_2$.
\end{example}
 
 

 
 
\section{Simplicity of $A_n$}
 
 
Of special interest are groups with no nontrivial normal subgroups.
Such groups are called {\bfi simple
groups}\index{Group!simple}\index{Simple group}.  Of course, we
already have a whole class of examples of simple groups, ${\Bbb Z}_p$,
where $p$ is prime.  These groups are trivially simple since they have
no proper subgroups other than the subgroup consisting solely of the
identity. Other examples of simple groups are not so easily found.
We can, however, show that the alternating group, $A_n$, is simple for
$n \geq 5$. The proof of this result requires several lemmas. 
 
 
\begin{lemma}
The alternating group $A_n$ is generated by $3$-cycles for $n \geq 3$.
\end{lemma}
 
\begin{proof}
To show that the 3-cycles generate $A_n$, we need only show that any
pair of transpositions can be written as the product of 3-cycles.
Since $(a b) = (b a)$, every pair of transpositions must be one of the
following: 
\begin{align*}
(ab)(ab) & = id \\
(ab)(cd) & = (acb)(acd) \\
(ab)(ac) & = (acb).
\end{align*}
\end{proof}
 
 
\begin{lemma}
Let $N$ be a  normal subgroup of $A_n$, where $n \geq 3$. If $N$ 
contains a $3$-cycle, then $N = A_n$. 
\end{lemma}
 
 
\begin{proof}
We will first show that $A_n$ is generated by 3-cycles of the specific
form $(ijk)$, where $i$ and $j$ are fixed in  $\{ 1, 2, \ldots, n \}$
and we let $k$ vary. Every 3-cycle is the product of 3-cycles of this 
form, since
\begin{align*}
(i a j) & = (i j a)^2  \\
(i a b) & = (i j b) (i j a)^2 \\
(j a b) & = (i j b)^2 (i j a) \\
(a b c) & = (i j a)^2 (i j c) (i j b)^2 (i j a).
\end{align*}
Now suppose that $N$ is a nontrivial normal subgroup of $A_n$ for $n 
\geq 3$  such that $N$ contains a 3-cycle of the form $(i j a)$. Using
the normality of $N$, we see that
\[
[(i j)(a k)](i j a)^2 [(i j)(a k)]^{-1} = (i j k)
\]
is in $N$. Hence, $N$ must contain all of the 3-cycles $(i j k)$ 
for $1 \leq k \leq n$. By Lemma~9.5, these 3-cycles generate $A_n$; 
hence, $N = A_n$. 
\end{proof}
 
 
\begin{lemma}
For $n \geq 5$, every normal subgroup $N$ of $A_n$ contains a
$3$-cycle. 
\end{lemma}
 
 
\begin{proof}
Let $\sigma$ be an arbitrary element in a normal subgroup $N$. There
are several possible cycle structures for $\sigma$.
\begin{itemize}
 
\item
$\sigma$ is a 3-cycle.
 
\item
$\sigma$ is the product of disjoint cycles, $\sigma = \tau(a_1 a_2
\cdots a_r) \in N$, where $r >3$.
 
 
\item
$\sigma$ is the product of disjoint cycles, $\sigma = \tau(a_1 a_2
a_3)(a_4 a_5 a_6)$.
 
 
\item
$\sigma = \tau(a_1 a_2 a_3)$, where $\tau$ is the product of disjoint
2-cycles. 
 
 
\item
$\sigma =
\tau (a_1 a_2) (a_3 a_4) $, where $\tau$ is the product of an even
number of disjoint 2-cycles. 
 
 
\end{itemize}
If $\sigma$ is a $3$-cycle, then we are done. If $N$ contains a
product of disjoint cycles, $\sigma$, and at least one of these cycles
has length greater than 3, say $\sigma = \tau(a_1 a_2 \cdots a_r)$,
then   
\[
(a_1 a_2 a_3)\sigma(a_1 a_2 a_3)^{-1}
\]
is in $N$ since $N$ is normal; hence,
\[
\sigma^{-1}(a_1 a_2 a_3)\sigma(a_1 a_2 a_3)^{-1}
\]
is also in $N$. Since
\begin{align*}
\lefteqn{\sigma^{-1}(a_1 a_2 a_3)\sigma(a_1 a_2 a_3)^{-1} } \\
& = \sigma^{-1}(a_1 a_2 a_3)\sigma(a_1 a_3 a_2) \\
& = (a_1 a_2 \cdots a_r)^{-1}\tau^{-1}(a_1 a_2 a_3) 
      \tau(a_1 a_2 \cdots a_r)(a_1 a_3 a_2) \\
& = (a_1 a_r a_{r-1} \cdots a_2 )(a_1 a_2 a_3) 
      (a_1 a_2 \cdots a_r)(a_1 a_3 a_2) \\
& = (a_1 a_3 a_r),
\end{align*}
$N$ must contain a 3-cycle; hence, $N = A_n$.
 
 
 
 
Now suppose that $N$ contains a disjoint product of the form
\[
\sigma = \tau(a_1 a_2 a_3)(a_4 a_5 a_6).
\]
Then
\[
\sigma^{-1}(a_1 a_2 a_4)\sigma(a_1 a_2 a_4)^{-1} \in N
\]
since
\[
(a_1 a_2 a_4)\sigma(a_1 a_2 a_4)^{-1} \in N.
\]
So
\begin{align*}
\lefteqn{\sigma^{-1}(a_1 a_2 a_4)\sigma(a_1 a_2 a_4)^{-1} } \\
& = [ \tau (a_1 a_2 a_3) (a_4 a_5 a_6) ]^{-1}  (a_1 a_2 a_4) 
      \tau (a_1 a_2 a_3) (a_4 a_5 a_6) (a_1 a_2 a_4)^{-1} \\
& = (a_4 a_6 a_5) (a_1 a_3 a_2) \tau^{-1}(a_1 a_2 a_4)  
      \tau (a_1 a_2 a_3) (a_4 a_5 a_6) (a_1 a_4 a_2) \\
& = (a_4 a_6 a_5)(a_1 a_3 a_2) (a_1 a_2 a_4)
      (a_1 a_2 a_3) (a_4 a_5 a_6)(a_1 a_4 a_2) \\
& = (a_1 a_4 a_2 a_6 a_3).
\end{align*}
So $N$ contains a disjoint cycle of length greater than 3, and we can
apply the previous case. 
 
 
Suppose $N$ contains a disjoint product of the form $\sigma = \tau(a_1
a_2 a_3)$, where $\tau$ is the product of disjoint 2-cycles. Since
$\sigma \in N$, $\sigma^2 \in N$, and
\begin{align*}
\sigma^2
& = \tau(a_1 a_2 a_3)\tau(a_1 a_2 a_3) \\
& =(a_1 a_3 a_2).
\end{align*}
So $N$ contains a 3-cycle.
 
 
The only remaining possible case is a disjoint product of the form
\[
\sigma = \tau (a_1 a_2) (a_3 a_4),
\]
where $\tau$ is the product of an even number of disjoint 2-cycles.
But 
\[
\sigma^{-1}(a_1 a_2 a_3)\sigma(a_1 a_2 a_3)^{-1}
\]
is in $N$ since $(a_1 a_2 a_3)\sigma(a_1 a_2 a_3)^{-1}$ is in $N$; and
so 
\begin{align*}
\lefteqn{\sigma^{-1}(a_1 a_2 a_3)\sigma(a_1 a_2 a_3)^{-1} } \\
& = \tau^{-1} (a_1 a_2) (a_3 a_4) (a_1 a_2 a_3) 
      \tau (a_1 a_2)(a_3 a_4)(a_1 a_2 a_3)^{-1} \\
& = (a_1 a_3)(a_2 a_4).
\end{align*}
Since $n \geq 5$, we can find $b \in \{1, 2, \ldots, n \}$ such that
$b \neq a_1, a_2, a_3, a_4$. Let $\mu = (a_1 a_3 b)$. Then
\[
\mu^{-1} (a_1 a_3)(a_2 a_4) \mu (a_1 a_3)(a_2 a_4) \in N
\]
and
\begin{align*}
\lefteqn{\mu^{-1} (a_1 a_3)(a_2 a_4) \mu (a_1 a_3)(a_2 a_4) } \\
& = (a_1 b a_3)(a_1 a_3)(a_2 a_4) 
      (a_1 a_3 b)(a_1 a_3)(a_2 a_4) \\
& = (a_1 a_3 b ).
\end{align*}
Therefore, $N$ contains a 3-cycle. This completes the proof of the
lemma.  
\end{proof}
 
 
\begin{theorem}
The alternating group, $A_n$, is simple for $n \geq 5$. 
\end{theorem}
 
\begin{proof}
Let $N$ be a normal subgroup of $A_n$. By Lemma~9.7, $N$ contains a
3-cycle. By Lemma~9.6, $N = A_n$; therefore, $A_n$ contains no proper
nontrivial normal subgroups for $n \geq 5$.
\end{proof} 
 
 
\histhead
 
 
\noindent{\small \histf
One of the foremost problems of group theory has been to classify all
simple finite groups\index{Group!simple}. This problem is over a
century old and has been solved only in the last few years. In a
sense, finite simple groups are the building blocks of all finite
groups.  The first nonabelian simple groups to be discovered were the
alternating groups.  Galois was the first to prove that $A_5$ was
simple. Later mathematicians, such as C.~Jordan\index{Jordan, C.} and
L.~E.~Dickson,\index{Dickson, L. E.} found several infinite families of
matrix groups that were simple. Other families of simple groups were
discovered in the 1950s.  At the turn of the century, William
Burnside\index{Burnside, William} conjectured that all nonabelian
simple groups must have even order. In 1963, W. Feit\index{Feit, W.}
and J. Thompson\index{Thompson, J.} proved Burnside's conjecture and
published their results in the paper ``Solvability of Groups of Odd
Order,'' which appeared in the {\it Pacific Journal of Mathematics}.
Their proof, running over 250 pages, gave impetus to a program in the
1960s and 1970s to classify all finite simple groups.  Daniel
Gorenstein\index{Gorenstein, Daniel} was the organizer of this
remarkable effort. One of the last simple groups was the ``Monster,''
discovered by R.~Greiss\index{Greiss, R.}. The Monster, a $\mbox{196,833}
\times \mbox{196,833}$ matrix group, is one of the 26 sporadic, or
special, simple groups. These sporadic simple groups are groups that
fit into no infinite family of simple groups. 
\histbox 
}
 

 
 
\markright{EXERCISES}
\section*{Exercises}
\exrule
 
 
 
{\small
 
 
\begin{enumerate}
 
 
\bf\item\rm
For each of the following groups $G$, determine whether $H$ is a normal
subgroup of $G$. If $H$ is a normal subgroup, write out a Cayley table
for the factor group $G/H$.
\begin{enumerate}
 
 \bf\item\rm
$G = S_4$ and $H = A_4$

 \bf\item\rm
$G = A_5$ and $H = \{ (1), (123), (132) \}$
 
 \bf\item\rm
$G = S_4$ and $H = D_4$
 
 \bf\item\rm
$G = Q_8$ and $H = \{ 1, -1, i, -i \}$
 
 \bf\item\rm
$G = {\Bbb Z}$ and $H = 5 {\Bbb Z}$
 
\end{enumerate}
 
 
\bf\item\rm
Find all the subgroups of $D_4$. Which subgroups are normal? What are
all the factor groups of $D_4$ up to isomorphism?
 
 
\bf\item\rm
Find all the subgroups of the quaternion group, $Q_8$. Which subgroups
are normal? What are all the factor groups of $Q_8$ up to isomorphism?
 
 
%\bf\item\rm
%Prove that $\det( AB) = \det(A) \det(B)$ for $A, B \in GL_2( {\Bbb R}
%)$. This shows that the determinant is a homomorphism from $GL_2(
%{\Bbb R} )$ to ${\Bbb R}^*$. 
% 
 
 
%\bf\item\rm
%Which of the following maps are homomorphisms? If the map is a
%homomorphism, what is the kernel? 
%\begin{enumerate}
% 
% \bf\item\rm
%$\phi : {\Bbb R}^\ast \rightarrow GL_2 ( {\Bbb R})$ defined by
%\]
%\phi( a ) =
%\left(
%\begin{array}{cc}
%1 & 0 \\
%0 & a
%\end{array}
%\right)
%\]
% 
% \bf\item\rm
%$\phi : {\Bbb R} \rightarrow GL_2 ( {\Bbb R})$ defined by
%\]
%\phi( a ) =
%\left(
%\begin{array}{cc}
%1 & 0 \\
%a & 1
%\end{array}
%\right)
%\]
% 
% \bf\item\rm
%$\phi : GL_2 ({\Bbb R})   \rightarrow {\Bbb R}$ defined by
%\]
%\phi
%\left(
%\left(
%\begin{array}{cc}
%a & b \\
%c & d
%\end{array}
%\right)
%\right)
%= a + d
%\]
% 
% \bf\item\rm
%$\phi : GL_2 ( {\Bbb R})   \rightarrow {\Bbb R}^\ast$ defined by 
%\]
%\phi
%\left(
%\left(
%\begin{array}{cc}
%a & b \\
%c & d
%\end{array}
%\right)
%\right)
%= ad -bc
%\]
% 
% \bf\item\rm
%$\phi : {\Bbb M}_2( {\Bbb R})   \rightarrow {\Bbb R}$ defined by
%\]
%\phi
%\left(
%\left(
%\begin{array}{cc}
%a & b \\
%c & d
%\end{array}
%\right)
%\right)
%= b,
%\]
%where ${\Bbb M}_2( {\Bbb R})$ is the additive group of $2 \times
%2$ matrices with entries in ${\Bbb R}$.
% 
%\end{enumerate}
 
 
\bf\item\rm
Let $T$ be the group of nonsingular upper triangular $2 \times 2$
matrices with entries in ${\Bbb R}$; that is, matrices of the form
\[
\begin{pmatrix}
a & b \\
0 & c
\end{pmatrix},
\]
where $a$, $b$, $c \in {\Bbb R}$ and $ac \neq 0$. Let $U$ consist of
matrices of the form 
\[
\begin{pmatrix}
1 & x \\
0 & 1
\end{pmatrix},
\]
where $x \in {\Bbb R}$.
\begin{enumerate}
 
 \bf\item\rm 
Show that $U$ is a subgroup of $T$.
 
 \bf\item\rm 
Prove that $U$ is abelian.
 
 \bf\item\rm 
Prove that $U$ is normal in $T$.
 
 \bf\item\rm  
Show that $T/U$ is abelian.
 
 \bf\item\rm
Is $T$ normal in $GL_2( {\Bbb R})$?
 
\end{enumerate}
 
 
%\bf\item\rm
%Let $A$ be an $m \times n$ matrix.  Show that matrix multiplication,
%$x \mapsto Ax$, defines a homomorphism $\phi : {\Bbb R}^n \rightarrow
%{\Bbb R}^m$. 
% 
% 
%\bf\item\rm
%Let $\phi : {\Bbb Z} \rightarrow {\Bbb Z}$ be given by $\phi(n) = 7n$.
%Prove that $\phi$ is a group homomorphism. Find the kernel and the
%image of $\phi$.
% 
% 
%\bf\item\rm
%Describe all of the homomorphisms from ${\Bbb Z}_{24}$ to ${\Bbb
%Z}_{18}$. 
% 
% 
%\bf\item\rm
%Describe all of the homomorphisms from ${\Bbb Z}$ to ${\Bbb Z}_{12}$. 
 
 
%\bf\item\rm
%In the group ${\Bbb Z}_{24}$, let $H = \langle 4 \rangle$ and $N =
%\langle 6 \rangle$. 
%\begin{enumerate}
% 
% \bf\item\rm
%List the elements in $HN$ (we usually write $H + N$ for these additive
%groups) and $H \cap N$. 
% 
% \bf\item\rm
%List the cosets in $HN/N$, showing the elements in each coset.
% 
% \bf\item\rm
%List the cosets in $H/(H \cap N)$, showing the elements in each coset. 
% 
% \bf\item\rm
%Give the correspondence between $HN/N$ and $H/(H \cap N)$ described in
%the proof of the Second Isomorphism Theorem. 
% 
%\end{enumerate}
 
 
%***************************THEORY******************
 
 
%\bf\item\rm
%If $G$ is an abelian group and $n \in {\Bbb N}$, show that $\phi : G
%\rightarrow G$  defined by $g \mapsto g^n$ is a group homomorphism. 
 
 
\bf\item\rm
Show that the intersection of two normal subgroups is a normal
subgroup. 
 
 
%\bf\item\rm
%If $\phi : G \rightarrow H$ is a group homomorphism and $G$ is
%abelian, prove that $\phi(G)$ is also abelian. 
% 
% 
%\bf\item\rm
%If $\phi : G \rightarrow H$ is a group homomorphism and $G$ is cyclic,
%prove that $\phi(G)$ is also cyclic. 
 
 
%\bf\item\rm
%Show that a homomorphism defined on a cyclic group is completely
%determined by its action on the generator of the group.

\bf\item\rm
If $G$ is abelian, prove that $G/H$ must also be abelian.
 
\bf\item\rm
Prove or disprove: If $H$ is a normal subgroup of $G$ such that $H$
and $G/H$ are abelian, then $G$ is abelian. 
 
 

\bf\item\rm
If $G$ is cyclic, prove that $G/H$ must also be cyclic.


\bf\item\rm
Prove or disprove: If $H$ and $G/H$ are cyclic, then $G$ is cyclic.
 
 
\bf\item\rm
Let $H$ be a subgroup of index 2 of a group $G$. Prove that $H$ must
be a normal subgroup of $G$. Conclude that $S_n$ is not simple.
 
 
\bf\item\rm
Let $G$ be a group of order $p^2$, where $p$ is a prime number. If $H$
is a subgroup of $G$ of order $p$, show that $H$ is normal in $G$.
Prove that $G$ must be abelian. 
 
 
\bf\item\rm
If a group $G$ has exactly one subgroup $H$ of order $k$, prove that
$H$ is normal in $G$. 
 
 
%\bf\item\rm
%Prove or disprove: ${\Bbb Q} / {\Bbb Z} \cong {\Bbb Q}$.
% 
 
\bf\item\rm
Define the {\bfi centralizer\/}\index{Element!centralizer
of}\index{Centralizer!of an element} of an element $g$ in a group $G$
to be the set  
\[
C(g) = \{ x \in G : xg = gx \}.
\]
Show that $C(g)$ is a subgroup of $G$.  If $g$ generates a normal
subgroup of $G$, prove that $C(g)$ is normal in $G$.
 
 
\bf\item\rm
Recall that the {\bfi center\/}\index{Group!center of} of a group $G$ is
the set 
\[
Z(G) = \{ x \in G : xg = gx \mbox{ for all $g \in G$ } \}.
\]
\begin{enumerate}
 
 \bf\item\rm
Calculate the center of $S_3$.
 
 \bf\item\rm
Calculate the center of $GL_2 ( {\Bbb R} )$.
 
 \bf\item\rm
Show that the center of any group $G$ is a normal subgroup of $G$. 
 
 \bf\item\rm
If $G / Z(G)$ is cyclic, show that $G$ is abelian.
 
\end{enumerate}
 
 
%\bf\item\rm
%Let $G$ be a finite group and $N$ a normal subgroup of $G$. If $H$ is
%a subgroup of $G/N$, prove that $\phi^{-1}(H)$ is a subgroup in $G$ of
%order $|H| \cdot |N|$, where $\phi : G \rightarrow G/N$ is the
%canonical homomorphism. 
 
 
\bf\item\rm
Let $G$ be a group and let $G' = \langle aba^{- 1} b^{-1} \rangle$;
that is, $G'$ is the subgroup of all finite products of elements in
$G$ of the form $aba^{-1}b^{-1}$.  The subgroup $G'$ is called the
{\bfi commutator
subgroup\/}\index{Subgroup!commutator}\label{commutatorsubgroup} of $G$.  
\begin{enumerate}
 
 \bf\item\rm
Show that $G'$ is a normal subgroup of $G$.

 \bf\item\rm
Let $N$ be  a normal subgroup of $G$.  Prove that $G/N$ is abelian if
and only if $N$ contains the commutator subgroup of $G$.
 
\end{enumerate}

 
%\bf\item\rm
%Let $G_1$ and $G_2$ be groups, and let $H_1$ and $H_2$ be normal subgroups
%of $G_1$ and $G_2$ respectively. Let $\phi : G_1 \rightarrow G_2$ be a
%homomorphism. Show that $\phi$ induces a natural homomorphism
%$\overline{\phi} : (G_1/H_1) \rightarrow (G_2/H_2)$ if $\phi(H_1) \subseteq
%H_2$. 
% 
% 
%\bf\item\rm
%If $H$ and $K$ are normal subgroups of $G$ and $H \cap K = \{ e \}$,
%prove that $G$ is isomorphic to a subgroup of $G/H \times G/K$.
% 
% 
% 
%\bf\item\rm
%Let $\phi : G_1 \rightarrow G_2$ be a surjective group homomorphism.
%Let $H_1$ be a normal subgroup of $G_1$ and suppose that $\phi(H_1) =
%H_2$.  Prove or disprove that $G_1/H_1 \cong G_2/H_2$.
% 
% 
%\bf\item\rm
%Let $\phi : G \rightarrow H$ be a group homomorphism.  Show that
%$\phi$ is one-to-one if and only if $\phi^{-1}(e) = \{ e \}$.


\end{enumerate}
}
 


%\subsection*{Additional Exercises: Automorphisms}
% 
% 
%{\small
%\begin{enumerate}
% 
% 
%\bf\item\rm
%Let $Aut(G)$ be the set of all automorphisms of $G$; that is,
%isomorphisms from $G$ to itself. Prove this set forms a group and is a
%subgroup of the group of permutations of $G$; that is, $Aut(G) \leq S_G$. 
% 
% 
%\bf\item\rm
%An {\bfi inner automorphism\/}\index{Automorphism!inner} of $G$,
%\]
%i_g : G \rightarrow G,
%\]
%is defined by the map
%\]
%i_g(x) = g x g^{-1},
%\]
%for $g \in G$. Show that $i_g \in Aut(G)$.
% 
% 
%\bf\item\rm
%The set of all inner automorphisms is denoted by $Inn(G)$. Show that
%$Inn(G)$ is a subgroup of $Aut(G)$. 
% 
% 
%\bf\item\rm
%Find an automorphism of a group $G$ that is not an inner automorphism.
% 
% 
%\bf\item\rm
%Let $G$ be a group and $i_g$ be an inner automorphism of $G$, and
%define a map 
%\]
%G \rightarrow Aut(G)
%\]
%by
%\]
%g \mapsto i_g.
%\]
%Prove that this map is a homomorphism with image $Inn(G)$ and kernel
%$Z(G)$. Use this result to conclude that 
%\]
%G/Z(G) \cong Inn(G).
%\]
% 
% 
%\bf\item\rm
%Compute $Aut(S_3)$ and $Inn(S_3)$.  Do the same thing for $D_4$.
% 
% 
%\bf\item\rm
%Find all of the homomorphisms $\phi : {\Bbb Z} \rightarrow {\Bbb Z}$.
%What is $Aut({\Bbb Z})$? 
% 
% 
%\bf\item\rm
%Find all of the automorphisms of ${\Bbb Z}_8$.  Prove that $Aut({\Bbb
%Z}_8) \cong U(8)$. 
% 
% 
%\bf\item\rm
%For $k \in {\Bbb Z}_n$, define a map $\phi_k : {\Bbb Z}_n \rightarrow
%{\Bbb Z}_n$ by $a \mapsto ka$.  Prove that $\phi_k$ is a homomorphism. 
% 
% 
%\bf\item\rm
%Prove that $\phi_k$ is an isomorphism if and only if $k$ is a generator
%of ${\Bbb Z}_n$. 
% 
% 
%\bf\item\rm
%Show that every automorphism of ${\Bbb Z}_n$ is of the form $\phi_k$,
%where $k$ is a generator of ${\Bbb Z}_n$. 
% 
% 
%\bf\item\rm
%Prove that $\psi : U(n) \rightarrow Aut({\Bbb Z}_n)$ is an
%isomorphism, where $\psi : k \mapsto \phi_k$. 
% 
% 
%\end{enumerate}
%}
 
 
 
 
 
