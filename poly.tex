%%%%(c)
%%%%(c)  This file is a portion of the source for the textbook
%%%%(c)
%%%%(c)    Abstract Algebra: Theory and Applications
%%%%(c)    Copyright 1997 by Thomas W. Judson
%%%%(c)
%%%%(c)  See the file COPYING.txt for copying conditions
%%%%(c)
%%%%(c)
\chapter{Polynomials}\label{poly}
 
Most people are fairly familiar with polynomials by the time they begin to study abstract algebra.  When we examine polynomial expressions such as 
\begin{eqnarray*} 
p(x) & = & x^3 -3x +2 \\
q(x) & = & 3x^2 -6x +5,
\end{eqnarray*}
we have a pretty good idea of what $p(x) + q(x)$ and $p(x) q(x)$ mean.  We just add and multiply polynomials as functions; that is, 
\begin{eqnarray*}
(p +q)(x) & = & p(x) + q(x) \\
& = &  ( x^3 - 3 x + 2 ) + ( 3 x^2 - 6 x + 5 ) \\
& = & x^3 + 3 x^2 - 9 x + 7
\end{eqnarray*}
and
\begin{eqnarray*}
(p q)(x) & = & p(x)  q(x) \\
& = &  ( x^3 - 3 x + 2 )  ( 3 x^2 - 6 x + 5 ) \\
& = & 3 x^5 - 6 x^4 - 4 x^3 + 24 x^2 - 27 x + 10.
\end{eqnarray*}
It is probably no surprise that polynomials form a ring.  In this chapter we shall emphasize the algebraic structure of polynomials by studying polynomial rings.  We can prove many results for polynomial rings that are similar to the theorems we proved for the integers.  Analogs of prime numbers, of the division algorithm, and of the Euclidean algorithm exist for polynomials.   

 
\section{Polynomial Rings}\label{poly_section_1}
 
Throughout this chapter we shall assume that $R$ is a commutative ring with identity.  Any expression of the form 
$$
f(x) = \sum^{n}_{i=0} a_i x^i = a_0 + a_1 x +a_2 x^2 + \cdots + a_n x^n, 
$$
where $a_i \in R$ and $a_n \neq 0$, is called a {\bfi polynomial over $R$}\index{Polynomial!definition of} with {\bfi indeterminate}\index{Indeterminate} $x$.  The elements $a_0, a_1, \ldots, a_n$ are called the {\bfi coefficients\/} of $f$.  The coefficient $a_n$ is called the {\bfi leading coefficient}\index{Polynomial!leading coefficient of}.  A polynomial is called {\bfi monic\/}\index{Polynomial!monic}\index{Monic polynomial} if the leading coefficient is 1.  If $n$ is the largest nonnegative number for which $a_n \neq 0$, we say that the {\bfi degree\/}\index{Polynomial!degree of} of $f$ is $n$ and write $\deg f(x) = n$\label{polydegree}.  If no such $n$ exists---that is, if $f=0$ is the zero polynomial---then the degree of $f$ is defined to be $-\infty$.  We will denote the set of all polynomials with coefficients in a ring $R$ by $R[x]$\label{polynomialring}.  Two polynomials are equal exactly when their corresponding coefficients are equal; that is, if we let    
\begin{eqnarray*}
p(x) & = & a_0 + a_1 x + \cdots + a_n x^n \\
q(x) & = & b_0 + b_1 x + \cdots + b_m x^m,
\end{eqnarray*}
then $p(x) = q(x)$ if and only if $a_i = b_i$ for all $i \geq 0$.

To show that the set of all polynomials forms a ring, we must first define addition and multiplication.  We define the sum of two polynomials as follows.  Let
\begin{eqnarray*}
p(x) & = & a_0 + a_1 x + \cdots + a_n x^n \\
q(x) & = & b_0 + b_1 x + \cdots + b_m x^m.
\end{eqnarray*}
Then the sum of $p(x)$ and $q(x)$ is
$$
p(x) + q(x) = c_0 + c_1 x + \cdots + c_k x^k,
$$
where $c_i = a_i + b_i$ for each $i$.  We define the product of $p(x)$ and $q(x)$ to be 
$$
p(x) q(x) = c_0 + c_1 x + \cdots + c_{m + n} x^{m + n},
$$
where
$$
c_i = \sum_{k = 0}^i a_k b_{i - k} = a_0  b_i + a_1 b_{i -1} + \cdots + a_{i -1} b _1 + a_i b_0
$$
for each $i$.  Notice that in each case some of the coefficients may be zero. 

\begin{example}{poly_operations}
Suppose that
$$
p(x) = 3 + 0 x + 0 x^2 + 2 x^3 + 0 x^4
$$
and
$$
q(x) = 2 + 0 x - x^2 + 0 x^3 + 4 x^4
$$
are polynomials in ${\mathbb Z}[x]$.  If the coefficient of some term in a polynomial is zero, then we usually just omit that term.  In this case we  would write $p(x) =  3 + 2 x^3$ and $q(x) = 2 - x^2 + 4 x^4$.  The sum of these two polynomials is
$$
p(x) + q(x)= 5 - x^2 + 2 x^3 + 4 x^4.
$$
The product,
$$
p(x) q(x) = (3 + 2 x^3)( 2 - x^2 + 4 x^4 ) =  6 - 3x^2 + 4 x^3 + 12 x^4  - 2 x^5 + 8 x^7,
$$
can be calculated either by determining the $c_i$'s in the definition or by simply multiplying polynomials in the same way as we have
always done.
\end{example}

\begin{example}{poly_domain}
Let
$$
p(x) = 3 + 3 x^3
$$
and
$$
q(x) = 4 + 4 x^2 + 4 x^4
$$
be polynomials in ${\mathbb Z}_{12}[x]$. The sum of $p(x)$ and $q(x)$ is $7 + 4 x^2 + 3 x^3 + 4 x^4$.  The product of the two polynomials is  the zero polynomial.  This example tells us that $R[x]$ cannot be an integral domain if $R$ is not an integral domain.
\mbox{\hspace{1in}}
\end{example}
 
\medskip
 
\begin{theorem}\label{poly_theorem_1}
Let $R$ be a commutative ring with identity.  Then $R[x]$ is a commutative ring with identity.
\end{theorem}

\begin{proof}
Our first task is to show that $R[x]$ is an abelian group under polynomial addition.  The zero polynomial, $f(x) = 0$, is the additive identity.  Given a polynomial $p(x) = \sum_{i = 0}^{n} a_i x^i$, the inverse of $p(x)$ is easily verified to be $-p(x) = \sum_{i = 0}^{n} (-a_i) x^i = -\sum_{i = 0}^{n} a_i x^i$.  Commutativity and associativity follow immediately from the definition of polynomial addition and from the fact that addition in $R$ is both commutative and associative.

To show that polynomial multiplication is associative, let
\begin{eqnarray*}
p(x) & = & \sum_{i=0}^{m} a_i x^i, \\
q(x) & = & \sum_{i=0}^{n} b_i x^i, \\
r(x) & = & \sum_{i=0}^{p} c_i x^i. 
\end{eqnarray*}
Then
\begin{eqnarray*}
[p(x) q(x)] r(x) 
& = &
\left[
\left(
\sum_{i=0}^{m} a_i x^i 
\right)
\left( 
\sum_{i=0}^{n} b_i x^i
\right)
\right]
\left(
\sum_{i=0}^{p} c_i x^i
\right) \\
& = &
\left[
\sum_{i=0}^{m+n}
\left( 
\sum_{j=0}^{i} a_j b_{i-j}
\right) x^i
\right]
\left(
\sum_{i=0}^{p} c_i x^i
\right) \\
& = &
\sum_{i=0}^{m+n+p} 
\left[
\sum_{j=0}^{i}
\left(
\sum_{k=0}^j a_k b_{j-k} 
\right) c_j
\right]
 x^i \\
& = &
\sum_{i=0}^{m+n+p} 
\left(
\sum_{j+k+l=i} a_j b_k c_r
\right) x^i \\
& = &
\sum_{i=0}^{m+n+p}
\left[
\sum_{j=0}^{i} a_j 
\left(
\sum_{k=0}^{i-j} b_k c_{i-j-k}
\right)
\right]  x^i \\
& = &
\left(
\sum_{i=0}^{m} a_i x^i 
\right)
\left[
\sum_{i=0}^{n+p} 
\left(
\sum_{j=0}^{i} b_j c_{i-j}
\right) x^i
\right] \\
& = &
\left(
\sum_{i=0}^{m} a_i x^i 
\right)
\left[
\left( 
\sum_{i=0}^{n} b_i x^i
\right)
\left(
\sum_{i=0}^{p} c_i x^i
\right)
\right] \\
& = & p(x) [ q(x) r(x) ]
\end{eqnarray*}
The commutativity and distribution properties of polynomial multiplication are proved in a similar manner.  We shall leave the proofs of these properties as an exercise.
\end{proof}

\begin{proposition}\label{poly_theorem_2}
Let $p(x)$ and $q(x)$ be polynomials in $R[x]$, where $R$ is an integral domain.  Then $\deg p(x) + \deg q(x) = \deg( p(x) q(x) )$.  Furthermore, $R[x]$ is an integral domain.
\end{proposition}

\begin{proof}
Suppose that we have two nonzero polynomials 
$$
p(x) = a_m x^m + \cdots + a_1 x + a_0
$$
and 
$$
q(x) = b_n x^n + \cdots + b_1 x + b_0
$$
with $a_m \neq 0$ and $b_n \neq 0$. The degrees of $p$ and $q$ are $m$ and $n$, respectively.  The leading term of $p(x) q(x)$ is $a_m b_n x^{m + n}$, which cannot be zero since $R$ is an integral domain; hence, the degree of $p(x) q(x)$ is $m + n$, and $p(x)q(x) \neq 0$.  Since $p(x) \neq 0$ and $q(x) \neq 0$ imply that $p(x)q(x) \neq 0$, we know that $R[x]$ must also be an integral domain.
\end{proof}

\medskip

We also want to consider polynomials in two or more variables, such as $x^2 - 3 x y + 2 y^3$.  Let $R$ be a ring and suppose that we are given two indeterminates $x$ and $y$.  Certainly we can form the ring $(R[x])[y]$.  It is straightforward but perhaps tedious to show that $(R[x])[y] \cong R([y])[x]$.  We shall identify these two rings by this isomorphism and simply write $R[x,y]$.  The ring $R[x, y]$ is called the {\bfi ring of polynomials in two indeterminates $x$ and $y$ with coefficients in} $R$.  We can define the {\bfi ring of polynomials in}\index{Polynomial!in $n$ indeterminates} $n$ {\bfi indeterminates with coefficients in} $R$ similarly.  We shall denote this ring by $R[x_1, x_2, \ldots, x_n]$\label{notepolynvar}.  

\begin{theorem}\label{poly_theorem_3}
Let $R$ be a commutative ring with identity and $\alpha \in R$.  Then we have a ring homomorphism  $\phi_{\alpha} : R[x] \rightarrow R$\label{noteevalhomo} defined by  
$$
\phi_{\alpha} (p(x) ) = p( \alpha ) = a_n \alpha^n + \cdots + a_1 \alpha + a_0,
$$
where $p( x ) = a_n x^n + \cdots + a_1 x + a_0$.
\end{theorem}

\begin{proof}
Let $p(x) = \sum_{i = 0}^n a_i x^i$ and $q(x) = \sum_{i = 0}^m b_i x^i$.  It is easy to show that $\phi_{\alpha}(p(x) + q(x)) = \phi_{\alpha}(p(x))  + \phi_{\alpha}(q(x))$.  To show that multiplication is preserved under the map $\phi_{\alpha}$, observe that
\begin{eqnarray*}
\phi_{\alpha} (p(x) ) \phi_{\alpha} (q(x) ) 
& = & 
p( \alpha ) q(\alpha) \\
& = & 
\left(
\sum_{i = 0}^n a_i \alpha^i
\right)
\left(
\sum_{i=0}^m b_i \alpha^i
\right) \\
& = & 
\sum_{i=0}^{m+n} 
\left(
\sum_{k=0}^i a_k b_{i-k}
\right)  \alpha^i \\
& = &
\phi_{\alpha} (p(x) q(x) ). 
\end{eqnarray*}
\end{proof}

\medskip

The map  $\phi_{\alpha} : R[x] \rightarrow R$ is called the {\bfi evaluation homomorphism\/}\index{Homomorphism!evaluation} at~$\alpha$. 
 

\section{The Division Algorithm}

Recall that the division algorithm for integers (Theorem~1.3) says
that if $a$ and $b$ are integers with $b>0$, then there exist unique
integers $q$ and $r$ such that $a = bq+r$, where $0 \leq r < b$. The
algorithm by which $q$ and $r$ are found is just long division.  A
similar theorem exists for polynomials.	The division algorithm for
polynomials has several important consequences. Since its proof is
very similar to the corresponding proof for integers, it is worthwhile
to review Theorem~1.3 at this point.  
 
 
\begin{theorem} 
{\bf (Division Algorithm)}\index{Division algorithm!for
polynomials}\index{Algorithm!division} Let $f(x)$ and $g(x)$ be two
nonzero polynomials in $F[x]$, where $F$ is a field and  $g(x)$ is a
nonconstant polynomial.  Then there exist unique polynomials $q(x),
r(x) \in F[x]$ such that 
$$
f(x) = g(x) q(x) + r(x),
$$
where either  $\deg r(x) < \deg g(x)$ or $r(x)$ is the zero
polynomial. 
\end{theorem}
 
 
\begin{proof}
We will first consider the existence of $q(x)$ and $r(x)$. Let $S = \{
f(x) - g(x) h(x)   :  h(x) \in F[x]  \}$ and assume that 
$$
g(x) = a_0 + a_1 x + \cdots + a_n x^n
$$
is a polynomial of degree $n$. This set is nonempty since $f(x) \in
S$. If $f(x)$ is the zero polynomial, then 
$$
0 = f(x) = 0 \cdot g(x) + 0;
$$
hence, both $q$ and $r$ must also be the zero polynomial. 
 
 
Now suppose that the zero polynomial is not in $S$. In this case the
degree of every polynomial in $S$ is nonnegative.  Choose a polynomial
$r(x)$ of smallest degree in $S$; hence, there must exist a $q(x) \in
F[x]$ such that  
$$
r(x) = f(x) - g(x) q(x),
$$
or 
$$
f(x) = g(x ) q(x) + r(x).
$$
We need to show that the degree of $r(x)$ is less than the degree of
$g(x)$. Assume that $\deg g(x) \leq \deg r(x)$. Say $r(x) = b_0 + b_1 
x + \cdots + b_m x^m$ and $m \geq n$. Then
\begin{eqnarray*}
f(x) - g(x) [ q(x) - (b_m/a_n) x^{m-n} ]
& = & f(x) - g(x) q(x) \\
&   & +  (b_m/a_n) x^{m-n} g(x)  \\
& = & r(x) + (b_m/a_n) x^{m-n} g(x) \\
& = & r(x) + b_m x^m \\
&   & + \mbox{ terms of lower degree}
\end{eqnarray*}
is in $S$. This is a polynomial of lower degree than $r(x)$, which
contradicts the fact that $r(x)$ is a polynomial of smallest degree
in $S$; hence, $\deg r(x) < \deg g(x)$.
 
 
To show that  $q(x)$ and $r(x)$ are unique, suppose that there exist
two other polynomials $q'(x)$ and $r'(x)$ such that $f(x) = g(x) q'(x)
+ r'(x)$ and $\deg r'(x) < \deg g(x)$ or $r'(x) = 0$, so that
$$
f(x) = g(x) q(x) + r(x) = g(x) q'(x) + r'(x),
$$
and
$$
g(x) [q(x) - q'(x) ] = r'(x) - r(x).
$$
If $g$ is not the zero polynomial, then 
$$
\deg( g(x) [q(x) - q'(x) ] )= \deg( r'(x) - r(x) ) \geq \deg g(x).
$$
However, the degrees of both $r(x)$ and $r'(x)$ are strictly less than
the degree of $g(x)$; therefore, $r(x) = r'(x)$ and $q(x) = q'(x)$.
\end{proof}
 
 
\begin{example}{poly_division}
The division algorithm merely formalizes long division of polynomials,
a task we have been familiar with since high school. For example,
suppose that we divide $x^3 - x^2 + 2 x - 3$ by $x - 2$.  
$$
\begin{array}{rrcrcrcr}
        &  x^2  &  +  &      x  &  +  &    4  &     &     \\ \cline{2-8}
 \multicolumn{1}{r|}{x - 2}
  &  x^3  &  -  &    x^2  &  +  &  2 x  &  -  &  3  \\
        &  x^3  &  -  &  2 x^2  &     &       &     &     \\ \cline{2-8}
        &       &     &    x^2  &  +  &  2 x  &  -  &  3  \\
        &       &     &    x^2  &  -  &  2 x  &     &     \\ \cline{4-8}
        &       &     &         &     &  4 x  &  -  &  3  \\
        &       &     &         &     &  4 x  &  -  &  8  \\ \cline{6-8}
        &       &     &         &     &       &     &  5 
\end{array}
$$
Hence, $x^3 - x^2 + 2 x - 3 = (x - 2) (x^2 + x + 4 ) + 5$.
\end{example}

 
 
Let $p(x)$ be a polynomial in $F[x]$ and $\alpha \in F$.  We say that
$\alpha$ is a {\bfi zero\/}\index{Polynomial!zero of}\index{Zero!of a
polynomial} or {\bfi root\/}\index{Polynomial!root of} of $p(x)$ if
$p(x)$ is in the kernel of the evaluation homomorphism
$\phi_{\alpha}$. All we are really saying here is that $\alpha$ is a
zero of $p(x)$ if $p(\alpha) = 0$.  
 
 
\begin{corollary}
Let $F$ be a field.
An element $\alpha \in F$ is a zero of $p(x) \in F[x]$ if and only if
$x - \alpha$ is a factor of $p(x)$ in $F[x]$. 
\end{corollary}
 
 
\begin{proof}
Suppose that $\alpha \in F$ and $p( \alpha ) = 0$. By the division
algorithm, there exist polynomials $q(x)$ and $r(x)$ such that
$$
p(x) = (x -\alpha) q(x) + r(x)
$$
and the degree of $r(x)$ must be less than the degree of $x -\alpha$.
Since the degree of $r(x)$ is less than 1, $r(x) = a$ for $a \in F$;
therefore, 
$$
p(x) = (x -\alpha) q(x) + a.
$$
But 
$$
0 = p(\alpha) = 0 \cdot q(x) + a = a;
$$
consequently, $p(x) = (x - \alpha) q(x)$, and $x - \alpha$ is a factor 
of $p(x)$.
 
 
Conversely, suppose that $x - \alpha$ is a factor of $p(x)$; say $p(x)
= (x - \alpha) q(x)$. Then $p( \alpha ) = 0 \cdot q(x) = 0$. 
\end{proof}
 
 
\begin{corollary}\label{zeros_poly}
Let $F$ be a field. A nonzero polynomial $p(x)$ of degree $n$ in
$F[x]$ can have at most $n$ distinct zeros in $F$.  
\end{corollary}
 
 
\begin{proof}
We will use induction on the degree of $p(x)$. If $\deg p(x) = 0$,
then $p(x)$ is a constant polynomial and has no zeros.  Let $\deg p(x)
= 1$. Then $p(x) = ax +b$ for some $a$ and $b$ in $F$. If $\alpha_1$ and
$\alpha_2$ are zeros of $p(x)$, then $a\alpha_1 + b = a\alpha_2 +b$ or
$\alpha_1 = \alpha_2$. 
 
Now assume that $\deg p(x) > 1$. If $p(x)$ does not have a zero in
$F$, then we are done.  On the other hand, if $\alpha$ is a zero of
$p(x)$, then $p(x) = (x - \alpha ) q(x)$ for some $q(x) \in F[x]$ by
Corollary~15.5. The degree of $q(x)$ is $n-1$ by Proposition~15.2.
Let $\beta$ be some other zero of $p(x)$ that is distinct from
$\alpha$. Then $p(\beta) = (\beta - \alpha) q(\beta) = 0$. Since
$\alpha \neq \beta$ and $F$ is a field, $q(\beta ) = 0$. By our
induction hypothesis, $p(x)$ can have at most $n -1$ zeros in $F$ that
are distinct from $\alpha$. Therefore, $p(x)$ has at most $n$ distinct
zeros in $F$.
\end{proof}
 
 
\medskip
 
 
Let $F$ be a field.  A monic polynomial $d(x)$ is a {\bfi greatest
common divisor\/}\index{Polynomial!greatest common divisor
of}\index{Greatest common divisor!of two polynomials} of polynomials 
$p(x), q(x) \in F[x]$ if $d(x)$ evenly divides both $p(x)$ and $q(x)$;
and, if for any other polynomial $d'(x)$ dividing both $p(x)$ and
$q(x)$,  $d'(x) \mid d(x)$.  We write $d(x) = \gcd( p(x), q( x))$. Two
polynomials $p(x)$ and $q(x)$ are {\bfi relatively prime\/} if $\gcd(
p(x), q(x) ) = 1$. 
 
 
\begin{proposition}
Let $F$ be a field and suppose that $d(x)$ is the greatest common
divisor of two polynomials $p(x)$ and $q(x)$ in $F[x]$. Then there
exist polynomials $r(x)$ and $s(x)$ such that
$$
d(x) = r(x) p(x) + s(x) q(x).
$$
Furthermore, the greatest common divisor of two polynomials is unique. 
\end{proposition}
 
 
\begin{proof}
Let $d(x)$ be the monic polynomial of smallest degree in the set 
$$
S = \{ f(x) p(x) + g(x) q(x) : f(x), g(x) \in F[x]  \}.
$$
 We can write
$d(x) = r(x) p(x) + s(x) q(x)$ for two polynomials  $r(x)$ and $s(x)$
in $F[x]$. We need to show that $d(x)$ divides both $p(x)$ and $q(x)$.
We shall first show that $d(x)$ divides $p(x)$. By the division
algorithm, there exist polynomials $a(x)$ and $b(x)$ such that
$p(x) = a(x) d(x) + b(x)$, where $b(x)$ is either the
zero polynomial or $\deg b(x) < \deg d(x)$.  Therefore, 
\begin{eqnarray*}
b(x) & = & p(x) - a(x) d(x) \\
& = & p(x) - a(x)( r(x) p(x) + s(x) q(x))  \\
& = & p(x) - a(x) r(x) p(x) - a(x)  s(x) q(x) \\
& = & p(x)( 1  - a(x) r(x) ) + q(x) ( - a(x)  s(x) )
\end{eqnarray*}
is a linear combination of $p(x)$ and $q(x)$ and therefore must be in
$S$. However, $b(x)$ must be the zero polynomial since $d(x)$ was
chosen to be of smallest degree; consequently, $d(x)$ divides $p(x)$. 
A symmetric argument shows that $d(x)$ must also divide $q(x)$; hence,
$d(x)$ is a common divisor of $p(x)$ and $q(x)$. 
 
 
To show that $d(x)$ is a greatest common divisor of $p(x)$ and $q(x)$,
suppose that $d'(x)$ is another common divisor of $p(x)$ and $q(x)$.
We will show that $d'(x) \mid d(x)$.  Since $d'(x)$ is a common
divisor of $p(x)$ and $q(x)$, there exist polynomials $u(x)$ and
$v(x)$ such that $p(x) = u(x) d'(x)$ and $q(x) = v(x) d'(x)$.
Therefore, 
\begin{eqnarray*}
d(x) & = & r(x) p(x) + s(x) q(x) \\
& = &  r(x) u(x) d'(x) + s(x) v(x) d'(x) \\
& = & d'(x) [r(x) u(x) + s(x) v(x)].
\end{eqnarray*}
Since $d'(x) \mid d(x)$, $d(x)$ is a greatest common divisor of $p(x)$ 
and $q(x)$.
 
 
Finally, we must show that the greatest common divisor of $p(x)$ and
$q(x))$ is unique. Suppose that $d'(x)$ is another greatest common
divisor of $p(x)$ and $q(x)$. We have just shown that there exist
polynomials $u(x)$ and $v(x)$ in $F[x]$ such that $d(x) = d'(x)[r(x)
u(x) + s(x) v(x)]$. Since 
$$
\deg d(x) =  \deg d'(x) + \deg[r(x) u(x) + s(x) v(x)]
$$
and $d(x)$ and $d'(x)$ are both greatest common divisors, $\deg d(x) =
\deg d'(x)$. Since $d(x)$ and $d'(x)$ are both monic polynomials of
the same degree, it must be the case that $d(x) =~d'(x)$.
\end{proof}
 
 
\medskip
 
 
Notice the similarity between the proof of Proposition~15.7 and the 
proof of Theorem~1.4.
 
 
 
\section{Irreducible Polynomials}
 
 
A nonconstant polynomial $f(x) \in F[x]$ is {\bfi
irreducible\/}\index{Polynomial!irreducible}\index{Irreducible polynomial}
over a field $F$ if $f(x)$ cannot be expressed as a product of two
polynomials $g(x)$ and $h(x)$ in $F[x]$, where the degrees of $g(x)$
and $h(x)$ are both smaller than the degree of $f(x)$.  Irreducible
polynomials function as the ``prime numbers'' of polynomial rings.
 
 
\begin{example}{poly_irred}
The polynomial $x^2 - 2 \in {\mathbb Q}[x]$ is irreducible since it
cannot be factored any further over the rational numbers. Similarly,
$x^2 + 1$ is  irreducible over the real numbers. 
\end{example}
 
 
\begin{example}{finite_poly}
The polynomial $p(x) = x^3 + x^2 + 2$ is irreducible over ${\mathbb
Z}_3[x]$. Suppose that this polynomial was reducible over ${\mathbb
Z}_3[x]$.  By the division algorithm there would have to be a factor
of the form $x - a$, where $a$ is some element in ${\mathbb Z}_3[x]$.
Hence, it would have to be true that $p(a) = 0$.  However,
\begin{eqnarray*}
p(0) & = & 2 \\
p(1) & = & 1 \\
p(2) & = & 2.
\end{eqnarray*}
Therefore, $p(x)$ has no zeros in ${\mathbb Z}_3$ and must be
irreducible. 
\end{example}
 
 
\begin{lemma}
Let $p(x) \in {\mathbb Q}[x]$.  Then
$$
p(x) = \frac{r}{s}(a_0 + a_1 x + \cdots + a_n x^n),
$$
where $r, s, a_0, \ldots, a_n$ are integers, the $a_i$'s are
relatively prime, and $r$ and $s$ are relatively prime. 
\end{lemma}
 
 
\begin{proof}
Suppose that
$$
p(x) = \frac{b_0}{c_0} + \frac{b_1}{c_1} x + \cdots + \frac{b_n}{c_n}
x^n,
$$
where the $b_i$'s and the $c_i$'s are integers. We can rewrite $p(x)$
as 
$$
p(x) = \frac{1}{c_0 \cdots c_n} (d_0 + d_1 x + \cdots + d_n x^n),
$$
where $d_0, \ldots, d_n$ are integers. Let $d$ be the greatest common
divisor of $d_0, \ldots, d_n$.  Then
$$
p(x) = \frac{d}{c_0 \cdots c_n} (a_0 + a_1 x + \cdots + a_n x^n),
$$
where $d_i = d a_i$ and the $a_i$'s are relatively prime. Reducing $d
/(c_0 \cdots c_n)$ to its lowest terms, we can write
$$
p(x) = \frac{r}{s}(a_0 + a_1 x + \cdots + a_n x^n), 
$$
where $\gcd(r,s) = 1$.
\end{proof}
 
 
\begin{theorem}[Gauss's Lemma]
Let $p(x) \in {\mathbb Z}[x]$ be a monic polynomial such that $p(x)$
factors into a product of two polynomials $\alpha(x)$ and $\beta(x)$
in ${\mathbb Q}[x]$, where the degrees of both $\alpha(x)$ and $\beta(x)$
are less than the degree of $p(x)$. Then $p(x) = a(x) b(x)$, where
$a(x)$ and $b(x)$ are monic polynomials in ${\mathbb Z}[x]$ with $\deg
\alpha(x) = \deg a(x)$ and $\deg \beta(x) = \deg b(x)$. 
\end{theorem}
 
 
\begin{proof}
By Lemma~15.8, we can assume that
\begin{eqnarray*}
\alpha(x)  & = &  \frac{c_1}{d_1} (a_0 + a_1 x + \cdots + a_m x^m )=
\frac{c_1}{d_1} \alpha_1(x) \\
\beta(x)  & = &  \frac{c_2}{d_2} (b_0 + b_1 x + \cdots + b_n x^n)  =
\frac{c_2}{d_2} \beta_1(x),
\end{eqnarray*}
where the $a_i$'s are relatively prime and the $b_i$'s are relatively
prime. Consequently, 
$$
p(x) = \alpha(x) \beta(x) = \frac{c_1 c_2}{d_1 d_2} \alpha_1(x)
\beta_1(x) = \frac{c}{d} \alpha_1(x) \beta_1(x),
$$
where  $c/d$ is the product of $c_1/d_1$ and $c_2/d_2$ expressed in
lowest terms. Hence, $d p(x) = c \alpha_1(x) \beta_1(x)$. 

 
 
If $d = 1$, then $c a_m b_n = 1$ since $p(x)$ is a monic polynomial.
Hence, either $c=1$ or $c = -1$. If $c=1$, then either $a_m = b_n = 1$ or
$a_m = b_n = -1$. In the first case $p(x) = \alpha_1(x) \beta_1(x)$,
where $\alpha_1(x)$ and $\beta_1(x)$ are monic polynomials with $\deg
\alpha(x) = \deg \alpha_1(x)$ and $\deg \beta(x) = \deg \beta_1(x)$.
In the second case $a(x) = -\alpha_1(x)$ and $b(x) = -\beta_1(x)$ are
the correct monic polynomials since $p(x) = (-\alpha_1(x))(-
\beta_1(x)) = a(x) b(x)$. The case in which $c = -1$ can be handled
similarly. 
 
 
Now suppose that $d \neq 1$. Since $\gcd(c, d) = 1$, there exists a
prime $p$ such that $p \mid d$ and $p \notdivide c$. Also, since the
coefficients of $\alpha_1(x)$ are relatively prime, there exists a
coefficient $a_i$ such that $p \notdivide a_i$.  Similarly, there exists
a coefficient $b_j$ of $\beta_1(x)$ such that $p \notdivide b_j$. Let 
$\alpha_1'(x)$ and $\beta_1'(x)$ be the polynomials in ${\mathbb Z}_p[x]$
obtained by reducing the coefficients of $\alpha_1(x)$ and
$\beta_1(x)$ modulo $p$. Since $p \mid d$, $\alpha_1'(x) \beta_1'(x) = 
0$ in ${\mathbb Z}_p[x]$. However, this is impossible since neither
$\alpha_1'(x)$ nor $\beta_1'(x)$ is the zero polynomial and ${\mathbb
Z}_p[x]$ is an integral domain.  Therefore, $d=1$ and the theorem is
proven. 
\end{proof}
 
 
\begin{corollary}
Let $p(x) = x^n + a_{n-1} x^{n-1} + \cdots + a_0$ be  a polynomial
with coefficients in ${\mathbb Z}$ and $a_0 \neq 0$. If $p(x)$ has a zero
in ${\mathbb Q}$, then $p(x)$ also has a zero $\alpha$ in ${\mathbb Z}$.
Furthermore, $\alpha$ divides $a_0$.  
\end{corollary}
 
 
\begin{proof}
Let $p(x)$ have a zero $a \in {\mathbb Q}$. Then $p(x)$ must have a
linear factor $x-a$.  By Gauss's Lemma, $p(x)$ has a factorization
with a linear factor in ${\mathbb Z}[x]$. Hence, for some $\alpha \in
{\mathbb Z}$ 
$$
p(x) = (x - \alpha)( x^{n-1} + \cdots - a_0 / \alpha ).
$$
Thus $a_0 /\alpha \in {\mathbb Z}$ and so $\alpha \mid a_0$.
\end{proof}
 
 
\begin{example}{poly_factor}
Let $p(x) = x^4 - 2 x^3 + x + 1$. We shall show that $p(x)$ is
irreducible over ${\mathbb Q}[x]$.  Assume that $p(x)$ is reducible. Then
either $p(x)$ has a linear factor, say $p(x) = (x - \alpha) q(x)$,
where $q(x)$ is a polynomial of degree three, or $p(x)$ has two 
quadratic factors. 
 
 
If $p(x)$ has a linear factor in ${\mathbb Q}[x]$, then it has a zero in
${\mathbb Z}$.  By  Corollary~15.10, any zero must divide 1 and therefore
must be $\pm 1$; however, $p(1) = 1$ and $p(-1)= 3$. Consequently, we
have eliminated the possibility that $p(x)$ has any linear factors.   
 
 
Therefore, if $p(x)$ is reducible it must factor into two quadratic 
polynomials, say
\begin{eqnarray*}
p(x) & = & (x^2 + ax + b )( x^2 + cx + d ) \\
& = & x^4 + (a + c)x^3 + (ac + b + d)x^2 + (ad + bc)x + bd,
\end{eqnarray*}
where each factor is in ${\mathbb Z}[x]$ by Gauss's Lemma. Hence,
\begin{eqnarray*}
a + c & = & - 2 \\
ac + b + d & = & 0 \\
ad + bc & = & 1 \\
bd & = & 1.
\end{eqnarray*}
Since $bd = 1$, either $b = d = 1$ or $ b = d = -1$. In either case $b
= d$ and so 
$$
ad + bc  = b( a + c ) = 1.
$$
Since $a + c = -2$, we know that $-2b = 1$. This is impossible since
$b$ is an integer. Therefore, $p(x)$ must be irreducible over ${\mathbb
Q}$. 
\end{example}
 
 
\begin{theorem}[Eisenstein's Criterion]\index{Eisenstein's Criterion}
Let $p$ be a prime and suppose that
$$
f(x) = a_n x^n + \cdots + a_0 \in {\mathbb Z}[x].
$$
If $p \mid a_i$ for $i = 0, 1, \ldots, a_{n-1}$, but $p \notdivide a_n$
and $p^2 \notdivide a_0$, then $f(x)$ is irreducible over ${\mathbb Q}$. 
\end{theorem}
 
 
\begin{proof}
By Gauss's Lemma, we need only show that $f(x)$ does not factor into
polynomials of lower degree in ${\mathbb Z}[x]$. Let  
$$
f(x) = (b_rx^r + \cdots + b_0)(c_s x^s + \cdots + c_0 )
$$
be a factorization in ${\mathbb Z}[x]$, with $b_r$ and $c_s$ not equal to
zero and $r, s < n$. Since $p^2$ does not divide $a_0 = b_0 c_0$,
either $b_0$ or $c_0$ is not divisible by $p$. Suppose that $p \notdivide
b_0$ and $p \mid c_0$. Since $p \notdivide a_n$ and $a_n = b_r c_s$,
neither $b_r$ nor $c_s$ is divisible by $p$. Let $m$ be the smallest
value of $k$ such that $p \notdivide c_k$. Then  
$$
a_m = b_0 c_m + b_1 c_{m-1} + \cdots + b_m c_0
$$
is not divisible by $p$, since each term on the right-hand side of the
equation is divisible by $p$ except for $b_0 c_m$.  Therefore, $m =n$
since $a_i$ is divisible by $p$ for $m < n$.  Hence, $f(x)$ cannot be
factored into polynomials of lower degree and therefore must be
irreducible. 
\end{proof}
 
 
\begin{example}{poly_eisen}
The polynomial
$$
p(x) = 16 x^5  -9 x^4 + 3x^2 + 6 x - 21
$$
is easily seen to be irreducible over ${\mathbb Q}$ by Eisenstein's
Criterion if we let $p = 3$.
\end{example}
 
 
\medskip
 
 
Eisenstein's Criterion is more useful in constructing irreducible
polynomials of a certain degree over ${\mathbb Q}$ than in determining the
irreducibility of an arbitrary polynomial in ${\mathbb Q}[x]$: given an
arbitrary polynomial, it is not very likely that we can apply
Eisenstein's Criterion.  The real value of Theorem~15.11 is that we now
have an easy method of generating irreducible polynomials of any
degree. 
 
 
 
\subsection*{Ideals in $F[x]$}
 
 
Let $F$ be a field. Recall that a principal ideal in $F[x]$ is an
ideal $\langle p(x) \rangle$ generated by some polynomial $p(x)$; that
is,
$$
\langle p(x) \rangle = \{ p(x) q(x) : q(x) \in F[x] \}.
$$
 
 
\begin{example}{poly_ideal}
The polynomial $x^2$ in $F[x]$ generates the ideal $\langle x^2
\rangle$ consisting of all polynomials with no constant term or term
of degree 1.
\end{example}
 
 
\begin{theorem}
If $F$ is a field, then every ideal in $F[x]$ is a principal ideal. 
\end{theorem}
 
 
\begin{proof}
Let $I$ be an ideal of $F[x]$.  If $I$ is the zero ideal, the theorem
is easily true.  Suppose that $I$ is a nontrivial ideal in $F[x]$, and
let $p(x) \in I$ be a nonzero element of minimal degree. If $\deg
p(x)= 0$, then $p(x)$ is a nonzero constant and 1 must be in $I$.
Since 1 generates all of $F[x]$, $\langle 1 \rangle = I = F[x]$ and
$I$ is again a principal ideal. 
 
 
Now assume that $\deg p(x) \geq 1$ and let $f(x)$ be any element in
$I$.  By the division algorithm there exist $q(x)$ and $r(x)$ in
$F[x]$ such that $f(x) = p(x) q(x) + r(x)$ and $\deg r(x) < \deg 
p(x)$. Since $f(x), p(x) \in I$ and $I$ is an ideal, $r(x) = f(x) -
p(x) q(x)$ is also in $I$. However, since we chose $p(x)$ to be of
minimal degree, $r(x)$ must be the zero polynomial. Since we can write
any element $f(x)$ in $I$ as $p(x) q(x)$ for some $q(x) \in F[x]$, it
must be the case that $I = \langle p(x) \rangle$. 
\end{proof}
 
 
\begin{example}{poly_xy}
It is not the case that every ideal in the ring $F[x,y]$ is a
principal ideal. Consider the ideal of $F[x, y]$ generated by the
polynomials $x$ and $y$.  This is the ideal of $F[x, y]$ consisting of
all polynomials with no constant term. Since both $x$ and $y$ are in
the ideal, no single polynomial can generate the entire ideal.
\end{example}
 
 
\begin{theorem}
Let $F$ be a field and suppose that $p(x) \in F[x]$. Then the ideal
generated by $p(x)$ is maximal if and only if $p(x)$ is irreducible.
\end{theorem}
 
 
\begin{proof}
Suppose that $p(x)$ generates a maximal ideal of $F[x]$. Then $\langle
p(x) \rangle$ is also a prime ideal of $F[x]$. Since a maximal ideal
must be properly contained inside $F[x]$, $p(x)$ cannot be a constant
polynomial. Let us assume that $p(x)$ factors into two polynomials of
lesser degree, say $p(x) = f(x) g(x)$. Since $\langle p(x) \rangle$ is
a prime ideal one of these factors, say $f(x)$, is in $\langle p(x)
\rangle$ and therefore be a multiple of $p(x)$. But this would imply
that $\langle p(x) \rangle \subset \langle f(x) \rangle$, which is 
impossible since $\langle p(x) \rangle$ is maximal.
 
 
Conversely, suppose that $p(x)$ is irreducible over $F[x]$. Let $I$ be
an ideal in $F[x]$ containing $\langle p(x) \rangle$. By Theorem~15.12,
$I$ is a principal ideal; hence, $I = \langle f(x) \rangle$ for some
$f(x) \in F[x]$. Since $p(x) \in I$, it must be the case that $p(x) =
f(x) g(x)$ for some $g(x) \in F[x]$. However, $p(x)$ is irreducible;
hence, either $f(x)$ or $g(x)$ is a constant polynomial. If $f(x)$ is
constant, then $I = F[x]$ and we are done.  If $g(x)$ is constant, then
$f(x)$ is a constant multiple of $I$ and $I = \langle p(x) \rangle$.
Thus, there are no proper ideals of $F[x]$ that properly contain~
\mbox{$\langle p(x)\rangle$}. 
\end{proof}

\histhead

\noindent{\small \histf
Throughout history, the solution of polynomial equations has been a challenging problem.  The Babylonians knew how to solve the equation $ax^2+bx+c=0$.  Omar Khayyam (1048--1131) devised methods of solving cubic equations through the use of  geometric constructions and conic sections.  The algebraic solution of the general cubic equation $ax^3+bx^2+cx+d=0$ was not discovered until the sixteenth century.  An Italian mathematician, Luca Paciola (ca. 1445--1509), wrote  in {\it Summa de Arithmetica} that the solution of the cubic was impossible.  This was taken as a challenge by the rest of the mathematical community.

Scipione del Ferro\index{Ferro, Scipione del} (1465--1526), of the University of Bologna, solved the ``depressed cubic,'' 
$$
ax^3 + cx + d = 0.
$$
He kept his solution an absolute secret.  This may seem surprising today, when mathematicians are usually very eager to publish their results, but in the days of the Italian Renaissance secrecy was customary. Academic appointments were not easy to secure and depended on the ability to prevail in public contests.  Such challenges could be issued at any time.  Consequently, any major new discovery was a valuable weapon in such a contest. If an opponent presented a list of problems to be solved, del Ferro could in turn present a list of depressed cubics.  He kept the secret of his discovery throughout his life, passing it on only on his deathbed to his student Antonio Fior\index{Fior, Antonio} (ca. 1506--?).  

Although Fior was not the equal of his teacher, he immediately issued a challenge to Niccolo Fontana (1499--1557).  Fontana was known as Tartaglia\index{Tartaglia} (the Stammerer).  As a youth he had suffered a blow from the sword of a French soldier during an attack on his village. He survived the savage wound, but his speech was permanently impaired.  Tartaglia sent Fior a list of 30 various mathematical problems; Fior countered by sending Tartaglia a list of 30 depressed cubics.  Tartaglia would either solve all 30 of the problems or absolutely fail.  After much effort Tartaglia finally succeeded in solving the depressed cubic and defeated Fior, who faded into obscurity. 

At this point another mathematician, Gerolamo Cardano\index{Cardano, Gerolamo} (1501--1576), entered the story.  Cardano wrote to Tartaglia, begging him for the solution to the depressed cubic.  Tartaglia refused several of his requests, then finally revealed the solution to Cardano after the latter swore an oath not to publish the secret or to pass it on to anyone else. Using the knowledge that he had obtained from Tartaglia, Cardano eventually solved the general cubic 
$$
a x^3 + bx^2 +cx +d = 0.
$$
Cardano shared the secret with his student, Ludovico Ferrari\index{Ferrari, Ludovico} (1522--1565), who solved the general quartic equation, 
$$
a x^4 + b x^3 + cx^2 + d x + e =0.
$$
In 1543, Cardano and Ferrari examined del Ferro's papers and discovered that he had also solved the depressed cubic.  Cardano felt that this relieved him of his obligation to Tartaglia, so he proceeded to publish the solutions in {\it Ars Magna} (1545), in which he gave credit to del Ferro for solving the special case of the cubic.  This resulted in a bitter dispute between Cardano and Tartaglia, who published the story of the oath a year later.
\histbox
} 
 

\markright{EXERCISES}
\section*{Exercises}
\exrule

{\small
\begin{enumerate}

\item
List all of the polynomials of degree 3 or less in ${\mathbb Z}_2[x]$.

\item
Compute each of the following.
\begin{enumerate}
 
 \item
$(5x^2 + 3x - 4) + (4x^2 - x + 9)$ in ${\mathbb Z}_{12}$
 
 \item
$(5x^2 + 3x - 4) (4x^2 - x + 9)$ in ${\mathbb Z}_{12}$
 
 \item
$(7x^3 + 3x^2 - x) + (6x^2 - 8x + 4)$ in ${\mathbb Z}_9$
 
 \item
$(3x^2 + 2x - 4) + (4x^2 + 2)$ in ${\mathbb Z}_5$
  
 \item
$(3x^2 + 2x - 4) (4x^2 + 2)$ in ${\mathbb Z}_5$
 
 \item
$(5x^2 + 3x - 2)^2$ in ${\mathbb Z}_{12}$
 
\end{enumerate}

\item
Use the division algorithm to find $q(x)$ and $r(x)$ such that $a(x) = q(x) b(x) + r(x)$ with $\deg r(x) < \deg b(x)$ for each of the following  pairs of polynomials.  
\begin{enumerate}
 
 \item
$p(x) = 5 x^3 + 6x^2 -  3 x + 4$ and $q(x) = x - 2$ in ${\mathbb Z}_7[x]$
 
 \item
$p(x) = 6 x^4 - 2 x^3 +  x^2 - 3 x + 1$ and $q(x) = x^2 + x - 2$ in ${\mathbb Z}_7[x]$
 
 \item
$p(x) = 4 x^5 - x^3 + x^2 + 4$ and $q(x) = x^3 - 2$ in ${\mathbb Z}_5[x]$ 
 
 \item
$p(x) = x^5 + x^3 -x^2 - x$ and $q(x) = x^3 + x$ in ${\mathbb Z}_2[x]$
 
\end{enumerate}

\item
Find the greatest common divisor of each of the following pairs $p(x)$ and $q(x)$ of polynomials. If $d(x) = \gcd( p(x), q(x) )$, find two polynomials $a(x)$ and $b(x)$ such that $a(x) p(x) + b(x) q(x) = d(x)$. 
\begin{enumerate}
 
 \item
$p(x) = 7x^3 + 6x^2 - 8x + 4$ and $q(x) = x^3 + x - 2$, where $p(x), q(x) \in {\mathbb Q}[x]$  
 
 \item
$p(x) = x^3 + x^2 - x + 1$ and $q(x) = x^3 + x - 1$, where $p(x), q(x) \in {\mathbb Z}_2[x]$
 
 \item
$p(x) = x^3 + x^2 - 4x + 4$ and $q(x) = x^3 + 3 x -2$, where $p(x), q(x) \in {\mathbb Z}_5[x]$
 
 \item
$p(x) = x^3 - 2 x + 4$ and $q(x) = 4 x^3 + x + 3$, where $p(x), q(x) \in {\mathbb Q}[x]$ 
 
\end{enumerate}

\item
Find all of the zeros for each of the following polynomials.
\begin{multicols}{2}
\begin{enumerate}

\item 
$5x^3 + 4x^2 - x + 9$ in ${\mathbb Z}_{12}$

\item 
$3x^3 - 4x^2 - x + 4$ in ${\mathbb Z}_{5}$

\item 
$5x^4 + 2x^2 - 3$ in ${\mathbb Z}_{7}$

\item 
$x^3 + x + 1$ in ${\mathbb Z}_2$

\end{enumerate}
\end{multicols}
 
\item
Find all of the units in ${\mathbb Z}[x]$.

\item
Find a unit $p(x)$ in ${\mathbb Z}_4[x]$ such that $\deg p(x) > 1$.

\item
Which of the following polynomials are irreducible over ${\mathbb Q}[x]$?
\begin{multicols}{2}
\begin{enumerate}

\item 
$x^4 - 2 x^3 + 2x^2 + x + 4$

\item 
$x^4 - 5x^3 + 3x -2$

\item 
$3 x^5 - 4 x^3 - 6 x^2 + 6$

\item 
$5 x^5 - 6 x^4 - 3 x^2 + 9 x - 15$

\end{enumerate}
\end{multicols}
 
\item
Find all of the irreducible polynomials of degrees 2 and 3 in ${\mathbb Z}_2[x]$. 
 
\item
Give two different factorizations of $x^2 + x + 8$ in ${\mathbb Z}_{10}[x]$. 

\item
Prove or disprove: There exists a polynomial $p(x)$ in ${\mathbb Z}_6[x]$ of degree $n$ with more than $n$ distinct zeros. 
 

%*************************THEORY*************************

\item
If $F$ is a field, show that $F[x_1, \ldots, x_n]$ is an integral  domain. 

\item
Show that the division algorithm does not hold for ${\mathbb Z}[x]$.  Why does it fail?

\item
Prove or disprove: $x^p + a$ is irreducible for any $a \in {\mathbb Z}_p$, where $p$ is prime.

\item
Let $f(x)$ be irreducible.  If $f(x) \mid p(x)q(x)$, prove that either $f(x) \mid p(x)$ or $f(x) \mid q(x)$.

\item
Suppose that $R$ and $S$ are isomorphic rings.  Prove that $R[x] \cong S[x]$.

\item
Let $F$ be a field and $a \in F$.  If $p(x) \in F[x]$, show that $p(a)$ is the remainder obtained when $p(x)$ is divided by $x-a$.

\item
Let ${\mathbb Q}^*$ be the multiplicative group of positive rational numbers.  Prove that ${\mathbb Q}^*$ is isomorphic to $( {\mathbb Z}[x], +)$.

\item
{\bf Cyclotomic Polynomials.}
The polynomial
$$
\Phi_n(x) = \frac{x^n - 1}{x - 1} = x^{n - 1} + x^{n - 2} + \cdots + x + 1
$$
is called the {\bfi cyclotomic polynomial}\index{Polynomial!cyclotomic}.  Show that $\Phi_p(x)$ is irreducible over ${\mathbb Q}$ for any prime $p$. 

\item
If $F$ is a field, show that there are infinitely many irreducible polynomials in $F[x]$.

\item
Let $R$ be a commutative ring with identity. Prove that multiplication is commutative in $R[x]$.

\item
Let $R$ be a commutative ring with identity. Prove that multiplication is distributive in $R[x]$.
 
\item
Show that $x^p - x$ has $p$ distinct zeros in ${\mathbb Z}_p[x]$, for any prime $p$.  Conclude that therefore
$$
x^p-x = x(x-1)(x-2) \cdots (x - (p - 1)).
$$
 
\item
Let $F$ be a ring and $f(x) = a_0 + a_1 x + \cdots + a_n x^n$ be in $F[x]$. Define $f'(x) = a_1  + 2 a_2 x + \cdots + n a_n x^{n - 1}$ to be the {\bfi derivative\/}\index{Derivative} of $f(x)$. 
\begin{enumerate}
 
 \item
Prove that
$$
(f + g)'(x) = f'(x) + g'(x).
$$
Conclude that we can define a homomorphism of abelian groups $D : F[x] \rightarrow F[x]$ by $(D(f(x)) = f'(x)$.
 
 \item
Calculate the kernel of $D$ if $\mbox{char} F = 0$.
 
 \item
Calculate the kernel of $D$ if $\mbox{char} F = p$.
 
 \item
Prove that
$$
(fg)'(x) = f'(x)g(x) + f(x) g'(x).
$$
 
 \item
Suppose that we can factor a polynomial $f(x) \in F[x]$ into linear factors, say
$$
f(x) = a(x - a_1) (x - a_2) \cdots ( x - a_n).
$$
Prove that $f(x)$ has no repeated factors if and only if $f(x)$ and $f'(x)$ are relatively prime.
 
\end{enumerate}

\item
Let $F$ be a field. Show that $F[x]$ is never a field.

\item
Let $R$ be an integral domain.  Prove that $R[x_1, \ldots, x_n]$ is an integral domain.
 
\item
Let $R$ be a commutative ring with identity.  Show that $R[x]$ has a subring $R'$ isomorphic to $R$.
 
\item
Let $p(x)$ and $q(x)$ be polynomials in $R[x]$, where $R$ is a commutative ring with identity.  Prove that  $\deg( p(x) + q(x) ) \leq \max( \deg p(x), \deg q(x) )$. 

\end{enumerate}
}
 

\subsection*{Additional Exercises:  Solving the Cubic and Quartic \\
Equations}

{\small
 
\begin{enumerate}
 
\item
Solve the general quadratic equation 
$$
ax^2 + bx + c = 0
$$
to obtain
$$
x = \frac{-b \pm \sqrt{b^2 - 4ac}}{2a}.
$$
The {\bfi discriminant\/}\index{Discriminant!of the quadratic equation} of the quadratic equation $\Delta = b^2 - 4ac$ determines the nature of the solutions of the equation.  If $\Delta > 0$, the equation has two distinct real solutions.  If $\Delta = 0$, the equation has a single repeated real root. If $\Delta < 0$, there are two distinct imaginary solutions. 

\item
Show that any cubic equation of the form
$$
x^3 + bx^2 + cx + d = 0
$$
can be reduced to the form $y^3 + py + q = 0$ by making the substitution $x = y - b/3$.

\item
Prove that the cube roots of 1 are given by
\begin{eqnarray*}
\omega & = & \frac{-1+ i \sqrt{3}}{2} \\
\omega^2 & = & \frac{-1- i \sqrt{3}}{2} \\
\omega^3 & = & 1.
\end{eqnarray*}

\item
Make the substitution 
$$
y = z - \frac{p}{3 z}
$$
for $y$ in the equation $y^3 + py + q = 0$ and obtain two solutions $A$ and $B$ for~$z^3$.

\item
Show that the product of the solutions obtained in (4) is $-p^3/27$, deducing that $\sqrt[3]{A B} = -p/3$.

\item 
Prove that the possible solutions for $z$ in (4) are given by 
$$
\begin{array}{cccccc}
\sqrt[3]{A}, & \omega \sqrt[3]{A}, & \omega^2 \sqrt[3]{A}, &
\sqrt[3]{B}, & \omega \sqrt[3]{B}, & \omega^2 \sqrt[3]{B}
\end{array}
$$
and use this result to show that the three possible solutions for $y$
are 
$$
\omega^i \sqrt[3]{-\frac{q}{2}+ \sqrt{\frac{p^3}{27}  + \frac{q^2}{4}}  } +
\omega^i \sqrt[3]{-\frac{q}{2}- \sqrt{\frac{p^3}{27}  + \frac{q^2}{4}}  },
$$
where $i = 0, 1, 2$.
 
\item
The {\bfi discriminant\/}\index{Discriminant!of the cubic equation} of the cubic equation is  
$$
\Delta = \frac{p^3}{27}  + \frac{q^2}{4}.
$$
Show that $y^3 + py + q=0$ 
\begin{enumerate}
 
\item
has three real roots, at least two of which are equal, if $\Delta = 0$.
 
\item
has one real root and two conjugate imaginary roots if $\Delta > 0$.
 
\item
has three distinct real roots if $\Delta < 0$.
 
\end{enumerate}

\item
Solve the following cubic equations.
\begin{multicols}{2}
\begin{enumerate}

\item 
$x^3 - 4x^2 + 11 x + 30 = 0$

\item 
$x^3 - 3x +5 = 0$

\item 
$x^3 - 3x +2 = 0$

\item 
$x^3 + x + 3 = 0$

\end{enumerate}

\end{multicols} 
 
\item
Show that the general quartic equation
$$
x^4 + ax^3 + bx^2 + cx + d =0
$$
can be reduced to
$$
y^4 + py^2 + qy + r = 0
$$
by using the substitution $x = y - a/4$.
 
 
\item
Show that
$$
\left(
y^2 + \frac{1}{2} z
\right)^2 =
(z - p)y^2 - qy + 
\left(
\frac{1}{4} z^2 - r
\right).
$$ 
 
\item
Show that the right-hand side of (10) can be put in the form $(my + k)^2$ if and only if
$$
q^2 - 4(z - p)\left(
\frac{1}{4} z^2 - r
\right) = 0.
$$
 
\item
From (11) obtain the {\bfi resolvent cubic equation}\index{Resolvent cubic equation}
$$
z^3 - pz^2 - 4rz + (4pr - q^2) = 0.
$$
Solving the resolvent cubic equation, put the equation found in (10) in the form
$$
\left(
y^2 + \frac{1}{2} z
\right)^2 
=
(my + k)^2
$$ 
to obtain the solution of the quartic equation.

\item
Use this method to solve the following quartic equations.
\begin{multicols}{2}
\begin{enumerate}

\item 
$x^4 - x^2 - 3x + 2 = 0$

\item 
$x^4 +  x^3 - 7 x^2 - x + 6 = 0$

\item 
$x^4 -2 x^2 + 4 x -3 = 0$

\item 
$x^4 - 4 x^3 + 3x^2 - 5x +2 = 0$

\end{enumerate}
\end{multicols} 
 
\end{enumerate}
 
}
 
 
 
 
 
 
