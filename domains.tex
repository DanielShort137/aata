%%%%(c)
%%%%(c)  This file is a portion of the source for the textbook
%%%%(c)
%%%%(c)    Abstract Algebra: Theory and Applications
%%%%(c)    Copyright 1997 by Thomas W. Judson
%%%%(c)
%%%%(c)  See the file COPYING.txt for copying conditions
%%%%(c)
%%%%(c)
\chap{Integral Domains}{domains}
 
One of the most important rings we study is the ring of integers.  It was our first example of an algebraic structure: the first polynomial
ring that we examined was ${\mathbb Z}[x]$.  We also know that the integers sit naturally inside the field of rational numbers, ${\mathbb
Q}$.  The ring of integers is the model for all integral domains.  In this chapter we will examine integral domains in general, answering
questions about the ideal structure of integral domains, polynomial rings over integral domains, and whether or not an integral domain can
be embedded in a field.
 

\section{Fields of Fractions}

Every field is also an integral domain; however, there are many integral domains that are not fields.  For example, the integers ${\mathbb Z}$ are an integral domain but not a field.  A question that naturally arises is how we might associate an integral domain with a field.  There is a natural way to construct the rationals ${\mathbb Q}$ from the integers: the rationals can be represented as formal quotients of two integers.  The rational numbers are certainly a field.  In fact, it can be shown that the rationals are the smallest field that contains the integers.  Given an integral domain $D$, our question now becomes how to construct a smallest field $F$ containing $D$.  We will do this in the same way as we constructed the rationals from the integers.  

An element $p/q \in {\mathbb Q}$ is the quotient of two integers $p$ and $q$; however, different pairs of integers can represent the same
rational number.  For instance, $1/2 = 2/4 = 3/6$. We know that 
\[
\frac{a}{b} = \frac{c}{d}
\]
if and only if $ad = bc$. A more formal way of considering this problem is to examine fractions in terms of equivalence relations.  We can think of elements in ${\mathbb Q}$ as ordered pairs in ${\mathbb Z} \times {\mathbb Z}$.  A quotient $p/q$ can be written as $(p, q)$.  For instance, $(3, 7)$ would represent the fraction $3/7$.  However, there are problems if we consider all possible pairs in ${\mathbb Z} \times {\mathbb Z}$.  There is no fraction $5/0$ corresponding to the pair $(5,0)$.  Also, the pairs $(3,6)$ and $(2,4)$ both represent the fraction $1/2$.  The first problem is easily solved if we require the second coordinate to be nonzero.  The second problem is solved by considering two pairs $(a, b)$ and $(c, d)$ to be equivalent if $ad = bc$.

If we use the approach of ordered pairs instead of fractions, then we can study integral domains in general.  Let $D$ be any integral domain and let 
\[
S = \{ (a, b) : a, b \in D \mbox{ and } b \neq 0 \}.
\]
Define a relation on $S$ by $(a, b) \sim (c, d)$ if $ad=bc$.

\begin{lemma}
The relation $\sim$ between elements of $S$ is an equivalence relation. 
\end{lemma}
 
\begin{proof}
Since $D$ is commutative, $ab = ba$; hence, $\sim$ is reflexive on $D$.
Now suppose that $(a,b) \sim (c,d)$. Then $ad=bc$ or $cb = da$.
Therefore, $(c,d) \sim (a, b)$ and the relation is symmetric. Finally,
to show that the relation is transitive, let $(a, b) \sim (c, d)$ and
$(c, d) \sim (e,f)$. In this case $ad=bc$ and $cf = de$. Multiplying
both sides of $ad=bc$ by $f$ yields
\[
a f d = a d f = b c f = b d e = bed.
\]
Since $D$ is an integral domain, we can deduce that $af = be$ or 
$(a,b ) \sim (e, f)$. \mbox{\hspace{1in}}
\end{proof}
 

\medskip
 

We will denote the set of equivalence classes on $S$ by $F_D$. We now
need to define the operations of addition and multiplication on
$F_D$.  Recall how fractions are added and multiplied in ${\mathbb
Q}$:
\begin{align*}
\frac{a}{b} + \frac{c}{d} & = \frac{ad + b c}{b d}; \\
\frac{a}{b} \cdot \frac{c}{d} & = \frac{ac}{b d}.
\end{align*}
It seems reasonable to define the operations of addition and
multiplication on $F_D$ in a similar manner.  If we denote the
equivalence class of $(a, b) \in S$ by $[a, b]$, then we are led to
define the operations of addition and multiplication on $F_D$ by
\[
[a, b] + [c, d] = [ad + b c,b d]
\]
and
\[
[a, b] \cdot [c, d] = [ac, b d],
\]
respectively.  The next lemma demonstrates that these operations are
independent of the choice of representatives from each equivalence
class.
 
 
\begin{lemma}
The operations of addition and multiplication on $F_D$ are well-defined.
\end{lemma}
 

\begin{proof}
We will prove that the operation of addition is well-defined.  The
proof that multiplication is well-defined is left as an exercise.
Let $[a_1, b_1] = [a_2, b_2]$ and $[c_1, d_1] =[ c_2, d_2]$.  We must
show that
\[
[a_1 d_1 + b_1 c_1,b_1 d_1] = [a_2 d_2 + b_2 c_2,b_2 d_2]
\]
or, equivalently, that
\[
(a_1 d_1 + b_1 c_1)( b_2 d_2) = (b_1 d_1) (a_2 d_2 + b_2 c_2).
\]
Since  $[a_1, b_1] = [a_2, b_2]$ and $[c_1, d_1] =[ c_2, d_2]$, we
know that $a_1 b_2 = b_1 a_2$ and $c_1 d_2 = d_1 c_2$.  Therefore,
\begin{align*}
(a_1 d_1 + b_1 c_1)( b_2 d_2) 
& = 
a_1 d_1 b_2 d_2 + b_1 c_1 b_2 d_2 \\
& =
a_1 b_2 d_1 d_2 + b_1 b_2 c_1 d_2 \\
& =
b_1 a_2 d_1 d_2 + b_1 b_2 d_1 c_2 \\
& = 
(b_1 d_1) (a_2 d_2 + b_2 c_2).
\end{align*}
\end{proof}
 

\begin{lemma}\label{domains:field_fractions_lemma}
The set of equivalence classes of $S$, $F_D$, under the equivalence
relation $\sim$, together with the operations of addition and 
multiplication defined~by
\begin{align*}
[a, b] + [c, d] & = [ad + b c, b d] \\
{[ a, b]} \cdot [c, d] & = [ac, b d],
\end{align*}
is a field.
\end{lemma}
 

\begin{proof}
The additive and multiplicative identities are $[0,1]$ and $[1,1]$, 
respectively. To show that $[0,1]$ is the additive identity, observe 
that
\[
[a, b] + [0, 1] =  [ a 1 + b 0, b 1] = [a,b].
\]
It is easy to show that $[1, 1]$ is the multiplicative identity. Let
$[a, b] \in F_D$ such that $a \neq 0$. Then $[b, a]$ is also in $F_D$
and $[a,b] \cdot [b, a] = [1,1]$; hence, $[b, a]$ is the
multiplicative inverse for $[a, b]$.  Similarly, $[-a,b]$ is the
additive inverse of $[a, b]$.  We leave as exercises the verification
of the associative and  commutative properties of multiplication in
$F_D$. We also leave it to the reader to show that $F_D$ is an abelian
group under addition.  
 
 
It remains to show that the distributive property holds in $F_D$;
however, 
\begin{align*}
[a, b] [e, f] + [c, d][ e, f ] 
& = 
[a e, b f ] + [c e, d f] \\
& =
[a e d f + b f c e, b d f^2 ] \\
& =
[a e d + b c e, b d f ] \\
& =
[a d e + b c e, b d f ] \\
& =
( [a, b]  + [c, d] ) [ e, f ] 
\end{align*}
and the lemma is proved.
\end{proof} 	


\medskip

The field $F_D$ in Lemma~\ref{domains:field_fractions_lemma} is called the \boldemph{field of
fractions}\index{Field!of fractions} or \boldemph{field of
quotients}\index{Field!of quotients} of the integral domain $D$.  
 

\begin{theorem}\label{domains:field_of_quotients_ther}
Let $D$ be an integral domain.  Then $D$ can be embedded in a field of
fractions $F_D$, where any element in $F_D$ can be expressed as the
quotient of two elements in $D$.  Furthermore, the field of fractions
$F_D$ is unique in the sense that if $E$ is any field containing $D$,
then there exists a map $\psi : F_D \rightarrow E$ giving an isomorphism
with a subfield of $E$ such that $\psi(a) = a$ for all elements $a \in
D$. 
\end{theorem}
 

\begin{proof}
We will first demonstrate that $D$ can be embedded in the field 
$F_D$.  Define a map $\phi : D \rightarrow F_D$ by $\phi(a) 
= [a, 1]$.  Then for $a$ and $b$ in $D$,
\[
\phi( a + b ) = [a+b, 1] = [a, 1] + [b, 1] = \phi(a ) + \phi(b)
\]							       
and
\[
\phi( a b ) = [a b, 1] = [a, 1]  [b, 1] = \phi(a ) \phi(b);
\]
hence, $\phi$ is a homomorphism.  To show that $\phi$ is one-to-one,
suppose that $\phi(a) = \phi( b)$.  Then $[a, 1] = [b, 1]$, or $a = a1
= 1b = b$. Finally, any element of $F_D$ can be expressed as the quotient
of two elements in $D$, since   
\[
\phi(a) [\phi(b)]^{-1} = [a, 1] [b, 1]^{-1} = [a, 1] \cdot [1, b]
= [a, b].
\]

%Typo corrected.  Suggested by Nathan Lander.
%TWJ 2/13/2012

 
Now let $E$ be a field containing $D$ and define a map $\psi :F_D
\rightarrow E$ by $\psi([a, b]) = a b^{-1}$.  To show that $\psi$ is
well-defined, let $[a_1, b_1] = [a_2, b_2]$. Then $a_1 b_2 = b_1 a_2$.
Therefore, $a_1 b_1^{-1} = a_2 b_2^{-1}$  and $\psi( [a_1, b_1]) =
\psi( [a_2, b_2])$.
 

If $[a, b ]$ and $[c, d]$ are in $F_D$, then
\begin{align*}
\psi( [a, b] + [c, d] ) 
& = \psi( [ad + b c, b d ] ) \\
& =  (ad +b c)(b d)^{-1} \\
& = a b^{-1} + c d^{-1} \\
& = \psi( [a, b] ) + \psi( [c, d] )
\end{align*}
and
\begin{align*}
\psi( [a, b] \cdot [c, d] ) & = \psi( [ac, b d ] )\\
 & =  (ac)(b d)^{-1}\\
& = a b^{-1}  c d^{-1}\\
 & = \psi( [a, b] )  \psi( [c, d] ).
\end{align*}
Therefore, $\psi$ is a homomorphism.
 

To complete the proof of the theorem, we need to show that $\psi$ is
one-to-one.  Suppose that $\psi( [a, b] ) = ab^{-1} = 0$. Then $a =
0b = 0$ and $[a, b] = [0, b]$.  Therefore, the kernel of $\psi$ is
the zero element $[ 0, b]$ in $F_D$, and $\psi$ is injective.
\mbox{\hspace{1in}}
\end{proof}
 

\begin{example}{rational_polys}
Since ${\mathbb Q}$ is a field, ${\mathbb Q}[x]$ is an integral domain. The
field of fractions of ${\mathbb Q}[x]$ is the set of all rational
expressions $p(x)/q(x)$, where $p(x)$ and $q(x)$ are polynomials over
the rationals and $q(x)$ is not the zero polynomial. We will denote 
this field by ${\mathbb Q}(x)$.\label{noteratpoly} 
\end{example}

 

We will leave the proofs of the following corollaries of Theorem~\ref{domains:field_of_quotients_ther}
as exercises. 
 

\begin{corollary}\label{domains:char_zero_rationals_corollary}
Let $F$ be a field of characteristic zero. Then $F$ contains a
subfield isomorphic to ${\mathbb Q}$.	       
\end{corollary}


\begin{corollary}\label{domains:char_p_Zp_corollary}
Let $F$ be a field of characteristic $p$. Then $F$ contains a
subfield isomorphic to ${\mathbb Z}_p$.	       
\end{corollary}
 


\section{Factorization in Integral Domains}

 
The building blocks of the integers are the prime numbers.  If $F$ is
a field, then irreducible polynomials in $F[x]$ play a role that is
very similar to that of the prime numbers in the ring of integers.
Given an arbitrary integral domain, we are led to the following
series of definitions. 
 

Let $R$ be a commutative ring with identity, and let $a$ and $b$ be
elements in $R$.  We say that $a$ \boldemph{divides} $b$, and write $a \mid
b$, if there exists an element $c \in R$ such that $b = ac$.  A \boldemph{
unit}\index{Unit} in $R$ is an element that has a multiplicative
inverse.  Two elements $a$ and $b$ in $R$ are said to be \boldemph{
associates}\index{Element!associate}\index{Associate elements} if
there exists a unit $u$ in $R$ such that $a = ub$.   
 

Let $D$ be an integral domain.  A nonzero element $p \in D$ that is
not a unit is said to be \boldemph{
irreducible}\index{Element!irreducible}\index{Irreducible element}
provided that whenever $p = ab$, either $a$ or $b$ is a unit.
Furthermore, $p$ is \boldemph{
prime}\index{Prime element}\index{Element!prime} if whenever $p \mid
ab$ either $p \mid a$ or $p \mid b$.


\begin{example}{Qxy}
It is important to notice that prime and irreducible elements do not
always coincide. Let $R$ be the subring (with identity) of ${\mathbb Q}[x, y]$ generated
by $x^2$, $y^2$, and $xy$.  Each of these elements is irreducible in
$R$; however, $xy$ is not prime, since $xy$ divides $x^2 y^2$ but does
not divide either $x^2$ or $y^2$.
\end{example}

%We are assuming that R is an integral domain and contains 1.  TWJ 8/9/2012

 
 
The Fundamental Theorem of Arithmetic states that every positive
integer $n >1$ can be factored into a product of prime numbers $p_1
\cdots p_k$, where the $p_i$'s are not necessarily distinct. We also
know that such factorizations are unique up to the order of the
$p_i$'s. We can easily extend this result to the integers. The
question arises of whether or not such factorizations are possible in
other rings.  Generalizing this definition, we say an integral domain
$D$ is a \boldemph{unique factorization domain},\index{Unique
factorization domain (UFD)}\index{Domain!unique factorization} or \boldemph{
UFD}, if $D$ satisfies the following criteria.  
\begin{enumerate}
 
\item
Let $a \in D$ such that $a \neq 0$ and $a$ is not a unit.
Then $a$ can be written as the product of irreducible
elements in $D$.
 
\item %Definition corrected thanks to Pascal Honore - TWJ 9/10/2010
Let $a = p_1 \cdots p_r = q_1 \cdots q_s$, where the $p_i$'s and the
$q_i$'s are irreducible. Then $r=s$ and there is a $\pi \in S_r$ such
that $p_i$ and $q_{\pi(j)}$ are associates for \mbox{$j = 1, \ldots, r$}. 
 
\end{enumerate}


\begin{example}{Z_UFD}
The integers are a unique factorization domain by the Fundamental
Theorem of Arithmetic.
\end{example}
 


\begin{example}{not_a_UFD}
Not every integral domain is a unique factorization domain. The
subring ${\mathbb Z}[ \sqrt{3}\, i ] = \{ a + b \sqrt{3}\, i\}$ of the
complex numbers is an integral domain (Exercise~\ref{rings:gaussian_exercise}, Chapter~\ref{rings}). Let
$z = a + b \sqrt{3}\, i$ and define \mbox{$\nu : {\mathbb Z}[ \sqrt{3}\, i ]
\rightarrow {\mathbb N} \cup \{ 0 \}$} by $\nu( z) = |z|^2 = a^2 + 3 b^2$.
It is clear that $\nu(z) \geq 0$ with equality when $z = 0$. Also,
from our knowledge of complex numbers we know that $\nu(z w) = \nu(z)
\nu(w)$. It is easy to show that if $\nu(z) = 1$, then $z$ is a unit,
and that the only units of ${\mathbb Z}[ \sqrt{3}\, i ]$ are 1 and $-1$.   
 

We claim that 4 has two distinct factorizations into irreducible
elements: 
\[
4 = 2 \cdot 2 = (1 - \sqrt{3}\, i) (1 + \sqrt{3}\, i).
\]
We must show that each of these factors is an irreducible element in
${\mathbb Z}[ \sqrt{3}\, i ]$. If 2 is not irreducible, then $2 = z w$ for
elements $z, w$ in ${\mathbb Z}[ \sqrt{3}\, i ]$ where $\nu( z) = \nu(w) =
2$. However, there does not exist an element in $z$ in ${\mathbb
Z}[\sqrt{3}\, i ]$ such that $\nu(z) = 2$ because the equation $a^2 + 3
b^2 = 2$ has no integer solutions. Therefore, 2 must be irreducible. A
similar argument shows that both $1 - \sqrt{3}\, i$ and $1 + \sqrt{3}\, i$
are irreducible. Since 2 is not a unit multiple of either $1 - 
\sqrt{3}\, i$ or $1 + \sqrt{3}\, i$, 4 has at least two distinct
factorizations into irreducible elements.
\mbox{\hspace{1in}}
\end{example}
 

 
\subsection*{Principal Ideal Domains}
 

Let $R$ be a commutative ring with identity. Recall that a principal
ideal generated by $a \in R$ is an ideal of the form $\langle a
\rangle = \{ ra : r \in R \}$. An integral domain in which every ideal
is principal is called a \boldemph{principal ideal domain},\index{Principal
ideal domain (PID)}\index{Domain!principal ideal} or  \boldemph{PID}. 
 

\begin{lemma}\label{domains:PI_lemma}
Let $D$ be an integral domain and let $a, b \in D$.
Then  
\begin{enumerate}

\rm\item\it
$a \mid b \Leftrightarrow \langle b \rangle \subset \langle a
\rangle$. 

\rm\item\it
$a$ and $b$ are associates $\Leftrightarrow$ $\langle b \rangle =
\langle a \rangle$.

\rm\item\it
$a$ is a unit in $D$ $\Leftrightarrow$ $\langle a \rangle = D$.

\end{enumerate} 
\end{lemma}


\begin{proof}
(1)
Suppose that $a \mid b$. Then $b = ax$ for some $x \in D$. Hence, for
every $r$ in $D$, $br =(ax)r = a(xr)$ and $\langle b \rangle \subset
\langle a \rangle$. Conversely, suppose that $\langle b \rangle
\subset \langle a \rangle$. Then $b \in \langle a \rangle$.
Consequently, $b =a x$ for some $x \in D$.  Thus, $a \mid b$.
 

(2)
Since $a$ and $b$ are associates, there exists a unit $u$ such that 
$a = u b$.  Therefore, $b \mid a$ and $\langle a \rangle \subset 
\langle b \rangle$. Similarly, $\langle b \rangle \subset \langle a 
\rangle$. It follows that $\langle a \rangle = \langle b \rangle$.
Conversely, suppose that $\langle a \rangle = \langle b
\rangle$. By part (1), $a \mid b$ and $b \mid a$. Then $a = bx$ and $b
= ay$ for some $x, y \in D$. Therefore, $a = bx = ayx$. Since $D$ is
an integral domain, $x y =1$; that is, $x$ and $y$ are units and $a$
and $b$ are associates.   


(3)
An element $a \in D$ is a unit if and only if $a$ is an associate of
1. However, $a$ is an associate of 1 if and only if $\langle a \rangle
= \langle 1 \rangle = D$. 
\end{proof}


\begin{theorem}
Let $D$ be a PID and $\langle p \rangle$ be a nonzero ideal in $D$. 
Then $\langle p \rangle$ is a maximal ideal if and only if $p$ is
irreducible.
\end{theorem}


\begin{proof}
Suppose that $\langle p \rangle$ is a maximal ideal.  If some element
$a$ in $D$ divides $p$, then $\langle p \rangle \subset \langle a
\rangle$.  Since $\langle p \rangle$ is maximal, either $D = \langle
a \rangle$ or $\langle p \rangle = \langle a \rangle$.  Consequently,
either $a$ and $p$ are associates or $a$ is a unit.  Therefore, $p$
is irreducible.


Conversely, let $p$ be irreducible. If $\langle a \rangle$ is an
ideal in $D$ such that  $\langle p \rangle \subset \langle a \rangle
\subset D$, then $a \mid p$. Since $p$ is irreducible, either $a$ must
be a unit or $a$ and $p$ are associates. Therefore, either $D =
\langle a \rangle$ or $\langle p \rangle = \langle a \rangle$.  Thus,
$\langle p \rangle$ is a maximal ideal.
\end{proof}


\begin{corollary}\label{domains:irredPIDcorollary}
Let $D$ be a PID. If $p$ is irreducible, then $p$ is prime.
\end{corollary}


\begin{proof}
Let $p$ be irreducible and suppose that $p \mid ab$.  Then $\langle
ab \rangle \subset \langle p \rangle$. By Corollary~\ref{rings:max_ideal_corollary}, since
$\langle p \rangle$ is a maximal ideal, $\langle p \rangle$ must also
be a prime ideal. Thus, either $a \in \langle p \rangle$ or $b \in
\langle p \rangle$.  Hence, either $p \mid a $ or $p \mid b$. 
\end{proof}


\begin{lemma}\label{domains:ACC_lemma}
Let $D$ be a PID.  Let $I_1, I_2, \ldots$ be a set of ideals such that 
$I_1 \subset I_2 \subset \cdots$. Then there exists an integer $N$
such that $I_n = I_N$ for all $n \geq N$.
\end{lemma}

 

\begin{proof}
We claim that $I= \bigcup_{i=1}^\infty I_i$ is an ideal of $D$. Certainly
$I$ is not empty, since $I_1 \subset I$ and $0 \in I$. If $a, b \in I$,
then $a \in I_i$ and $b \in I_j$ for some $i$ and $j$ in ${\mathbb N}$.
Without loss of generality we can assume that $i \leq j$.  Hence, $a$
and $b$ are both in $I_j$ and so $a - b$ is also in $I_j$. Now let $r
\in D$ and $a \in I$. Again, we note that $a \in I_i$ for some
positive integer $i$.  Since $I_i$ is an ideal, $ra \in I_i$ and hence
must be in $I$. Therefore, we have shown that $I$ is an ideal in $D$.

%Typo corrected.  Suggested by C. Wall.  TWJ 2/16/2012


Since $D$ is a principal ideal domain, there exists an element 
$\overline{a} \in D$ that generates $I$. Since $\overline{a}$ is in 
$I_N$ for some $N \in {\mathbb N}$, we know that $I_N = I = \langle 
\overline{a} \rangle$. Consequently, $I_n = I_N$ for $n \geq N$.
\end{proof}		    
 

\medskip


Any commutative ring satisfying the condition in Lemma~\ref{domains:ACC_lemma} is said
to satisfy the \boldemph{ascending chain condition},\index{Ascending chain
condition} or \boldemph{ACC}.  Such rings are called \boldemph{
Noetherian rings},\index{Ring!Noetherian} after Emmy Noether.  


\begin{theorem}
Every PID is a UFD.
\end{theorem}
 

\begin{proof}
\emph{Existence of a factorization.}
Let $D$ be a PID and $a$ be a nonzero element in $D$ that is not a
unit. If $a$ is irreducible, then we are done. If not, then there
exists a factorization $a = a_1 b_1$, where neither $a_1$ nor $b_1$ is
a unit. Hence, $\langle a \rangle \subset \langle a_1 \rangle$. By
Lemma~\ref{domains:PI_lemma}, we know that $\langle a \rangle \neq \langle a_1 \rangle$;
otherwise, $a$ and $a_1$ would be associates and $b_1$ would be a unit,
which would contradict our assumption. Now suppose that $a_1 =  a_2
b_2$, where neither $a_2$ nor $b_2$ is a unit. By the same argument as
before, $\langle a_1 \rangle \subset \langle a_2 \rangle$.  We can
continue with this construction to obtain an ascending chain of ideals
\[
\langle a \rangle \subset \langle a_1 \rangle \subset \langle a_2
\rangle \subset \cdots.
\]
By Lemma~\ref{domains:ACC_lemma}, there exists a positive integer $N$ such that
$\langle a_n \rangle = \langle a_N \rangle$ for all $n \geq N$. 
Consequently, $a_N$ must be irreducible. We have now shown that $a$ is
the product of two elements, one of which must be irreducible.  

%Label repaired.  Suggested by R. Beezer.
%TWJ 12/19/2011


Now suppose that $a = c_1 p_1$, where $p_1$ is irreducible. If $c_1$
is not a unit, we can repeat the preceding argument to conclude that
$\langle a \rangle \subset \langle c_1 \rangle$. Either $c_1$ is
irreducible or $c_1 = c_2 p_2$, where $p_2$ is irreducible and $c_2$
is not a unit.  Continuing in this manner, we obtain another chain of
ideals
\[
\langle a \rangle \subset \langle c_1 \rangle \subset \langle c_2
\rangle \subset \cdots.	   
\]
This chain must satisfy the ascending chain condition; therefore, 
\[
a = p_1 p_2 \cdots p_r
\]
for irreducible elements $p_1, \ldots, p_r$.


\emph{Uniqueness of the factorization.}
To show uniqueness, let
\[
a= p_1 p_2 \cdots p_r = q_1 q_2 \cdots q_s,
\]
where each $p_i$ and each $q_i$ is irreducible.  Without loss of
generality, we can assume that $r < s$. Since $p_1$ divides $q_1 q_2
\cdots q_s$, by Corollary~\ref{domains:irredPIDcorollary} it must divide some $q_i$. By
rearranging the $q_i$'s, we can assume that $p_1 \mid q_1$; hence,
$q_1 = u_1 p_1$ for some unit $u_1$ in $D$. Therefore,
\[
a = p_1 p_2 \cdots p_r = u_1 p_1 q_2 \cdots q_s
\]
or 
\[
p_2 \cdots p_r = u_1 q_2 \cdots q_s.
\]
Continuing in this manner, we can arrange the $q_i$'s such that $p_2
= q_2, p_3 = q_3, \ldots, p_r = q_r$, to obtain
\[
u_1 u_2 \cdots u_r q_{r+1} \cdots q_s = 1.
\]
In this case $q_{r+1} \cdots q_s$ is a unit, which contradicts the
fact that $q_{r+1}, \ldots, q_s$ are irreducibles. Therefore, $r=s$
and the factorization of $a$ is unique.
\end{proof}
%Label repaired.  Suggested by R. Beezer.
%TWJ - 12/19/2011

\begin{corollary}
Let $F$ be a field.  Then $F[x]$ is a UFD.
\end{corollary}

\begin{example}{Zx_UFD}
Every PID is a UFD, but it is not the case that every UFD is a PID. In Corollary~\ref{domains:int_poly_UFD_corollary}, we will prove that ${\mathbb Z}[x]$ is a UFD.  However,
${\mathbb Z}[x]$ is not a PID.  Let $I = \{ 5 f(x) + x g(x) : f(x), g(x)  \in {\mathbb Z}[x] \}$.  We can easily show that $I$ is an ideal of 
${\mathbb Z}[x]$.  Suppose that $I = \langle p(x) \rangle$.  Since $5 \in I$,  $5 = f(x) p(x)$.  In this case $p(x) = p$ must be a constant.  Since $x  \in I$, $x = p g(x)$; consequently, $p = \pm 1$. However, it follows from this fact that $\langle p(x) \rangle = {\mathbb Z}[x]$. But this would 
mean that 3 is in $I$. Therefore, we can write $3 = 5 f(x) + x g(x)$ for  some $f(x)$ and $g(x)$ in ${\mathbb Z}[x]$.  Examining the constant term of  this polynomial, we see that $3 = 5 f(x)$, which is impossible. 
\end{example}


\subsection*{Euclidean Domains}

We have repeatedly used the division algorithm when proving results about either ${\mathbb Z}$ or $F[x]$, where $F$ is a field.  We
should now ask when a division algorithm is available for an integral domain. 

Let $D$ be an integral domain such that for each $a \in D$ there is  a nonnegative integer $\nu(a)$\label{notevaluation} satisfying the
following conditions. 
\begin{enumerate}
 
\item
If $a$ and $b$ are nonzero elements in $D$, then $\nu(a) \leq \nu(ab)$.  
 
\item
Let $a, b \in D$ and suppose that $b \neq 0$. Then there exist elements $q, r \in D$ such that $a = bq +r$ and either $r=0$ or
$\nu(r) < \nu(b)$. 
 
\end{enumerate}
Then $D$ is called a \boldemph{Euclidean domain}\index{Euclidean domain}\index{Domain!Euclidean} and $\nu$ is called a \boldemph{Euclidean valuation}\index{Euclidean valuation}.  


\begin{example}{abs_value}
Absolute value on ${\mathbb Z}$ is a Euclidean valuation.
\end{example}


\begin{example}{Fx_Euc_val}
Let $F$ be a field. Then the degree of a polynomial in $F[x]$ is a Euclidean valuation. 
\end{example}


\begin{example}{Gaussian_valuation}
Recall that the Gaussian integers in Example~\ref{example:rings:gaussian_ring} of Chapter~\ref{rings} are defined by
\[
{\mathbb Z}[i] = \{  a + b i : a, b \in {\mathbb Z} \}.
\]
We usually measure the size of a complex number $a + bi$ by its absolute value, $|a + bi| = \sqrt{ a^2 + b^2}$; however, $\sqrt{a^2 + b^2}$ may not be an integer. For our valuation we will let $\nu(a + bi) = a^2 + b^2$ to ensure that we have an integer.  

We claim that $\nu( a+ bi) = a^2 + b^2$ is a Euclidean valuation on ${\mathbb Z}[i]$. Let $z, w \in {\mathbb Z}[i]$.  Then $\nu( zw) = |zw|^2 =
|z|^2 |w|^2  = \nu(z) \nu(w)$.  Since $\nu(z) \geq 1$ for every nonzero $z \in {\mathbb Z}[i]$, $\nu( z) \leq \nu(z) \nu(w)$.

%correction to last sentence.  Suggested by R. Beezer.
%TWJ 2/6/2013

Next, we must show that for any $z= a+bi$ and $w = c+di$ in ${\mathbb Z}[i]$ with $w \neq 0$, there exist elements $q$ and $r$ in 
${\mathbb Z}[i]$  such that $z = qw + r$ with either $r=0$ or  $\nu(r) < \nu(w)$.  We can view $z$ and $w$ as elements in ${\mathbb
Q}(i) = \{ p + qi : p, q \in {\mathbb Q} \}$, the field of fractions of ${\mathbb Z}[i]$.  Observe that
\begin{align*}
z w^{-1} & = (a +b i) \frac{c -d i}{c^2 + d^2} \\
& =
\frac{ac + b d}{c^2 + d^2} + \frac{b c -ad}{c^2 + d^2}i \\
& =
\left( 
m_1 + \frac{n_1}{c^2 + d^2}
\right)
+ 
\left(
m_2 + \frac{n_2}{c^2 + d^2}
\right) i \\
& =
(m_1 + m_2 i) + \left( 
\frac{n_1}{c^2 + d^2}
+ 
\frac{n_2}{c^2 + d^2}i
\right) \\
& =
(m_1 + m_2 i) + (s + ti)
\end{align*}
in ${\mathbb Q}(i)$.  In the last steps we are writing the real and imaginary parts as an integer plus a proper fraction.  That is, we take
the closest integer $m_i$ such that the fractional part satisfies $|n_i / (a^2 + b^2)| \leq 1/2$.  For example, we write 
\begin{align*}
\frac{9}{8} & = 1 + \frac{1}{8} \\
\frac{15}{8} & = 2 - \frac{1}{8}.
\end{align*}
Thus, $s$ and $t$ are the ``fractional parts'' of $z w^{-1} = (m_1 + m_2 i) + (s + ti)$. We also know that $s^2 + t^2 \leq 1/4 + 1/4 =
1/2$.  Multiplying by $w$, we have
\[
z = z w^{-1} w = w (m_1 + m_2 i) + w (s + ti)  = q w + r,
\]
where $q = m_1 + m_2 i$ and $r =  w (s + ti)$.  Since $z$ and $qw$ are in ${\mathbb Z}[i]$, $r$ must be in ${\mathbb Z}[i]$.  Finally, we need to show that either $r = 0$ or $\nu(r) < \nu(w)$.  However,
\[
\nu(r) = \nu(w) \nu(s + ti) \leq \frac{1}{2} \nu(w) < \nu(w).
\]
\end{example}

\begin{theorem}
Every Euclidean domain is a principal ideal domain.
\end{theorem}
 
\begin{proof}
Let $D$ be a Euclidean domain and let $\nu$ be a Euclidean valuation
on $D$.  Suppose $I$ is a nontrivial ideal in $D$ and choose a nonzero
element $b \in I$ such that $\nu(b)$ is minimal for all $a \in I$. 
Since $D$ is a Euclidean domain, there exist elements $q$ and $r$ in
$D$ such that $a = bq +r$ and either $r=0$ or $\nu(r) < \nu(b)$. But
$r = a - bq$ is in $I$ since $I$ is an ideal; therefore, $r = 0$ by the
minimality of $b$. It follows that $a = bq$ and $I = \langle b
\rangle$. 
\end{proof}
 

\begin{corollary}
Every Euclidean domain is a unique factorization domain.
\end{corollary}


\subsection*{Factorization in $D[x]$}

One of the most important polynomial rings is ${\mathbb Z}[x]$.  One of the first questions that come to mind about ${\mathbb Z}[x]$ is whether or not it is a UFD.  We will prove a more general statement here.  Our first task is to obtain a more general version of Gauss's Lemma
(Theorem~\ref{poly:Gauss_lemma}).  

Let $D$ be a unique factorization domain and suppose that 
\[
p(x) = a_n x^n + \cdots + a_1 x + a_0
\]
in $D[x]$.  Then the \boldemph{content}\index{Polynomial!content of} of $p(x)$ is the greatest common divisor of $a_0, \ldots, a_n$.  We say
that $p(x)$ is \boldemph{primitive}\index{Primitive polynomial}\index{Polynomial!primitive} if $\gcd(a_0, \ldots, a_n ) = 1$.    

%Typo corrected.  Suggested by Nathan Lander.
%TWJ 2/13/2012

 
\begin{example}{poly_gcd}
In ${\mathbb Z}[x]$ the polynomial $p(x)= 5 x^4 - 3 x^3 + x -4$ is a primitive polynomial since the greatest common divisor of the
coefficients is 1; however, the polynomial $q(x) = 4 x^2 - 6 x + 8$ is not primitive since the content of $q(x)$ is 2.
\end{example}
 
\begin{theorem}[Gauss's Lemma]\index{Gauss's Lemma}\label{domains:Gauss_lemma}
Let $D$ be a UFD and let $f(x)$ and $g(x)$ be primitive polynomials in $D[x]$.  Then $f(x) g(x)$ is primitive.
\end{theorem}
 
\begin{proof}
Let $f(x) = \sum_{i=0}^{m} a_i x^i$ and $g(x) = \sum_{i=0}^{n} b_i x^i$.  Suppose that $p$ is a prime dividing the coefficients of $f(x)
g(x)$.  Let $r$ be the smallest integer such that $p \notdivide a_r$ and $s$ be the smallest integer such that $p \notdivide b_s$.  The
coefficient of $x^{r+s}$ in $f(x) g(x)$ is 
\[
c_{r+s} = a_0 b_{r+s} + a_1 b_{r+s-1} + \cdots + a_{r+s-1} b_1 +
a_{r+s} b_0. 
\]
Since $p$ divides $a_0, \ldots, a_{r-1}$ and $b_0, \ldots, b_{s-1}$, $p$ divides every term of $c_{r+s}$ except for the term $a_r b_s$.  However, since $p \mid c_{r+s}$, either $p$ divides $a_r$ or $p$ divides $b_s$. But this is impossible.
\end{proof}
 

\begin{lemma}
Let $D$ be a UFD, and let $p(x)$ and $q(x)$ be in $D[x]$. Then the
content of $p(x) q(x)$ is equal to the product of the contents of
$p(x)$ and~$q(x)$.
\end{lemma}

\begin{proof}
Let $p(x) = c p_1(x)$ and $q(x) = d q_1(x)$, where $c$ and $d$ are the
contents of $p(x)$ and $q(x)$, respectively.  Then $p_1(x)$ and
$q_1(x)$ are primitive. We can now write $p(x) q(x) = c d p_1(x)
q_1(x)$. Since $p_1(x) q_1(x)$ is primitive, the content of $p(x)
q(x)$ must be $cd$.
\end{proof}

\begin{lemma}\label{domains:UFD_factor_lemma}
Let $D$ be a UFD and $F$ its field of fractions. Suppose that $p(x)
\in D[x]$ and $p(x) = f(x) g(x)$, where $f(x)$ and $g(x)$ are in 
$F[x]$. Then $p(x) = f_1(x) g_1(x)$, where $f_1(x)$ and $g_1(x)$ are in
$D[x]$.  Furthermore, $\deg f(x) = \deg f_1(x)$ and $\deg g(x) = \deg
g_1(x)$. 
\end{lemma}

\begin{proof}
Let $a$ and $b$ be nonzero elements of $D$ such that $a f(x), b g(x)$
are in $D[x]$. We can find $a_1, b_2 \in D$ such that $a f(x) = a_1
f_1(x)$ and $b g(x) = b_1 g_1(x)$, where $f_1(x)$ and $g_1(x)$ are
primitive polynomials in $D[x]$. Therefore, $a b p(x) = (a_1 f_1(x))(
b_1 g_1(x))$.  Since $f_1(x)$ and $g_1(x)$ are primitive polynomials,
it must be the case that $ab \mid a_1 b_1$ by Gauss's Lemma. Thus there
exists a $c \in D$ such that $p(x) = c f_1(x) g_1(x)$. Clearly, $\deg
f(x) = \deg f_1(x)$ and $\deg g(x) = \deg g_1(x)$. 
\end{proof}

\medskip

The following corollaries are direct consequences of Lemma~\ref{domains:UFD_factor_lemma}. 

\begin{corollary}\label{domains:irred_poly_lemma}
Let $D$ be a UFD and $F$ its field of fractions.  A primitive
polynomial $p(x)$ in $D[x]$ is irreducible in $F[x]$ if and only if it
is irreducible in $D[x]$.
\end{corollary}

\begin{corollary}
Let $D$ be a UFD and $F$ its field of fractions.  If $p(x)$ is a monic
polynomial in $D[x]$ with $p(x) = f(x) g(x)$ in $F[x]$, then $p(x) =
f_1(x) g_1(x)$, where $f_1(x)$ and $g_1(x)$ are in $D[x]$. Furthermore,
$\deg f(x) = \deg f_1(x)$ and $\deg g(x) = \deg g_1(x)$.
\end{corollary}


\begin{theorem}
If $D$ is a UFD, then $D[x]$ is a UFD.
\end{theorem}
 
 
\begin{proof}
Let $p(x)$ be a nonzero polynomial in $D[x]$.  If $p(x)$ is a constant
polynomial, then it must have a unique factorization since $D$ is a
UFD. Now suppose that $p(x)$ is a polynomial of positive degree in
$D[x]$. Let $F$ be the field of fractions of $D$, and let $p(x) =
f_1(x) f_2(x) \cdots f_n(x)$ by a factorization of $p(x)$, where each
$f_i(x)$ is irreducible. Choose $a_i \in D$ such that $a_i f_i(x)$ is in
$D[x]$. There exist $b_1, \ldots, b_n \in D$ such that $a_i f_i(x) =
b_i g_i(x)$, where $g_i(x)$ is a primitive polynomial in $D[x]$. By 
Corollary \ref{domains:irred_poly_lemma}, each $g_i(x)$ is irreducible in $D[x]$. Consequently, 
we can write
\[
a_1 \cdots a_n p(x) = b_1 \cdots b_n g_1(x) \cdots g_n(x).
\]
Let $b = b_1 \cdots b_n$. Since $g_1(x) \cdots g_n(x)$ is primitive,
$a_1 \cdots a_n$ divides $b$. Therefore, $p(x) = a g_1(x) \cdots
g_n(x)$, where $a \in D$. Since $D$ is a UFD, we can factor $a$ as $u
c_1 \cdots c_k$, where $u$ is a unit and each of the $c_i$'s is
irreducible in $D$. 


We will now show the uniqueness of this factorization. Let
\[
p(x) = a_1 \cdots a_m f_1(x) \cdots f_n(x) = b_1 \cdots b_r g_1(x)
\cdots g_s(x)
\]
be two factorizations of $p(x)$, where all of the factors are
irreducible in $D[x]$.  By Corollary~\ref{domains:irred_poly_lemma}, each of the $f_i$'s and
$g_i$'s is irreducible in $F[x]$. The $a_i$'s and the $b_i$'s are
units in $F$. Since $F[x]$ is a PID, it is a UFD; therefore, $n=s$.
Now rearrange the $g_i(x)$'s so that $f_i(x)$ and $g_i(x)$ are
associates for $i = 1, \ldots, n$.  Then there exist $c_1, \ldots,
c_n$ and $d_1, \ldots, d_n$ in $D$ such that $(c_i / d_i) f_i(x) =
g_i(x)$ or $c_i f_i(x) = d_i g_i(x)$. The polynomials $f_i(x)$ and
$g_i(x)$ are primitive; hence, $c_i$ and $d_i$ are associates in $D$.
Thus, $a_1 \cdots a_m = u b_1 \cdots b_r$ in $D$, where $u$ is a unit
in $D$. Since $D$ is a unique factorization domain, $m =s$. Finally,
we can reorder the $b_i$'s so that $a_i$ and $b_i$ are associates for
each $i$. This completes the uniqueness part of the proof.   
\end{proof}


\medskip


The theorem that we have just proven has several obvious but important 
corollaries. 
 

\begin{corollary}
Let $F$ be a field. Then $F[x]$ is a UFD.
\end{corollary}


\begin{corollary}\label{domains:int_poly_UFD_corollary}
${\mathbb Z}[x]$ is a UFD.
\end{corollary}


\begin{corollary}
Let $D$ be a UFD. Then $D[x_1, \ldots, x_n]$ is a UFD. 
\end{corollary} 



\noindent \textbf{Remark.}
It is important to notice that every Euclidean domain is a PID and
every PID is a UFD. However, as demonstrated by our examples, the
converse of each of these statements fails.  There are principal
ideal domains that are not Euclidean domains, and there are unique
factorization domains that are not principal ideal domains (${\mathbb
Z}[x]$). 

\histhead

\noindent
Karl Friedrich Gauss\index{Gauss, Karl Friedrich}, born in Brunswick, Germany on April 30, 1777, is considered to be one of the greatest
mathematicians who ever lived.  Gauss was truly a child prodigy.  At the age of three he was able to detect errors in the books of his father's
business.  Gauss entered college at the age of 15.  Before the age of 20, Gauss was able to construct a regular 17-sided polygon with a ruler
and compass.  This was the first new construction of a regular $n$-sided polygon since the time of the ancient Greeks.  Gauss succeeded in showing that if $N= 2^{2^n}+1$ was prime, then it was possible to construct a regular $N$-sided polygon.  

Gauss obtained his Ph.D. in 1799 under the direction of Pfaff at the University of Helmstedt.  In his dissertation he gave the first complete proof of the Fundamental Theorem of Algebra, which states that every polynomial with real coefficients can be factored into linear factors over the complex numbers.  The acceptance of complex numbers was brought about by Gauss, who was the first person to use the notation of $i$ for $\sqrt{-1}$. 

Gauss then turned his attention toward number theory; in 1801, he published his famous book on number theory, \textit{Disquisitiones
Arithmeticae}.  Throughout his life Gauss was intrigued with this branch of mathematics.  He once wrote, ``Mathematics is the queen of the sciences, and the theory of numbers is the queen of mathematics.'' 

In 1807, Gauss was appointed director of the Observatory at the University of G\"{o}ttingen, a position he held until his death.  This position required him to study applications of mathematics to the sciences.  He succeeded in making contributions to fields such as astronomy, mechanics, optics, geodesy, and magnetism.  Along with Wilhelm Weber, he coinvented the first practical electric telegraph some years before a better version was invented by Samuel F. B. Morse. 

Gauss was clearly the most prominent mathematician in the world in the early nineteenth century. His status naturally made his discoveries
subject to intense scrutiny.  Gauss's  cold and distant personality many times led him to ignore the work of his contemporaries, making him many enemies.  He did not enjoy teaching very much, and young mathematicians who sought him out for encouragement were often
rebuffed.  Nevertheless, he had many outstanding students, including Eisenstein, Riemann, Kummer, Dirichlet, and Dedekind. Gauss also
offered a great deal of encouragement to Sophie Germain (1776--1831), who overcame the many obstacles facing women in her day to become a very prominent mathematician.  Gauss died at the age of 78 in G\"{o}ttingen on February 23, 1855.
\histbox

 
\markright{EXERCISES}
\section*{Exercises}
\exrule

{\small
\begin{enumerate}

% 2010/05/18 R Beezer, added squared-power on $a$ in valuation
\item
Let $z = a + b \sqrt{3}\, i$ be in ${\mathbb Z}[ \sqrt{3}\, i]$. If $a^2 +3 b^2 = 1$, show that $z$ must be a unit. Show that the only units of ${\mathbb Z}[ \sqrt{3}\, i ]$ are 1 and $-1$. 

\item
The Gaussian integers, ${\mathbb Z}[i]$, are a UFD.  Factor each of the following elements in ${\mathbb Z}[i]$ into a product of irreducibles.
\begin{multicols}{2}
\begin{enumerate}

\item 
5

\item 
$1 + 3i$

\item 
$6+8i$

\item 
2

\end{enumerate}
\end{multicols}

%*****************Theory---Fields of fractions****************
 
\item
Let $D$ be an integral domain. 
\begin{enumerate}
 
 \item
Prove that $F_D$ is an abelian group under the operation of addition.

 \item
Show that the operation of multiplication is well-defined in the field
of fractions, $F_D$.  
 
 \item
Verify the associative and commutative properties for multiplication
in $F_D$. 

\end{enumerate}


\item
Prove or disprove: Any subring of a field $F$ containing $1$ is an
integral domain.

\item
Prove or disprove: If $D$ is an integral domain, then every prime element in $D$ is also irreducible in $D$.

%Exercise suggested by R. Beezer. TWJ 5/15/2012


 
\item
Let $F$ be a field of characteristic zero. Prove that $F$ contains a
subfield isomorphic to ${\mathbb Q}$.
 
\item
Let $F$ be a field.
\begin{enumerate}

 \item
Prove that the field of fractions of $F[x]$, denoted by
$F(x)$\label{noteratfun},  is isomorphic to the set all rational
expressions $p(x) / q(x)$, where $q(x)$ is not the zero polynomial. 

 \item
Let $p(x_1, \ldots, x_n)$ and $q(x_1, \ldots, x_n)$ be polynomials in
$F[x_1, \ldots, x_n]$. Show that the set of all rational expressions
$p(x_1, \ldots, x_n) / q(x_1, \ldots, x_n)$ is isomorphic to the field
of fractions of $F[x_1, \ldots, x_n]$.  We denote the field of
fractions of $F[x_1, \ldots, x_n]$ by $F(x_1, \ldots,
x_n)$\label{noteratnvar}.    

\end{enumerate}


\item
Let $p$ be prime and denote the field of fractions of ${\mathbb Z}_p[x]$
by ${\mathbb Z}_p(x)$.  Prove that ${\mathbb Z}_p(x)$ is an infinite field
of characteristic $p$. 

 
\item
Prove that the field of fractions of the Gaussian integers, ${\mathbb
Z}[i]$, is 
\[
{\mathbb Q}(i) = \{ p + q i : p, q \in {\mathbb Q}  \}.
\]

 
\item
A field $F$ is called a \boldemph{prime
field}\index{Prime field}\index{Field!prime} if it has no proper
subfields. If $E$ is a subfield of $F$ and $E$ is a prime field, then  
$E$ is a \boldemph{prime
subfield}\index{Prime subfield}\index{Subfield!prime} of $F$. 
\begin{enumerate}
 
 \item
Prove that every field contains a unique prime subfield.
 
 \item
If $F$ is a field of characteristic 0, prove that the prime subfield
of $F$ is isomorphic to the field of rational numbers, ${\mathbb Q}$.
 
 \item
If $F$ is a field of characteristic $p$, prove that the prime subfield
of $F$ is isomorphic to  ${\mathbb Z}_p$. 
 
\end{enumerate}

 



%*****************Theory---Factorization**********************


 
 
 
\item
Let ${\mathbb Z}[ \sqrt{2}\, ] = \{ a + b \sqrt{2} : a, b \in {\mathbb Z} \}$. 
\begin{enumerate}

 \item
Prove that ${\mathbb Z}[ \sqrt{2}\, ]$ is an integral domain. 

 \item
Find all of the units in ${\mathbb Z}[\sqrt{2}\, ]$. 

 \item
Determine the field of fractions of ${\mathbb Z}[ \sqrt{2}\, ]$. 
 
 \item
Prove that  ${\mathbb Z}[ \sqrt{2} i ]$ is a Euclidean domain under the
Euclidean valuation $\nu( a + b \sqrt{2}\, i) = a^2 + 2b^2$. 
 
\end{enumerate}


\item
Let $D$ be a UFD. An element $d \in D$ is a \boldemph{greatest common
divisor of}\index{Greatest common divisor!of elements in a UFD} $a$
\boldemph{and} $b$ \boldemph{in} $D$ if $d \mid a$ and $d \mid b$ and $d$ is
divisible by any other element dividing both $a$ and $b$.  
\begin{enumerate}

 \item
If $D$ is a PID and $a$ and $b$ are both nonzero elements of $D$,
prove there exists a unique greatest common divisor of $a$ and $b$ up to associates. That is, if $d$ and $d'$ are both greatest common divisors of $a$ and $b$, then $d$ and $d'$ are associates.  We write $\gcd( a, b)$ for the greatest common divisor of $a$ and $b$. 

%Uniqueness statement qualified.  Suggested by R. Beezer. TWJ 5/15/2012

 \item
Let $D$ be a PID and $a$ and $b$ be nonzero elements of $D$. Prove
that there exist elements $s$ and $t$ in $D$ such that $\gcd(a, b) =
as + bt$.

\end{enumerate}


\item
Let $D$ be an integral domain. Define a relation on $D$ by $a \sim b$
if $a$ and $b$ are associates in $D$.  Prove that $\sim$ is an
equivalence relation on $D$.  


\item
Let $D$ be a Euclidean domain with Euclidean valuation $\nu$.  If $u$
is a unit in $D$, show that $\nu(u) = \nu(1)$.


\item
Let $D$ be a Euclidean domain with Euclidean valuation $\nu$.  If $a$
and $b$ are associates in $D$, prove that $\nu(a) = \nu(b)$.


\item
Show that ${\mathbb Z}[\sqrt{5}\, i]$ is not a unique factorization domain.



\item
Prove or disprove:  Every subdomain of a UFD is also a UFD.


\item
An ideal of a commutative ring $R$ is said to be \boldemph{finitely
generated}\index{Ring!finitely generated} if there exist elements
$a_1, \ldots, a_n$ in $R$ such that every element $r \in R$ can be
written as $a_1 r_1 + \cdots + a_n r_n$ for some $r_1, \ldots, r_n$ in
$R$.  Prove that $R$ satisfies the ascending chain condition if and
only if every ideal of $R$ is finitely generated.  


\item
Let $D$ be an integral domain with a descending chain of ideals $I_1
\supset I_2 \supset I_3 \supset \cdots$.  Suppose that there exists an $N$ such that
$I_k = I_N$ for all $k \geq N$. A ring satisfying this condition is
said to satisfy the \boldemph{descending chain condition},\index{Descending
chain condition} or \boldemph{DCC}. Rings satisfying the DCC are called
\boldemph{Artinian rings},\index{Ring!Artinian} after Emil
Artin\index{Artin, Emil}.  Show that if $D$ satisfies the descending chain condition, it must satisfy the ascending chain condition.
%Exercise corrected.  Suggested by K. Brooks.  TWJ - 5/15/2012
%Exercise corrected.  Suggested by R. Beezer.  TWJ - 8/9/2012


\item
Let $R$ be a commutative ring with identity. We define a \boldemph{
multiplicative subset}\index{Multiplicative subset} of $R$ to be a
subset $S$ such that $1 \in S$ and $ab \in S$ if $a, b \in S$. 
\begin{enumerate}

% 2010/05/18 R Beezer, too many s's, added two \ast to distinguish
 \item
Define a relation $\sim$ on $R \times S$ by $(a, s) \sim (a', s')$ if
there exists an $s^\ast \in S$ such that $s^\ast(s' a -s a') =0$. Show that
$\sim$ is an equivalence relation on $R \times S$.
 
 
 \item
Let $a/s$ denote the equivalence class of $(a,s) \in R \times S$ and
let $ S^{-1}R$ be the set of all equivalence classes with respect to
$\sim$.  Define  the operations of addition and multiplication on
$S^{-1} R$ by
\begin{align*}
\frac{a}{s} + \frac{b}{t} & = \frac{at + b s}{s t} \\
\frac{a}{s}  \frac{b}{t} & = \frac{a b}{s t},
\end{align*}
respectively. Prove that these operations are well-defined on $S^{-1}R$
and that $S^{-1}R$ is a ring with identity under these operations.
The ring $S^{-1}R$ is called the \boldemph{ring of
quotients}\index{Ring!of quotients} of $R$ with respect to $S$.


 
 \item
Show that the map $\psi : R \rightarrow S^{-1}R$ defined by $\psi(a)
= a/1$ is a ring homomorphism.
 
 \item
If $R$ has no zero divisors and $0 \notin S$, show that $\psi$ is
one-to-one.
 
 
\item
Prove that $P$ is a prime ideal of $R$ if and only if $S = R \setminus
P$ is a multiplicative subset of $R$. 
 
\item
If $P$ is a prime ideal of $R$ and $S = R \setminus P$, show that the
ring of quotients $S^{-1}R$ has a unique maximal ideal. Any ring
that has a unique maximal ideal is called a \boldemph{local
ring}\index{Ring!local}.  
 
 
\end{enumerate}



 
\end{enumerate}
}
 
 
 
\subsection*{References and Suggested Readings}
 
{\small
 
\begin{itemize}
 
\item[\textbf{[1]}] %Reference updated - TWJ 8/14/2010
Atiyah, M. F.  and MacDonald, I. G. \textit{Introduction to
Commutative Algebra}. Westview Press, Boulder, CO, 1994.
 


\item[\textbf{[2]}] %Reference updated - TWJ 8/14/2010
Zariski, O. and Samuel, P. \textit{Commutative Algebra}, vols. I
and II. Springer, New York, 1975, 1960. 
 
\end{itemize}
}
 
\sagesection
 

